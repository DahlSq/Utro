\documentclass[fontsize=11pt,a5paper,titlepage=firstcover]{scrbook}

\usepackage{ifxetex,ifluatex}

\ifxetex
    \usepackage{mathspec}
\else
    \usepackage{fontspec}
\fi
\defaultfontfeatures{Ligatures=TeX,Scale=MatchLowercase}

\usepackage{hyperref}
\hypersetup{unicode=true,
            pdftitle={Утро — время учеников},
            pdfauthor={DahlSq (Михаил Баландин)},
            pdfborder={0 0 0},
            breaklinks=true}
%\urlstyle{same}  % don't use monospace font for urls

\usepackage{graphicx,grffile}
\makeatletter
\def\maxwidth{\ifdim\Gin@nat@width>\linewidth\linewidth\else\Gin@nat@width\fi}
\def\maxheight{\ifdim\Gin@nat@height>\textheight\textheight\else\Gin@nat@height\fi}
\makeatother

\renewcommand{\thechapter}{\arabic{chapter}.}\renewcommand{\thesubsection}{\arabic{section}.\arabic{subsection}.}


% Scale images if necessary, so that they will not overflow the page
% margins by default, and it is still possible to overwrite the defaults
% using explicit options in \includegraphics[width, height, ...]{}
\setkeys{Gin}{width=\maxwidth,height=\maxheight,keepaspectratio}

\usepackage{polyglossia}
\setmainlanguage[]{russian}

\setlength{\emergencystretch}{3em}  % prevent overfull lines
\hfuzz=2.5pt
%\providecommand{\tightlist}{%
%  \setlength{\itemsep}{0pt}\setlength{\parskip}{0pt}}



\usepackage{indentfirst}
\newfontfamily\cyrillicfont{Georgia}   % обычный шрифт
\newfontfamily\cyrillicfonttt{iA Writer Duospace} % моноширинный шрифт
\newfontfamily\cyrillicfontsf{PT Sans} % шрифт без засечек
\setmainfont{Georgia}
\setmonofont{PT Mono}
\setsansfont{PT Sans}
\setlength{\parindent}{2em}
\setlength{\parskip}{0pt plus 0.5pt minus 0.5pt}
\pagestyle{plain}
%\raggedbottom
\clubpenalty=1500
\widowpenalty=1500
\exhyphenpenalty=300

\begin{document}

\begin{titlepage}
	\title{Утро — время учеников}
	\author{DahlSq\\(Михаил Баландин)}
	\date{2020}
	\publishers{\texttt{\url{http://fanfiction.balandin.online}}\\
		\texttt{\url{https://ficbook.net/authors/2707630}}}
\end{titlepage}

\frontmatter

\maketitle

\clearpage
\thispagestyle{empty}

\mbox{}

\vspace{2cm}

\hrule

\bigskip

\noindent Трое жеребят из Понивилля, называющие себя Меткоискателями, подросли и повзрослели. Судьба дала им таких наставников, о которых многие могли бы только мечтать… но подобное везение нужно отрабатывать. И они это сделают, ведь наступило их время — время учеников. История, начатая в «Среди Ночи и тех, кто в ней», продолжается.

\bigskip

\hrule

\vspace{2cm}

\noindent Автор обращает внимание читателя на то, что в рамках этой истории события сериала рассматриваются лишь до открытия Школы Дружбы включительно.

\bigskip

\noindent Использованные в тексте цитаты-однострочники, как правило, приводятся без кавычек и отсылок к источникам.

\clearpage

\tableofcontents


\mainmatter

\chapter*{Пролог, события которого аукались одной из Меткоискателей ещё много лет}\addcontentsline{toc}{chapter}{Пролог, события которого аукались одной из Меткоискателей ещё много лет}


--- Плохо!

--- Тьфу!~--- я выплёвываю облачный клок. Встаю на ноги.~--- Сама вижу.

--- Ничего ты не видишь. Плохо не то, что плохо, а что с новыми попытками лучше не становится. Смотри ещё раз!

Прыжок, в два взмаха набор, горизонт. Быстрый крен~--- у~него всё быстро и резко~--- и глубокий вираж буквально на кончике крыла, как на иголке циркуля. Ни капли высоты не потерял. Выровнялся, вскинулся в «кобру» и просто спрыгнул из воздуха на облако.

--- Ну, видела? Пробуй ещё.

Подпрыгиваю. Мне-то пары взмахов маловато, и на этом можно{\ldots}

--- Хватит, выходи!

Нет, нельзя. Специально ведь ограничивает высоту, чтобы я со сваливания мордой в облако приходила, а не чем ещё. Ну тогда{\ldots}

--- Куда собралась?! Влево!

З-з-зараза! Тоже ведь специально! После каждой попытки нижнее крыло только что судорогой не сводит, а ему в одну сторону подавай{\ldots}

Ну ладно. Горизонт, скорости набрать{\ldots} нельзя терять скорость на вираже, лучше уж{\ldots} ой, нет, мне это знать не положено{\ldots} левый крен{\ldots} блин, блин, блин, как сразу вниз-то потащило!.. скорость, скорость!.. как её удержать-то, щас крыло отвалится!.. а, бли-ин!!!

Через крыло влево, кувырок вокруг крыльевой оси, и мордой в облако.

--- Тьфу! Да нереально, блин, такое выкрутить!

Это я зря. Сейчас ведь будет «я с тебя не требую ничего такого, что не мог бы сам» и вот это всё.

--- Именно то, что ты сей момент себе подумала!

--- У тебя было лишних двести лет научиться!

--- И что? Надо будет драться~--- никто у тебя свидетельство о~рождении не спросит!

--- А не драться никак?~--- ой, хорошо, что подруги не слышат.

--- А зачем нужна такая жизнь, в которой не за что драться?

--- Сам-то ты за что?!

--- Прямо сейчас~--- за твоё небо! Ладно, давай по-другому. Держись сзади и выше меня!

Как стоял на облаке, так и провалился сквозь него. Я так не умею, мне сначала подпрыгнуть надо.

Падает. Облако высоко, падать можно долго. Увидел меня над собой, расправил крылья.

Горизонт. Змейка. Быстрая, с глубокими кренами и малыми радиусами. Опять ни капли высоты не потерял. Горизонт. Глубокий вираж. Горизонт. Восходящая бочка?!.. а, это он поток поймал. Петля. И выход прямо на то облако, с которого стартовал.

Опускаюсь рядом с ним.

--- Что видела?

--- Тебе всё перечислить?

--- Всё не надо. Сколько взмахов я сделал?

А{\ldots} э-э-э{\ldots} Ой. Блин. Ни фига́ ж себе.

--- Ни одного{\ldots}

--- Вывод?

--- Ну, на планировании одними перьями выкрутил.

--- Дальше?

--- И вираж тоже.

--- Правильно. Пробуй ещё. На той же высоте и влево.

--- Покажи хоть, как!

--- Смотри{\ldots}~--- сунул своё крыло мне под нос.~--- Нижнее. Вот так{\ldots}~--- перья топорщатся и выворачиваются в такое, что даже смотреть больно.

--- Уй-й-й-ё-ё-ё{\ldots}

--- Хочешь что-то сказать?

--- Не хочу{\ldots}

--- Добре. Не надо болтать, надо делать. Ну?

Прыжок, набор, горизонт, левый крен. Быстрее, пока скорость есть{\ldots} как он там извращался-то{\ldots} Бли-и-и-н, да как же такое возможно-то! Ой{\ldots}

Всё равно не получается, конечно. Ну, хоть на этот раз кувыркать не стало, просто влево через крыло перебросило. И не мордой в облако, а на ноги{\ldots} ну ладно, на колени.

--- Вот, лучше. Отрабатывай. Через десять дней покажешь, а~там видно будет.

«Лучше»~--- это хорошо. Это пока самое большое одобрение, которое мне доводилось слышать. Тётя Эпплджек, наверно, сказала бы, что когда он меня по-настоящему похвалит, в лесу что-то сдохнет.

Ничего. Никуда не денется.

Потому что я научусь. Сдохну, но научусь.


\chapter*{Глава первая, в которой Меткоискатели выясняют, чт\'о всему этому предшествовало}\addcontentsline{toc}{chapter}{Глава первая, в которой Меткоискатели выясняют, чт\'о всему этому предшествовало}

--- Ну чё, хвастайтесь,~--- предложила Эппл Блум, которой в~этом деле волей-неволей пришлось играть главную роль.

--- Нечем особо{\ldots}~--- вздохнула Свити Белль.~--- Есть три штуки, но всё не главное, а так, мелочи{\ldots}~--- Она выудила из своей сумки три свитка, положила их на стол и поочерёдно указала копытцем:~--- То, которое Луна написала Твайлайт, ну, с просьбой провести всё на ваших задворках. Старлайт его тогда видела и сделала мне копию, там ничего особенного и нету. То, которое Твайлайт написала Луне после самого́ «эксдерьмента», то есть на са́мом деле продиктовала. Это вообще легко было раздобыть, Спайк же его перед отправкой зачитывал вслух при Старлайт, мне и Флаттершай. Ну и, понятно, то, что я сама потом Луне написала по твоей просьбе. Ещё легче, и оно из этих трёх самое существенное, но всё равно мелочи. Главного-то нету, с которого всё началось, а без него всё ерунда.

--- А чё, Старлайт его разве не{\ldots}

--- Нет. Ей тогда не показывали, идея-то слишком смелой была, а уж когда все увидели, чем оно закончилось{\ldots} Спрашивать такое в Кантерлоте слишком нагло, а в за́мке Твайлайт копия если и хранится, то в какой-то личной нычке, куда вообще никому нет доступа. Я, конечно, спрашивала~--- говорит, что искала, но не нашла.

--- Фигово.

--- Не совсем,~--- заметила Скуталу.~--- Я кое-что раздобыла{\ldots} всего одно, но реально важное. Только это{\ldots} как-то оно сомнительно мне{\ldots}

--- А чё такое?

--- Да блин{\ldots} Дело-то серьёзное. Уверена, что нам за эти расследования{\ldots} э{\ldots} кру́пы на уши не натянут? Это ж не фигня какая вроде разбитой вазы Селестии, её же обхаянной выпечки или там кокнутого витража с одолением Найтмэр!

--- Ты за витраж-то отбрехалась?~--- поинтересовалась Свити.

--- Ага. Фигня вопрос, говорю же. Досадный инцидент на тренировке, не рассчитала маленько, дело житейское.

--- Тренировка? А воздушные патрули над дворцом?

--- А что, были?!~--- Скуталу сделала удивлённые глаза.~--- Не, не заметила. Если были, логично с них и спрашивать.

--- А на возмещение ущерба тебе не намекали?

--- Намекали. Но через полчаса Ночная стража выкатила петицию, что они сами вскладчину соберут за меня сколько надо. При условии, что там вместо одоления Найтмэр чего другое будет.

--- Кхм! Мы вообще-то о деле говорили,~--- напомнила Эппл Блум.

--- Так и я о деле. За меня-то с кокнутым витражом вписались, а за нас с этим расследованием кто впишется? Какое-то странное задание она тебе дала, совсем непонятное{\ldots}

--- Чё непонятного-то. Всё логично. Сказала, шо раз я «Очень Большую Игру» написала и твоё,~--- Эппл Блум кивнула на Свити,~--- ученичество раскопала, то теперь надо и про Большое Откровение писать, и про то, как я её писала, хотя бы самое интересное, а заодно и про собственное ученичество{\ldots} ну, с чего оно начиналось-то. А оно ж тем «эксдерьментом» и началось! Ещё говорила, шоб я сильно не парилась{\ldots} типа наверху в общих чертах в курсе, а результат она сама смотреть будет, и если не пойдёт, то сама же и перепишет. Как-то так.

--- Блин! Так что ж ты сразу-то не сказала?! Если там в курсе, так всё ж через Свити решаемо! «Дорогая принцесса Луна», все дела. Уж после того витража-то!

--- Думаешь?

--- Уверена! Если так, то вот{\ldots}~--- Скуталу вытащила из своей сумки толстый свиток, за который тут же ухватилась Свити Белль.

--- Эт’ чё?~--- поинтересовалась Эппл Блум.

--- Письмо, которое Селестия накатала Твайлайт в ответ на то{\ldots} ну, самое первое и главное.

--- Ого! Ты-то его откуда?!

--- А его, оказывается, Луна видела. Ну как видела, прямо через плечо читала по ходу написания. И моему потом излагала. А~он мне записал, как я попросила. Уточнила, конечно: мол, если это засветить, то нам за это ничего не будет? А он в своей манере: мол, что будет, то всё ваше{\ldots} Я почему и спросила.

--- Она, оказывается, и исходное письмо, вернее, статью видела{\ldots}~--- пробормотала Свити, быстро читая текст.~--- Значит, это тоже можно будет запросить{\ldots}

--- Ну, совсем-то наглеть не надо,~--- заметила Скуталу.~--- Тут столько цитат, что по ним всё легко понять{\ldots} вон, даже до меня всё дошло. Опять же, тут нормальным понячьим языком написано, а та статья{\ldots} не факт. Это ж Твайлайт. По мне, так вполне пойдёт за отправную точку.

--- Давайте вслух уже!~--- потребовала Эппл Блум.~--- Свити, читай ты.

--- Чего я-то?

--- Дак ведь про магию, кто ж ещё-то{\ldots}

Свити откашлялась и начала.

%\vspace{2mm}
\begin{center}
	* * * * * *
\end{center}
%\vspace{2mm}

Дорогая Твайлайт!

Никакие слова не могут передать моей радости от осознания того факта, что ты всё ещё продолжаешь считать меня своим научным копытоводителем. Просто прими это как данность. Твои достижения~--- как в части теории, так и в части практики~--- настолько велики, что подобное отношение с твоей стороны сделало бы честь кому угодно.

Равным образом, мне отрадно видеть, что дерзость твоего исследовательского ума ничуть не притупилась. При всех твоих обязанностях принцессы Эквестрии и директора Школы Дружбы --- а я лучше кого бы то ни было понимаю их обременительность! --- ты продолжаешь находить время и возможности для научной работы, что просто замечательно. Любезно присланный тобой черновик статьи свидетельствует об этом со всей очевидностью.

Должна, однако, напомнить, что наука требует беспристрастности, и коль скоро ты видишь меня своей наставницей, я в отношении статьи обязана такую беспристрастность проявить. Ситуация осложняется ещё и тем, что избранная тобой тема затрагивает не только научные, но и определённые морально-этические аспекты, к которым мы, как принцессы, должны быть особенно внимательными. Но давай обо всём по порядку.

Вводная часть статьи великолепна. Работа с источниками всегда была твоей сильной стороной, и аналитический обзор гипотез, пытавшихся логически объяснять и обосновывать эквестрийскую магию, удался тебе блестяще. Моя люб{\ldots}имая и люб{\ldots}ознательная и люб{\ldots}ЯЩАЯ ПОДГЛЯДЫВАТЬ ЧЕРЕЗ ПЛЕЧО сестра присоединяется к этому мнению и благодарит тебя за доставленное веселье. Действительно, некоторые из этих гипотез, предложенные за время её вынужденного отсутствия, довольно забавны.

Что же до меня, то я не поленилась сделать для себя копию этого фрагмента --- редко можно встретить столь качественный материал, заключённый в столь небольшом объёме. Перейдём, однако, к основному содержанию.

Ты постулируешь существование «наноботов» как «высокотехнологичных крошек», которыми заполнено всё окружающее нас пространство. Мне не очень понравился этот термин, и~сначала я решила было, что ты хочешь ввести квантование для «магического поля» (гипотезу которого ты честно рассмотрела и, будем называть вещи своими именами, справедливо высмеяла). Позже я увидела, что это не так, и хочу посоветовать тебе переписать соответствующий кусок текста, дабы отличие сразу же бросалось в глаза. Это стоит сделать, поскольку твоя гипотеза конкретизирована намного аккуратнее, нежели «магическое поле», и в нынешнем виде описание ввело меня в заблуждение. Что же до названия --- пусть пока будут «наноботы», поменять это слово при необходимости недолго, а в обсуждении оно удобно своей краткостью.

Итак, наноботы. Согласно твоим рассуждениям{\ldots}

\emph{«Наноботы способны на простейшие самостоятельные действия:} \textbf{левитация} \emph{относительно произвольно выбранного центра (в том числе и с отрицательными величинами силы притяжения), чем достигается абсолютная свобода перемещения;} \textbf{механическое взаимодействие} \emph{с соседними наноботами при произвольной степени жёсткости взаимодействия и силе разрыва связей;} \textbf{информационное взаимодействие} \emph{с ближайшими наноботами посредством частотно-модулированного излучения (предположительно, в~видимом диапазоне, но возможно наличие и второго канала с~более высокой частотой)».}

Три функции. Это немного, но позволяет многое объяснить, а потому твоя гипотеза вполне отвечает классическому «не умножай сущности сверх необходимости». Мне понравилась формулировка \emph{«активация ключевых команд, или так называемое колдовство, способно запускать очень сложные последовательности внутреннего взаимодействия наноботов, ведущие к заранее обусловленному результату»}. Кажется, тебе удалось придумать вполне энциклопедическое определение для термина «заклинание».

Здесь же находится фрагмент, который, при всём моём непредвзятом и критическом отношении, вызвал у меня самый настоящий восторг: \emph{«наличием наноботов обусловливается появление так называемых кьютимарок, они же метки или знаки отличия~--- наноботы, выстраивая свою колонию под поверхностным эпителием носителя, создают ясно различимый цветной рисунок»}. Я пока далека от того, чтобы безоговорочно принимать твою гипотезу, но это место представляется чрезвычайно многообещающим. Пожалуйста, продолжай размышлять в этом направлении~--- мне очень хотелось бы прочитать больше, нежели \emph{«механизм создания конкретного рисунка неизвестен, однако без сомнения можно утверждать, что он связан с~мыслительными процессами носителя»}.

Следующее рассуждение надолго озадачило меня. Вот оно:

\emph{«Наноботы, активируясь при управляющем воздействии, включают сигналы безопасности, дабы обозначить зону своей активности, упростить наведение и разграничить влияние. Этим обусловлено свечение области вокруг предмета взаимодействия (цвет обычно зависит от масти оператора). Также подсвечивается управляющая антенна и канал связи между антенной и предметом воздействия».}

Не слишком ли много усилий для объяснения того факта, что мы можем видеть магические ауры? Но потом~--- не буду скрывать, как раз при написании этого письма~--- мне пришла в голову интересная мысль. \emph{А для кого, собственно, эти «сигналы» предназначаются?} Если для мага («оператора» по твоей терминологии), то объяснение действительно выглядит притянутым. Но если допустить, что «сигналы» адресованы другим наноботам?! Здесь открываются новые горизонты для рассуждений. По твоим представлениям, магическая функция наноботов никак не привязана к личности «оператора», и тогда все эти «сигналы» могут быть частью сложного механизма установки приоритетов и реализации \emph{совместной} магии.

На твоём месте я переписала бы эту часть статьи. Если, говоря «разграничить влияние», ты имела в виду что-то вроде вышесказанного, то это стоит выразить более конкретно; в противном же случае подумай ещё над тем, \emph{что именно} ты хотела сказать.

Теперь я с тяжёлым сердцем перехожу к грустному моменту. Ты попыталась с точки зрения своей гипотезы объяснить явление телепортации, и это вполне логично. Но \emph{как} ты это сделала?

\emph{«Особым случаем работы наноботов является режим телепортации. Для управления данным режимом необходим штыревой управляющий элемент и драйвер управления телепортом, который невозможно использовать оператору без подготовки (согласно одной из теорий, далеко не все операторы способны его правильно ассимилировать). При телепортации тело или объект разбирается наноботами на составные части и тут же собирается из наличных атомов в другом месте. Режим работы телепорта может быть как мгновенным --- всё тело разбирается или собирается разом~--- так и длительным, при котором образуется видимая граница зоны действия телепорта (известная как портал)».}

Тебе не кажется, что во второй фразе ты слишком переусердствовала? Это можно было выразить гораздо проще и короче: телепортироваться могут лишь единороги, и то далеко не все способны этому обучиться. Но если бы дело было только в этом{\ldots}

Дорогая Твайлайт! Начиная с этого места, тебе следует смотреть на моё письмо так, будто оно написано не представителем академических кругов, анализирующим твою статью, а прежде всего принцессой Эквестрии. Возможно, тебе следует отвлечься на минуту от чтения, выпить воды или сделать ещё что-то в этом роде.

Тебе, вероятно, известно, что в нашей стране существует немало, скажем так, философских концепций, последователи которых исходят из довольно расплывчатого понятия «душа». Представители академических кругов обычно именуют их коротким словом «секты», однако я полагаю, что дружба со Старлайт Глиммер (передавай ей мой привет) отучила тебя от необдуманного употребления этого термина.

Пожалуйста, пойми меня правильно. Лично я ничего не имею против твоего объяснения феномена телепортации: оно логично и хорошо укладывается в рамки обсуждаемой гипотезы. Но поверь моему опыту: если с этим объяснением ознакомятся те, кто верит в существование «души», они незамедлительно поднимут множество крайне неудобных вопросов. И самым неудобным из них будет вопрос о том, сохраняется ли «душа» пони при «разборке» тела, почему она должна восстанавливаться при его воссоздании и, как следствие, \emph{можно ли считать тех, кто хоть однажды проходил через телепорт, настоящими пони}. Думаю, ты без всякого труда проведёшь все необходимые исторические параллели.

Я не знаю, что сказать тебе в этой связи. Правда, не знаю. Просто полагаюсь на твой здравый смысл и выражаю надежду, что ты посоветуешься со мной перед тем, как предпринимать какие-то практические шаги.

Наверняка предыдущие три абзаца расстроили тебя. Постараюсь немного утешить и снова надену маску представителя академических кругов, дабы похвалить.

\emph{«Следующий режим работы наноботов относится к так называемым полётам: наноботы при этом не включают сигнальных огней, так как воздействуют непосредственно на тело управляющего ими. Для данного режима требуются фазные управляющие элементы, известные как крылья,~--- на самом деле никакого аэродинамического функционала эта система нести не может, о чём свидетельствует их конструктив: нередко они имеют недостаточный размер, либо вовсе имеют отверстия. При раскрытии элементов управления наноботы создают поле сил, по своему воздействию напоминающее плотную среду, однако с минимальным сопротивлением движению».}\looseness=-1

Тебе удалось дать хорошее объяснение тому факту, что полёты пегасов тоже имеют в своей основе магию. Это действительно удачное место. Оно логично вытекает из всего сказанного ранее и хорошо согласуется с тем, что́ мне подумалось насчёт «сигнальных огней». В самом деле, при индивидуальности полёта отдельно взятого пегаса нет никакой необходимости в механизмах расстановки приоритетов и протоколах взаимодействия.

Пожалуй, здесь не хватает объяснений для двух других магических способностей пегасов~--- хождения по облакам и управления погодой~--- но с этим я не предвижу ни малейших затруднений (думаю, ты просто увлеклась и в исследовательском азарте выпустила их из вида). Пожалуйста, обрати на это внимание.

Следующая часть твоей статьи, как я сразу подумала, должна была быть особенно интересной. Тебе не хуже моего известно, насколько неудовлетворительно все существующие гипотезы объясняют связь земных пони с магией, отделываясь лишь самыми общими и невнятными формулировками.

Зная тебя, как свою лучшую ученицу, я практически не сомневалась, что ты предложишь объяснение, столь же оригинальное, сколь и неожиданное. И ты сделала это{\ldots} о да, сделала{\ldots}

Говорила ли я тебе, что ты обладаешь редким талантом удивлять меня сверх всяких мыслимых границ? Вопрос чисто риторический~--- я не перестаю говорить об этом с тех самых пор, как ты буквально посадила меня в лужу на уроке. Однако нынешний случай \emph{далеко} превзошёл все предыдущие.

\emph{«Наноботы созревают в земле и далее образуют плодовое тело в виде камня, который для активации должен пройти через пищеварительный тракт земного пони. В процессе активации освобождается избыточная энергия, чем и обусловлена физическая сила земных пони, значительно бо́льшая по сравнению с остальными двумя расами».}

У меня нет слов, Твайлайт, и это отнюдь не фигура речи. Всё, что написано далее, опять следует воспринимать как написанное не наставником, но принцессой Эквестрии.

Твоё рассуждение убийственно. Оно закрывает множество слабых мест в других гипотезах, даёт объяснение некоторым загадочным фактам (вроде того, зачем земным пони нужна способность выращивать и поедать камни, пусть даже в наши дни этим занимаются немногие) и значительно усиливает роль земных в эквестрийском социуме. Вместе с тем, оно \textbf{абсолютно неприемлемо}.\looseness=-1

Я говорю даже не об эстетическом аспекте~--- замени в своём тексте «пищеварительный тракт» на «желудочно-кишечный», и~ты поймёшь (надеюсь)~--- хотя и это немаловажно. Понимаешь ли ты, дорогая Твайлайт, какие последствия могут возникнуть, если ты опубликуешь \emph{такое}? Если нет, то позволь тебе обрисовать.

Знакомство широких масс с этой теорией немедленно вызовет всплеск шовинизма среди земных пони. Ситуация немедленно осложнится вспышками депрессии и фрустрации среди единорогов: если бы ты попробовала объяснить своей подруге Рэрити, что её магия, порождающая несравненную красоту, в свою очередь, порождается функционированием чьего-то желудочно-кишечного тракта, то получила бы вполне отчётливое представление. (Пожалуйста, обрати внимание на то, что я использовала сослагательное наклонение! «\emph{Если бы} попробовала», а не «попробуй», и я не сомневаюсь, что у тебя достанет воображения додумать остальное!) Пегасы, наверное, пережили бы такое откровение без особых проблем, но намного легче от этого не стало бы. Вспомни теперь пьесу о прошлом Эквестрии, которую ты и~твои подруги ставили в Кантерлоте на День Согревающего Очага --- и~поверь, что показанные там события на фоне описанного мною покажутся именно что детской пьесой.

Дочитав до этого места (а возможно, даже ещё и не дочитав), ты наверняка задалась вопросом о том, как же тебе поступить с~этой интересной и многообещающей гипотезой. Как я уже не раз повторила, \emph{я не знаю, что тут сказать}. Обрати, однако, внимание на то, что всюду в этом письме я использовала исключительно первое лицо \emph{единственного} числа: «\textbf{я} бы хотела», «\textbf{мне} интересно» и так далее. Поверь, я~сделала это не зря. В~свою очередь, я верю, что ты поймёшь это правильно и сделаешь нужные выводы.

Твоя уже далеко не молодая и очень желающая прожить остаток жизни спокойно наставница,~---
\begin{flushright}\textbf{С.}\end{flushright}
P.S. Едва я поставила свою подпись, как моя любимая и~проч. сестра предложила интересную (по её мнению) методику верификации твоей гипотезы. Идея заключалась в том, чтобы создать два фаерболла с одинаковыми магозатратами в~предполагаемых местах пониженной и повышенной концентрации наноботов, дабы по соотношению их видимых размеров оценить реальность и масштабы того самого предполагаемого освобождения избыточной энергии. В качестве места \emph{пониженной} концентрации она назвала собственные покои, и с этим трудно спорить (за последнюю тысячу с небольшим лет там определённо не практиковали камнеедение). Но когда местом \emph{повышенной} концентрации она назвала канализационный коллектор рабочих кварталов Кантерлота{\ldots} Дорогая Твайлайт, что позволено моей сестре (в конце концов, метан и его горючие свойства были изучены лишь около шестисот лет назад во время её отсутствия), то ни в коем случае не позволено исследователю с самым лучшим на данный момент образованием. Видишь, к~каким неожиданным и опасным последствиям может привести необдуманное распространение твоей гипотезы?!

P.P.S. Я рассказала сестре о свойствах метана, могущих внести серьёзную погрешность в предлагаемый ею эксперимент. А затем дала понять, что в случае чего экспериментатор будет иметь возможность на собственной шкуре проверить гипотезу о влиянии лунного вакуума на выведение с оной шкуры пятен и запахов. Если ты не заметила, я только что дала понять это и~тебе.

\begin{center}
	* * * * * *
\end{center}
\vspace{2mm}

--- Фу-у-у!.. --- поморщилась Свити Белль, дочитав до конца.~--- Это ж надо было додуматься!

--- Ты про кого щас?

--- Про Твайлайт, само собой! Я бы на её месте такое тоже заныкала, чтоб никто никогда не увидел.

--- А чё, прикольно! --- хихикнула Эппл Блум.~--- Ты только сестре своей не рассказывай, а то она Твайлайт за такое точно убьёт, про это правильно написано. Скутс, а ты чё скажешь?

--- В каком смысле?

--- Ну, в том, что полёты{\ldots} э-э{\ldots} на этом самом основаны. Ну, ты ж понимаешь.

--- А что такого? Жук-пуки на этой тяге точно летать могут, вон, Свити не даст соврать{\ldots}

--- Тебе тоже приятного аппетита!

--- Ладно, кончаем хиханьки. И впрямь доходчиво, чо. Ну, теперь понятно хотя бы, откуда у того «эксдерьмента» ноги растут. Свити, а ты говорила, шо там чё-то потом ещё было насчёт просьбы обделать это{\ldots} тьфу!.. я хотела сказать, устроить всё на нашей ферме?

--- Ага. Вот это,~--- Свити Белль указала копытцем на свиток.

--- Давай тогда, читай уж и его{\ldots}

%\vspace{2mm}
\begin{center}
	* * * * * *
\end{center}
%\vspace{2mm}

Твайлайт!

Не помню уже, говорила ли я тебе, что моя сестра жуткая перестраховщица --- но даже если и не говорила, то ты теперь сама видишь. Никакой, блин, любознательности уже не осталось, одно сплошное «как бы чего не вышло», тьфу!

Нет, ну, конечно, насчёт этого вашего метана она вообще-то права. Я так подумала --- оно и впрямь не след, фаерболлами по канализациям кидаться. Это даже и не по-королевски как-то. Между нами говоря, этот ваш метан, когда я про него почитала, презабавной штукой оказался --- это же просто шикарные возможности по боевому применению открываются! Вот например, можно взять{\ldots} хотя нет. Не на бумаге и не через почту. Видишь, вот уже и я от сестрицы нахваталась какбычегоневышлости, так что не вздумай ещё и ты (это тебе наше высочайшее повеление, если что).

Короче. Хоть насчёт канализации сестра и права, но твою шикарную гипотезу без проверки оставлять всё равно нельзя, и~не обращай внимания на эти намёки с угрозами~--- в любом случае, дальше Луны не сошлют (шутка). Так что я подумала и~придумала, про Луну как раз.

Сколько лет я своё светило таскаю --- напоминать не нужно? (Да я бы и не напомнила, возрастом не хвастаются.) Я к тому, что усилия контролировать и дозировать уж точно умею, можешь не сомневаться. Идея, стало быть, такая: я здесь у себя выбираю силу и на этой выбранной силе светило поднимаю. Документируем угол возвышения как функцию времени --- секстант ещё и в~прежние времена известен был, а секундомеры нынче хорошие делать научились. Это, значит, в условиях пониженной концентрации наноботов.

Собственно, то что я сейчас написала, всё уже проделано. Между нами опять же говоря, эта ваша Школа одарённых единорогов оказалась каким-то сборищем тупиц --- пошла я туда, попросила порекомендовать кого поспособнее (ну, думаю, сейчас мне кого-нибудь вроде тебя и порекомендуют). Мне и порекомендовали одного, а он дуб дубом оказался. Прикинь, сорок минут втолковывала ему суть предыдущего абзаца --- монаршим, между прочим, голосом! --- пока дошло. Ну, без уточнения про наноботов и концентрацию, конечно. Я даже сестру как-то немного пожалела: это ж ей, бедняге, в таких условиях пришлось целую тыщу лет тебя искать против меня{\ldots}

Но это неважно, а теперь главное. Пока я ещё держу в памяти то усилие, на котором поднимала в Кантерлоте, надо проделать всё то же самое где-нибудь в месте повышенной концентрации и потом замеры сравнить. Вот здесь-то мне и нужна твоя помощь.

Нет, в том, что ты сможешь снять зависимость возвышения от времени, я нимало не сомневаюсь. Проблема в том, что это самое место повышенной концентрации ещё найти нужно, и~тут я на тебя рассчитываю. Как я и говорила, в кантерлотскую канализацию я по здравому размышлению не полезу, ибо сие не по-королевски. Но ежели память нам не{\ldots} тьфу, если я правильно помню, то среди твоих понивилльских подруг есть потомственная фермерша из древнего рода земных пони --- Эпплджек, правильно? --- а если я правильно понимаю консервативность фермерского ремесла, то эта семья должна располагать у себя на ферме{\ldots} скажем так, эквивалентом канализационного коллектора в несколько меньших масштабах?

Пожалуйста, поговори с подругой и спроси её разрешения на использование сего прозаического заведения в высоких научных целях. Не сомневаюсь, что разрешение это ты получишь, ибо сама заинтересована в таком эксперименте, а посему красноречие твоё да будет использовано во всю силу. Думается, нелишним будет такоже особо прояснить фермерше Эпплджек необходимость держать язык за зубами, хотя, судя по твоим рассказам, с этим у неё и так всё в порядке. При надобности можешь привлекать и других своих подруг --- в общем, вверяю дальнейшую судьбу эксперимента в твои копыта.

Относительно моей сестры можешь не беспокоиться: она, конечно, зубами поскрипит, но супротив изложенного выше возражать не станет. Это я беру на себя, так что будь спокойна. Главное, чтобы было упомянутое разрешение --- ставить заведомо безопасный эксперимент на частной территории с ведома и согласия её законного владельца запретить никто не в силах, даже моя сестра.

С нетерпением ждущая твоего ответа ---
\begin{flushright}\textbf{Л.}\end{flushright}
\begin{center}
	* * * * * *
\end{center}

--- Во-от! --- торжествующе сказала Скуталу. --- Секстант! Секстант, а не{\ldots} то, что ты тогда написала.

--- То, что \emph{ты} мне тогда сказала, между прочим. Меня за это кто уж только не тюкал{\ldots}

--- Я тебе просто сказала, как это легче запомнить! Меня, между прочим, тоже тюкали, когда ты сказала, что это я сказала. Твоя сестра пыталась.

--- А ты?

--- Напомнила ей про то едрёно сено, она и увяла.

Свити Белль хрюкнула.

--- Лучше на всякий случай напомни, чё именно этим секстантом делают,~--- попросила Эппл Блум.

--- Углы на небе меряют. Они хотели два раза поднять Луну, на одинаковом усилии, но в разных условиях. И замерить, не будет ли в одном случае подъём быстрее, чем в другом. Теперь понятно, почему именно на ваших задворках хотели.

--- Эт’ я и сама поняла уже. Бред какой-то, но логичный.

--- Так Твайлайт же.

--- Ото ж. Значит, сначала было её письмо или статья, которую она старшенькой посылала и куски которой мы тут щас зачли с ответными комментариями. Потом младшенькая придумала эксперимент, но ей в Кантерлоте не разрешили, и тогда она надумала его тут у нас обделать{\ldots} тьфу, блин, вот же ж слово привязалось! Написала, значит, Твайлайт письмо, вот то самое, шо мы щас тож зачли. Чё там у них было дальше, мы своими глазами видели, я тогда ещё тот репортаж сочинила, и текст его мне давно восстановили. Вот он у меня здесь, ща тоже зачту, шоб память освежить{\ldots}

Эппл Блум потянулась к своей сумке, но её вдруг остановила Свити:

--- Может, лучше завтра? Поздно уже.

--- И чего?~--- удивилась Скуталу.~--- Чай, не маленькие уже. Ну и припозднимся немного, на ферме же, а не в лесу. Хочешь, я метнусь твою сестру предупредить?

--- Да не это! Ночь скоро. Если я сейчас напишу письмо с~просьбой, то е́сть шансы, что к утру она уже ответит. Тогда завтра с утра и сели бы дальше разбираться.

--- А! Дело говоришь. Ну, тады до завтра.

--- Только это{\ldots} вы первыми выход\'ите, ладно? --- смущённо сказала Свити.~--- Хочу попробовать отсюда домой телепортироваться{\ldots}

--- И чё?

--- Пока ещё плохо получается. Стесняюсь{\ldots}



\chapter*{Глава вторая, в которой Меткоискатели вспоминают, как это тогда происходило}\addcontentsline{toc}{chapter}{Глава вторая, в которой Меткоискатели вспоминают, как это тогда происходило}

Утром следующего дня Свити Белль сияла так, что всё было ясно без слов. Эппл Блум только и сказала:

--- Выкладывай уже.

--- Ещё три,~--- выложила и доложила единорожка.~--- Из них два очень важных, это те, которыми они меж собой под самый конец обменялись.

--- А третье?

--- Третье на самом деле из них первое. То, что Луна написала Твайлайт сразу после эксперимента, через пару часов буквально. Сейчас зачтём твой репортаж, а потом сразу и его, чтобы в точности всё как было. Говорила же я вчера, что лучше до утра отложить{\ldots}

--- Ладно, ладно, мы поняли уже{\ldots} Так чё, читать?

--- Ну да.

--- Щас{\ldots}~--- Эппл Блум развернула на столе бумагу с отпечатанным на машинке текстом.~--- Свити, ты-то видела ещё в черновике{\ldots} Скутс, а ты учти, шо я это написала для Бэбс, в Мэйнхэттен. Потому что{\ldots} а, хотя там же прямо в тексте говорится. В~общем, слушайте.
%\vspace{2mm}
\begin{center}
	* * * * * *
\end{center}
%\vspace{2mm}

Привет, Бэбс!

Я вообще тебе хотела только на той неделе писать, но после того, что вчера случилось, решила отписать прямо щас. Пока оно ещё хорошо помнится, а то ведь всей нашей семьи касается. Ты там расскажи потом у себя всем, ладно?

Случился тут у нас, если коротко, такой трындец, шо просто шухер. Это бабуля так выразилась, и оно правильно, но не очень понятно, поэтому я щас тебе подробно распишу, а ты постарайся всё представить, как оно было.

Значит, началось всё в обед. Сели мы за стол, а сестра такая говорит: никому, мол, не наедаться и не объедаться, чтоб весь вечер в сортире никто не сидел и он свободным был. Бабуля, натурально, тут же поинтересовалась, с какой бы это стати, а сестра такая: мол, принцессы Луна и Твайлайт Спаркл хотят в нём какой-то важный научный эксперимент делать. Ну, хотят и хотят, я, конечно, ни разу не поняла, почему именно в нашем сортире, но они ж принцессы, им, наверно, видней.

Вот. Пообедали, выходим~--- ага, это не только нам сказали, тёте Пинки Пай тоже говорили, и по-моему, зря. Потому что она все задворки вокруг будки нам обтянула какой-то красно-белой полосатой лентой, а сама скачет вдоль неё кругами и верещит: «пожалуйста, разойдитесь, здесь нет ничего интересного!!!». На весь Понивилль, а бабуля сказала, что и в Кантерлоте должно быть слышно. Я ещё подумала, что уж это бабуля загнула~--- а нет, не загнула, потому как к вечеру из этого самого ихнего Кантерлота целая куча корреспондентов понаехала. Наверно, и впрямь услышали. Ну, про них ещё речь впереди.

Где-то полчаса у нас в ушах звенело, а потом прилетела тётя Флаттершай со здоровенной клизмой (ей, похоже, тоже кто-то что-то сказал про сортир на весь вечер, но видимо чего-то перепутал или она сама недопоняла), тогда тётю Пинки срубило от смеха и она уползла рыдать и успокаиваться в свой «Сахарный уголок». Прикинь, я ж вообще не думала, что её можно вот так смехом вырубить, она ж по этой части сама кого хошь может!

Ну, потом девчонки подтянулись, я им рассказала чего от сестры услышала, потом мы до вечера маялись~--- интересно же, но ничего не понятно. А как солнце к закату спускаться начало, тут оно всё по-настоящему и поехало.

Сначала к нам принцессы Селестия (я её даже не ожидала) и Луна телепортировались, потом ещё принцесса Твайлайт Спаркл с какими-то хитрыми инструментами из своего замка. Ну то есть один инструмент был просто секундомер, нас по такому же на физре гоняют, а про второй когда сестра услышала как он называется (что-то вроде какого-то «секс-та́натоса», кажется, точно не помню уже), то начала разоряться, чтобы при детях неприличными словами не выражались, но принцессы её успокоили.

Потом, собственно, куча кантерлотских корреспондентов подвалила и даже один местный объявился~--- принёсся Фезервейт со своим фотиком (не знаю, помнишь ты его (Фезервейта, не фотик) или нет). Тётя Рэйнбоу Дэш его с полчаса по небу шугала, дядя Виндчейзер (это к которому Скутс пошла в ученицы, я тебе писала) ещё съехидничал, что выполнять перехват в~воздухе совсем не то, что балеты фигурять, а тётя Дэш услышала и немного выбесилась, но всё равно не выгнала и не поймала.

А потом мы с девчонками спохватились, что время-то идёт, а~ничего так и не понятно, и пристали к принцессе Селестии якобы от школьной газеты. Хотя нас из неё попёрли ещё когда{\ldots} ой, а~ведь аккурат тогда, когда мы принцессу Селестию, жрущую тортик, в той газете и пропечатали! Может, она поэтому на нас так косо и посмотрела. Особенно на Свити.

Я вообще давно подозревала, что Свити там в кантерлотском дворце чего-то наколбасила, ещё в тот раз, когда её туда из Холлоу-Шэйдс говорящим письмом отсылали, помнишь, я~рассказывала? Подозревала, а теперь почти уверена, хотя неважно.\looseness=-1

Значит, спросили мы принцессу, а она покосилась, скривилась, но всё-таки ответила. Буквально через губу. Что, дескать, принцесса Твайлайт Спаркл придумала каких-то онано{\ldots} то ли обормотов, то ли идиотов, я толком не расслышала, а принцесса Луна придумала с ними какую-то затею, которую в сортире можно делать, ну и вот. Тут опять моя сестра чего-то начала возбухать про детей и выражения, а принцесса Селестия на неё покосилась ещё косее и выразилась уже совсем через губу. Что, мол, эти дети хороши только пока маленькие, максимум в лужу посадить могут, а как вырастут, так ведь не успокоятся, пока на всех граблях танец не станцуют (это я совсем не поняла). И смотрит почему-то на принцессу Твайлайт Спаркл, ладно хоть не на нас. Сестра, по-моему, тоже ничего толком не поняла, но замолчала. А принцесса Селестия потом что-то ещё про какую-то сказку сказала, но это я мимо ушей пропустила, потому что как раз в этот момент принцесса Луна в будку залезла.

Я ещё было подумала, что может же не влезть, потому что будку эту брат делал по бабулиным размерам, но ничего, влезла, только с кряхтением, и даже дверь закрылась. А там же в двери есть такая дырка-сердечко, ну ты сама знаешь, как это делают, и~из этой дырки такие глаза хлоп-хлоп-хлоп. Так прикольно! Может, у какого корреспондента фотка найдётся и её в газете пропечатают, хотя если вспомнить, как нас за ту фотку с тортиком вздрючили, то может и не напечатают. Но я всё-таки надеюсь, и вы тоже в газеты посматривайте.

Вот. А потом оно началось. Сначала даже как-то и неинтересно было, я чуть не разочаровалась. Солнце, значит, село, начала Луна подниматься. В дверной дырке глаза хлоп-хлоп, а~принцесса Твайлайт тоже этак глазами: одним смотрит в секундомер, а другим зыркает на этот свой секс-танатос, или как его. И бубнит чего-то, а Спайк за ней на бумагу строчит. Я~прислушалась~--- цифирь какая-то. И так это, вроде быстрее и быстрее бубнит, я~сначала решила что мне кажется, но прислушалась~--- нет, не кажется. Спайк уже даже кряхтеть и~попискивать начал.

А глаза в дырке хлоп-хлоп, а потом принцесса Луна из будки вдруг как заорёт: «Твайлайт! Ничё не понимаю! Она уже сама прёт, щас энергия в обратку хлынет!». Принцесса Твайлайт Спаркл, видать, сразу сообразила, чего эта абракадабра значит, потому как сбледнула и тоже как заорёт дурным голосом: «Сбрасывай! Сбрасывай!».

Тут уже даже я маленько скумекала. А ведь и правда: ежели в случае аварии чего сбрасывать надо, то сортир очень даже удобная штука, их ведь над ямами ставят. (А ежели в случае аварии от страха прохватить может, так и вовсе удобно получается, впрочем неважно.)

Это я сейчас всё подробно рассказываю, а тогда знаешь как оно быстро происходило? Принцесса Луна опять орёт, только другое уже: «Застряла! Застряла!», а я ведь говорила, что брат эту будку по бабулиным размерам делал. А принцесса Твайлайт такая: «Рви!». И тут как треснуло! А потом как хлопнуло!

Ну, чего хлопнуло, мы сразу поняли, это принцесса Луна из будки телепортировалась. А что застряло и чего треснуло, поняли чуть позже, когда увидели, что у ей на крупе кусок будки надет~--- ну, тот самый, на котором сидят, точнее в который садятся, ты понимаешь. А потом хлопнуло ещё раз, или даже правильнее будет сказать «рвануло», как принцесса Твайлайт Спаркл выразилась.

Тут принцесса Селестия чего-то про какое-то метание прошептала, я ещё хотела сказать, что наверно лучше было всё-таки сбрасывать, а не рвать и метать, но побоялась попасть под горячее копыто.

В общем, оно рвануло. Поскольку сортир тот мой брат строил, а он всегда всё от души делает, то яма там большая была. А~поскольку сортир тот меня всего на год младше, в яме той было тоже немало, ты меня понимаешь. И всё это вверх фонтаном ка-ак даст!

А вверху там как раз Фезервейт со своим фотиком болтался, я тебе говорила. Но ему повезло, его в последний момент дядя Виндчейзер успел в сторону отдёрнуть. А корреспондентов никто не отдёргивал и им не повезло.

Помнишь, Бэбс, как в школе про «параболу ветвями вниз» толковали? Оказывается, прикольная штука, когда она не на картинке в учебнике, а по правде! Сначала вверх и в сторону, а~потом вниз и в сторону~--- яма и корреспонденты, это значит, у параболы корни, а где Фезервейт болтался, там аккурат посередине её вершина. Я теперь уж никогда не спутаю.

Значит, корреспондентов уляпало по первое число, а тётя Рэрити вдруг чего-то в пляс пустилась и копытами аплодировать стала. Ещё и кричалку какую-то распевала вроде хуфбольной: «Чем поливали, того и похлебали! Чем поливали, того и похлебали!». А тётя Флаттершай (она рядом с нами стояла) захихикала и что-то прошептала насчёт прошлогодней коллекции. Я опять не поняла, вроде ведь тётя Рэрити ничего не коллекционирует (она даже драгоценные камни не для коллекции, а для своих нарядов собирает), так что надо будет потом у тёти Флаттершай уточнить.

А тётя Рэйнбоу Дэш забрала Фезервейта у дяди Виндчейзера, начала его (Фезервейта) трясти и орать, что щас расшибёт этот дурацкий фотик об эту дурацкую бо́шку. Но не расшибла, только плёнку ему засветила и пару пендалей выписала.

А принцессы Луна и Твайлайт Спаркл стояли как раз в~проекции вершины, ежели опять школу вспомнить, и тут я~даже не знаю, что сказать, потому как рассказывать такие вещи о~принцессах довольно стрёмно. Просто бабулю процитирую (она-то старенькая, ей уже можно): «Видела я лошадей в яблоках, но чтоб в конских, и не просто лошадей, а аликорнов{\ldots}». Ну, ты поняла. Хотя корреспондентам, конечно, сильно круче досталось.\looseness=-1

Мы потом с девчонками опять пристали к принцессе Селестии, мол, что же это тут такое было-то, и что всё это значит? А~она вдруг мило разулыбалась, как будто всего пару минут назад морду от нас не воротила, и говорит. Дескать, что́ всё это значит~--- принцессам Луне и Твайлайт Спаркл теперь надолго разбираться хватит, и лично она их полёт фантазии предсказывать даже не берётся. Но одно можно сразу сказать (и тут лично ко мне повернулась)~--- что по теории принцессы Твайлайт наш род Эпплов выходит самым что ни на есть чистейшим родом земных пони от самой что ни на есть земли с её камнями и~корнями.

Поняла теперь? Я подумала, что это же круто, и сразу села тебе писать. Ты расскажи там у себя всем, чтоб знали. Или просто дай это письмо прочитать. (Я показывала черновик дяде Виндчейзеру, он всякие истории хорошо рассказывать умеет, так он одобрил, сказал что всё здорово получилось и посоветовал в~писатели идти. А чего, мож и пойду!)
\begin{flushright}Твоя \textbf{Эппл Блум}\end{flushright} 
P.S. Брат, конечно, сразу начал новую будку над новой ямой строить, но закончит ещё только к вечеру. А утром я по привычке спросонья пошла на старое место, уже там окончательно проснулась и чего-то задумалась. Посмотрела ещё раз~--- а ведь ветер вчера дул совсем не в том направлении, как от старой ямы к корреспондентам! И рог у принцессы Селестии чего-то всё время светился, даже после заката. Интересно, это как-то связано или нет?
\begin{center}
	* * * * * *
\end{center}

--- Прикольно!~--- хихикнула Скуталу.~--- Это, значит, и есть твой дебют, да?

--- Агась. Кто б мне тогда сказал, чем оно кончится{\ldots}

--- И что? Не стала бы связываться?

--- Стала бы,~--- вздохнула Эппл Блум.~--- Интересно же. Как ты там вчера{\ldots} это ж не то, что детская фигня с вазой, выпечкой и витражом{\ldots} Свити, слышь?..

--- А?

--- Объясни для тупых, чё там они рвать, метать и сбрасывать хотели-то?

--- Рвать~--- канал между Лу́ной и Луно́й. Энергетический. Когда испугались, что энергия по нему вот-вот в обратку хлынет. Сбрасывать~--- избыток энергии опять же, его в таких случаях обычно молнией в небо кидают, самый быстрый способ. Не успели ни то, ни другое. А то, что ты «метанием» обозвала, так это про метан шла речь. Газ такой, который{\ldots} э-э{\ldots} в общем, при этом деле скапливается. Горючий и взрывучий, ну ты сама видела. Про него в том письме, что мы вчера зачитывали, шла речь, помнишь? Ну, в постскриптуме.

--- Блин, это ж сколько раз я тогда в рассказе лажанулась-то! Рвать, метать и сбрасывать, секстант этот ваш опять же{\ldots}

--- Да ладно, прикольно же! Свити, давай теперь то, которое через пару часов после было написано.

--- Сейчас, оно короткое{\ldots}~--- та развернула свою бумагу.~--- Только это{\ldots} вы не удивляйтесь, оно так экспрессивно написано{\ldots} Я даже сама удивилась.
\begin{center}
	* * * * * *
\end{center}

ТВАЙЛАЙТ, БЛИН!

Не успела я толком отмыться, высушиться и завалиться отдохнуть, как сестрица нанесла мне визит. И сказала прямым текстом, чтобы мы не вздумали про этот эксперимент чего публиковать или обсуждать с кем не надо. Потому что иначе она тоже кое-чего опубликует. Я, конечно, спрашиваю, чего именно, а она мне фотографию предъявляет. Блин, блин, блин!

В принципе, я могла бы здесь это изображение воспроизвести, но не хочу лишний раз видеть его перед глазами. Поэтому просто поверь на слово: рулончик развевающегося пипифакса на твоём роге выглядит ничуть не более по-королевски, чем выдранный из нужника толчок на моём крупе! И это не говоря уже о том, что мы обе в тот момент были в крапинку! Кстати, на тебе эти крапинки куда больше заметны, чем на мне.

Снято сверху~--- не иначе, тот мелкий крылатый поганец успел-таки щёлкнуть, а сестрица успела из его фотоаппарата копию плёнки сдёрнуть до того, как твоя подруга эту плёнку ему засветила. Больше просто неоткуда было взяться, по корреспондентам-то я сама засветкой шарахнула, как только чуть опомнилась (кстати, надо тебя этому заклинанию научить, полезная штука для принцессы).

Помнишь, я тебе несколько раз повторяла, что сестру хорошо знаю? Ну так вот, это серьёзно! Реально, не вздумай публиковать, потому что в случае чего~--- светилом своим клянусь!~--- эквестрийская лунная программа будет продолжена, только уже под моим шефством.

Очень-очень сильно надеющаяся на твоё понимание~---
\begin{flushright}\textbf{Л.}\end{flushright}
P.S. Как я в следующий раз буду у вас в Понивилле, сведи меня с этим шустрым пацанёнком-фотографом, ладно? Не бойся, кошмарить его не буду~--- наоборот, хочу помочь ему с карьерой и натравить на сестрицу. У поганца явный талант.
\begin{center}
	* * * * * *
\end{center}

--- Эх,~--- вздохнула Эппл Блум.~--- Вот бы фотку ту добыть, а?

Свити решительно замотала головой:

--- Не буду такое просить, даже и не надейся.

--- Скутс?

--- Неа! Если моего о таком попросить{\ldots} точно круп на уши натянет. И скажет, что так и было, типа такие просьбы может высказывать только тот, кто крупом думает. Можешь, конечно, сама попросить, если хочешь{\ldots} но по-моему, твоя сестра ему потом за воспитательную работу ещё и спасибо скажет.

--- Эх, жалко{\ldots}

--- Вопрос на миллион бит,~--- ехидно сказала Свити.~--- А ты не думала, что тебе это задание дали с намёком? Если, значит, будешь нос совать куда не надо и любопытствовать сверх меры, то и тебя так же окатить могут?

--- Думала.

--- И что?

--- А, снявши голову, по гриве не плачут. Где наша не пропадала. Свити, слышь?..

--- Что опять?

--- А ты это заклинание, которым плёнку в фотике засветить можно, знаешь?

--- Знаю только, что оно существует. Теперь. Узна́ю, никуда оно не денется~--- посмотрю книжки, поспрашиваю Старлайт{\ldots} Тебе-то зачем?

--- Да мне как бы не его, а как раз наоборот. Шоб нельзя было засветить, защиту какую-нить.

--- А, поняла. Ладно, подумаю. Давайте, что ли, дальше читать?

--- Неа!~--- у Эппл Блум вдруг шевельнулись уши.~--- Давайте лучше пойдём обедать, а чтения потом.

--- Фи,~--- Свити поморщилась.~--- Как можно после этого прочитанного думать о еде?!

--- Ну, нам больше достанется!~--- обрадовалась Скуталу.

--- Не дождётесь!

И вся троица Меткоискателей ссы́палась вниз по лестнице из своего клубного домика. Сказать по правде, был он уже для них маловат, но не пренебрегать же из-за этого традициями?


\chapter*{Глава третья, в которой Меткоискатели узнаю́т, что было непосредственно вслед за этим}\addcontentsline{toc}{chapter}{Глава третья, в которой Меткоискатели узнаю́т, что было непосредственно вслед за этим}


--- Ф-фух{\ldots} чё там дальше-то было?~--- поинтересовалась Эппл Блум, отдуваясь после сытного обеда.~--- Свити? Я так понимаю, потом идёт та переписка, шо тебе ночью прислали?

--- Не совсем. Судя по всему, когда Твайлайт получила то письмо{\ldots} ну, про фотографию{\ldots} она на следующую ночь ещё один свой собственный эксперимент задумала. И сделала. Правда, поплохело ей с него так, что пришлось Флаттершай на помощь звать{\ldots} --- Свити Белль хихикнула: --- к кому другому обращаться явно постеснялась, принцесса же!

--- Ещё один?! Шо, опять?!

--- Так Твайлайт же! На этот раз не такой разрушительный, но всё равно поплохело. Она даже ответ написать сама не смогла, Спайку диктовала. Он за неё забеспокоился и перед отправкой прочитал Старлайт и Флаттершай{\ldots} ну, и я при этом была. Старлайт тоже забеспокоилась, попросила меня со своей стороны отписать, а потом ещё ты свой черновик показала, ну я~и отписала{\ldots}\looseness=-1

--- Погодь, эт’ потом. Давай по порядку. Значит, гришь, ещё один эксдерьмент учинила и Луне написала{\ldots}

--- Да. Поскольку я тогда всё слышала, то вчера просто попросила Старлайт запись с моей памяти снять. Она не шибко довольна осталась, но отказывать причин не было~--- лучше уж точная запись, чем если бы я по памяти восстановить попыталась и переврала чего-нибудь.

--- Логично. Читай.
\begin{center}
	* * * * * *
\end{center}

Луна!

Не пугай меня такими намёками~--- я и без того уже успела испугаться, что нас отправят исследовать эти самые свойства вакуума прямо с места эксперимента. Я даже чуть{\ldots} впрочем, неважно. Речь не о том, чего мы сейчас не должны делать, а о~том, что должны были сделать, но не сделали.

Мы с тобой совершили большую методическую ошибку, даже две. Во-первых, мы почему-то сосредоточились на конечной стадии эволюции наноботов, а ведь начальная не менее интересна! Во-вторых, мы совершенно забыли, что являясь аликорнами, сами несём в себе магию земных пони --- а значит, не было ни малейшей надобности ударять в грязь лицом на виду у всей понивилльской общественности, обращаясь за помощью к Эпплджек.

Исправляя эту вторую ошибку, я решила детально изучить не только финальную стадию, но и сам процесс, для чего заказала большую кастрюлю каменного супа.

Думаю, ты заметила некоторые флуктуации в лунной траектории этой ночью? Знай же, что это моих копыт, то есть моей магии дело~--- сначала я пробовала изменить движение светила, потом съела поварёшку супа и попробовала снова. После тарелки супа мне удалось изменить траекторию почти на полградуса за несколько секунд! К сожалению, когда мне удалось довести отклонение до более чем одного градуса, суп кончился, но это уже несущественно. Ты понимаешь? С учётом ранее полученных качественных результатов, это же значит, что теория окончательно подтверждена!

Осталось только решить, как уломать твою сестру с учётом обстоятельства, столь красочно описанного тобой в предыдущем письме. Возможно, свою роль сыграет тот факт, что это окончательное подтверждение было получено без жертв и неприятностей? Попробуй как-нибудь, всё-таки ты свою сестру знаешь лучше меня, как много раз мне на это указывала.

Ох, вспомнила! К сожалению, говоря «без жертв и неприятностей», я немного преувеличила. Ладно, сильно преувеличила. Этот каменный суп{\ldots} представляет собой довольно тяжёлую пищу. Во всех смыслах. Я до сих пор пытаюсь справиться с последствиями его употребления{\ldots} достаточно сказать, что пришлось обратиться за помощью к Флаттершай и той её штуке, которую она по ошибке приносила на место предыдущего эксперимента (если ты не поняла, о чём идёт речь, то и не развивай, пожалуйста, эту тему{\ldots} впрочем, не надо её развивать в любом случае). Короче говоря, не пытайся повторить мой опыт.
\begin{flushright}Твоя \textbf{Т}.\end{flushright}
P.S. А всё-таки они существуют!!!
\begin{center}
	* * * * * *
\end{center}

--- Каменный суп?! Фу. Буэ!~--- скривилась Эппл Блум.

--- Тебе-то чего «буэ»?~--- изумилась Скуталу.~--- Ты ж сама земная.

--- И чё с того? Дрянь редкостная. Ты сама-то его пробовала хоть раз?

--- Не-а.

--- Вот и не пробуй. Принцесса Дружбы, небось, плохого не посоветует. Ладно, неважно. Так чё получается --- те нанобиты{\ldots}

--- Наноботы!

--- Неважно! Они чё, реально существуют?!

--- Не скажу,~--- хихикнула Свити.~--- Потерпишь. Значит, Старлайт меня озадачила: дескать, ситуация из-под контроля выходит, напиши-ка ты на всякий случай тоже, небось, хуже-то не будет. И потом буквально через час ты такая со своим черновиком: дескать, тут чего-то странное творится, совсем непонятное, вот глянь, чего я надумала и напиши-ка на всякий случай, небось, хуже-то не будет{\ldots}

--- А чё, зря просила, что ль? Телекинез же у тебя с того письма не отвалился?

--- Не зря. Вот, слушайте, я себе ещё тогда сразу копию сделала{\ldots}
\begin{center}
	* * * * * *
\end{center}

Дорогая принцесса Луна!

Очень извиняюсь, что это письмо не про усвоенный урок, но меня очень просили, причём сразу двое, поэтому я обе просьбы в одно письмо и объединю. С Вашего позволения.

Начну с менее важного. Моя подруга Эппл Блум (Вы её знаете) на днях писала письмо своей кузине Бэбс Сид в Мэйнхэттен (её Вы знаете вряд ли), и я опять очень извиняюсь, но это письмо было про Ваш эксперимент. С принцессой Твайлайт Спаркл. Она (Эппл Блум) показывала мне черновик, там есть вот такой кусочек:

\emph{«Помнишь, Бэбс, как в школе про „параболу ветвями вниз“ толковали? Оказывается, прикольная штука, когда она не на картинке в учебнике, а по правде! Сначала вверх и в сторону, а~потом вниз и в сторону --- яма и корреспонденты, это значит, у параболы корни, а где Фезервейт болтался, там аккурат посередине её вершина».}

Про Фезервейта не обращайте внимания, хотя возможно, Вы его тогда заметили. Во время эксперимента. Так вот, Эппл Блум приставала ко мне с вопросами про какую-то параболическую магию, а я ей сказала, что ничего такого знать не знаю, а потом уже я к ней пристала с вопросом, зачем ей это, а она мне и сказала. Интересную вещь.

Оказывается, она на следующее утро (после Вашего эксперимента) пошла на то самое место. Она же, Вы понимаете, привыкла туда ходить по утрам, не в упрёк Вам будь сказано. И посмотрела ещё раз, где кто находился, и заметила, что ветер во время эксперимента был совсем не туда. Куда по этой самой параболе из ямы всё в корреспондентов прилетело. А потом ещё вспомнила, что у принцессы Селестии рог всё время светился, даже после заката, и я тогда тоже вспомнила, что действительно, светился.

И вот поэтому Эппл Блум спросила меня про эту самую параболическую магию, но я уже говорила, что сказала ей, что ничего такого не знаю, и спросила Старлайт Глиммер, но она тоже не знает, и тогда она (Эппл Блум) попросила меня написать Вам. Может, Вам что-то про это известно. Это, значит, было первое.

А теперь второе, более важное. Меня Старлайт Глиммер тоже попросила Вам написать и основные тезисы перечислила. Я, конечно, спросила, почему бы ей самой это не сделать, а она говорит~--- потому что при эксперименте не присутствовала, а я~присутствовала, и у меня, возможно, более убедительно получится, мол, незаинтересованному жеребёнку поверят больше, а~ещё Вы к моему стилю уже привыкли. Хотя мне кажется, её эта история уже просто достала. С принцессой Твайлайт Спаркл и наноботами.

То, что она попросила, это касается письма от принцессы Твайлайт Спаркл, которое Вам незадолго до моего прийти должно было. Спайк, который его писал под диктовку, нам потом прочитал~--- то есть он читал для Старлайт Глиммер, а я при этом присутствовала. И тётя Флаттершай. Потому что Спайк за неё беспокоится (за принцессу Твайлайт Спаркл, не за тётю Флаттершай).

Собственно, суть проблемы вот в чём. Меня просили Вам передать, что в том письме можно верить только одному утверждению --- предостережению от употребления каменного супа. Если ты, конечно, не земнопони. Потому что все остальные тамошние утверждения вытекают именно из этого употребления.\looseness=-1

Честное благородное слово, вот лучше его не есть и даже не смотреть на него. Я сама хоть и не ела, но видела, как ела принцесса Твайлайт Спаркл, и чем это кончилось. И теперь не то что сама не буду, но и другим не посоветую. Там в письме говорится про помощь от тёти Флаттершай с одной штукой, так это очень мягко сказано. Принцесса Твайлайт Спаркл после этого даже писать сама не могла, Спайку диктовала.

Я только одно уточнила~--- а как же все эти небесные градусы, что в письме перечислялись? Так тётя Флаттершай вздохнула и сказала, что при таких градусах на медицинском термометре недолго небо с землёй спутать, и там вообще всё что угодно померещиться могло, а Старлайт Глиммер сказала, что вот именно поэтому Вам написать и нужно. Чтобы Вы как следует разобрались, что же там с Луной происходило. Или не происходило. Вы уж, пожалуйста, разберитесь.

Вот, вроде всё. А, хотя ещё нет.

Дорогая принцесса Луна, Вы, главное, не беспокойтесь. Про тот эксперимент. Тут у нас в Понивилле ещё и не такое бывало. Мы с девчонками, правда, именно \emph{такого} не устраивали, но это только потому, что нам \emph{так} экспериментировать никто и не позволил бы. Так что всё ещё впереди.

Это шутка, конечно. А если серьёзно, то мы с девчонками сразу старших спросили, что же теперь будет. Так тётя Эпплджек только плечами пожала и сказала: «Спокойствие, только спокойствие! Дело-то житейское. Ща братец новую лабораторию отстроит. А старую отчистим как следует, навес над ней поставим и будем там ховаться, ежли их высочества нас ещё какими эксдерьментами осчастливят». А я спросила у сестры, как же все эти корреспонденты, так она только фыркнула и сказала: «Дорогуша! Если в какой газете и пропечатают, что „наш вонючий корреспондент передаёт из Понивилля без всяких подтверждений“, то дальше там может быть что угодно~--- всё едино, такой газетой только и сделают, что подотрутся!».

Так что беспокоиться не о чем, все всё понимают. И вот видите, без урока всё-таки не обошлось!

Ваша хоть и не ученица, но уже серьёзно задумывающаяся о~пределах возможного в Понивилле~---
\begin{flushright}\textbf{Свити Белль}\end{flushright}
P.S. Любимую вазу принцессы Селестии так и не удалось склеить? Я же не нарочно, честное слово.

\vspace{2mm}
\begin{center}
	* * * * * *
\end{center}
\vspace{2mm}

--- Это вот, значит, откуда слово «эксдерьмент» появилось{\ldots}~--- пробормотала Скуталу.~--- Хотя, с другой стороны, кто ещё мог такое сказануть{\ldots}

--- Ну да, сестра моя, кто ж ещё. А чё, ту вазу так и не склеили?~--- поинтересовалась Эппл Блум у Свити.

--- Нет, выбросили осколки в итоге.

--- А магией восстановить? Была бы как новенькая, и не скажешь что битая.

--- Это на глаз, а по ауре сразу видно, битая или нет. Это как{\ldots} ну, как для тебя яблоко-паданица. Отличишь ведь?

--- Конечно. Ладно, поняла. Ну, стал быть, написала ты, и?..

--- Ответа мне тогда не было, да я и не ждала, дальше там у них своя разборка пошла. Ну как разборка, вопрос-ответ. Это как раз два последних письма, что мне за ночь пришли. Причём одно опять на повышенных тонах, там часть даже заклинанием скрыта была. Читать?

--- Ещё спрашиваешь!
\begin{center}
	* * * * * *
\end{center}

Тия!

Я обращаюсь к тебе письменно, так как именно в письменной форме получила информацию, которую хотела бы обсудить и по поводу которой очень хотела бы получить с тебя кое-какие объяснения.

Прилагаю при сём два письма~--- от твоей лучшей ученицы и~от моей как бы протеже (впрочем, торжественно клянусь впредь это «как бы» не употреблять, девочка молодец). Если что, это копии~--- твой остроумный поступок меня кое-чему научил.

Пожалуйста, соблаговоли прочесть их оба, прежде чем развеять хорошо известное тебе заклинание, скрывающее остаток этого послания. После чего~--- жду ответа, как соловей лета.

\begin{center}\texttt{(начало заклинания)}\end{center}

\begin{quotation}
Твою мать, Тия, какого дискорда здесь вообще происходит?! Ты соображаешь, что творишь? Ладно ещё я, мне-то после того случая тысячелетней давности уже всё равно, каких собак на меня ни вешай и~чем ни поливай, но ты читала, до чего себя Твайлайт довела?! Она же твоя лучшая-любимая ученица вроде бы, или что-то изменилось? Девчонку-то за что?!

Я не понимаю, вот зачем надо было весь цирк городить? Неужели так трудно было просто сказать прямо~--- вариантов-то всего три: «да!», «нет!» и~«не моги!». Она же всё-таки принцесса уже, должна быть в состоянии понять.

Ну ладно~--- то, что ты содержимым ямы в прессу влепила, это не только дети, это я и сама догадалась (не верю я в такую случайную точность, особенно учитывая то, что кроме писак, только нас двоих и~задело). Не знаю уж, как ты отмажешься, если ещё и газетчики догадаются, но это будет твоя проблема, не моя. Хотя саму идею безоговорочно одобряю, независимо от того, что тебя побудило. Но когда я задумалась, что́ послужило причиной возникновения той параболы{\ldots}

Тия, я ведь не дура. Не мог скачок силы поднятия, пусть даже и резкий, привести к потере контроля над светилом и пустить всё приложенное в обратку. Ну не мог! Я специально поэкспериментировала на этот счёт~--- те самые флуктуации, которые бедная Твайлайт по наивности приняла за собственные результаты. Как ни крути, такое только со внешнего воздействия завернуть можно.

А тут мне приносят письмо о том, что сестрица-то моя, оказывается, и после заката рогом вовсю отсвечивала! Та самая сестрица, заметим в скобках, которая тысячу лет моё светило тягала и довольно-таки неплохо научилась с ним обращаться. Если это совпадение, то я{\ldots} да кой хрен совпадение!

Короче, вот тебе моё крайнее слово --- колись! И разруливай, потому что кроме как тебе, больше некому. Ты ведь тоже не дура и должна понимать, что дальше этот фарс продолжать нельзя. Иначе, помяни моё слово, разбитыми вазами дело не ограничится.

Кстати, занявшись этими изысканиями, я тут заодно ещё кое-что посчитала. Получается, что наших с Твайлайт объединённых сил сейчас как раз хватает, чтобы забросить на лунную орбиту что угодно. Или кого угодно. Забавно, правда?
\end{quotation}

\begin{center}\texttt{(конец заклинания)}\end{center}
\begin{center}* * * * * *\end{center}

--- И чё, даже не подписано?~--- поинтересовалась Эппл Блум.

--- А смысл подписывать? Это заклинание никто другой наложить не мог.

--- А, ну если так{\ldots}

--- Мне вот тоже начинает казаться,~--- заметила Скуталу,~--- что тут без намёка тебе не обошлось. Типа, вот чего может случиться с плохими, негодными журналистами.

--- Фигня, прорвёмся. Однако тут вот, я вижу, она и сама интересуется~--- «да» или «нет». Хоть и мельком.

--- Погоди, уже немного осталось. Всего одно письмо.
\vspace{2mm}
\begin{center}
	* * * * * *
\end{center}

Лулу!

Вынуждена признать, ты кругом права~--- я действительно задолжала несколько объяснений тебе и Твайлайт. Особенно Твайлайт.

Бедная девочка. Действительно, кому, как не мне, лучше всех знать неуёмность её научного любопытства{\ldots} но \emph{такого} даже я предположить не могла. Честное благородное слово, как выразилась бы твоя протеже. На этот счёт можешь не беспокоиться: я с Твайлайт в ближайшую пару дней поговорю (именно поговорю, лично) и сама всё объясню. Что же касается объяснений для тебя~--- изволь.

Знаешь, я тут тоже кое-что подсчитала. Государственному служащему полагается отпуск двадцать восемь дней в году, так что у меня за последнюю тысячу и восемь лет этих неиспользованных отпусков должно было набраться на семьдесят семь лет с хвостиком. Многие столько не живут.

Понимаешь, к чему эти выкладки? Я устала, Лулу. Правда, устала. Веришь, нет~--- мне пару недель назад даже сон снился. Про то, как ты вернулась с Луны абсолютно здравомыслящей само́й собой, в отместку закинула меня туда вместо себя, а сама взяла Твайлайт с подругами в помощницы и стала тут править. Исключительно светлый и добрый сон, как ни странно~--- я отдыхала, понимая, что ты меня быстро простишь и вернёшь обратно, а у вас тут всё получилось просто замечательно, хоть и не без трудностей.\footnote{По описанию этот сон до странности похож на сюжет повести «Практикум».} Самое забавное, что я{\ldots} впрочем, давай как-нибудь в другой раз, за чашкой чая.

Не последнюю роль в этой усталости сыграли те, кого я~полила. Загляни в «Эквестрийские скрепы» от третьего числа, в~«Кантерлотский колдовстворец» от седьмого{\ldots} Этот список даже бессмысленно продолжать, он устареет задолго до того, как ты дочитаешь моё письмо. Поэтому я с радостью ухватилась за такую возможность отомстить, при которой для борзописцев крайней выглядела бы \emph{ты}~--- на тебя-то, при твоём характере и былой репутации, никто не рискнёт всерьёз задрать перо. Пожалуйста, постарайся простить меня за это.

Ты, конечно, правильно определила «причину возникновения той параболы». А откуда ещё мне было косвенно взять энергию? Думаю, что за это ты на меня не очень сердишься, но всё же извинюсь ещё раз. За то, что по вам с Твайлайт тоже прилетело. Поверь, что это было непреднамеренно~--- я хоть и достаточно хороший маг, чтобы с первого раза попасть фонтаном в цель без тренировки, но всё-таки раньше никогда такого не делала и всех побочных эффектов предусмотреть не могла. Хорошо уже и то, что больше никого не задело, да и вас-то лишь забрызгало, а не облило с ног до головы. Я не нарочно, честное слово.

Теперь перехожу к самому трудному моменту, из которого проистекло всё предыдущее. Сначала процитирую тебя:

\emph{«Я не понимаю, вот зачем надо было весь цирк городить? Неужели так трудно было просто сказать прямо~--- вариантов-то всего три: „да!“, „нет!“ и „не моги!“. Она же всё-таки принцесса уже, должна быть в состоянии понять»}.

Если бы всё было так просто, Лулу. Если бы{\ldots} Проблема-то как раз в том, что я не знаю~--- «да» или «нет». Веришь, нет~--- НЕ ЗНАЮ. Для Твайлайт довольно естественно даже и сейчас считать, что я знаю всё, но тебе-то прекрасно известно, что это не так! А почему я не сказала «не моги!»~--- вот это как раз потому, что ЗНАЮ. Слишком хорошо знаю свою лучшую ученицу, и просто поверь, что ничем хорошим такой запрет не кончился бы.

Я сделала то, что сделала, именно из соображения «она же всё-таки принцесса уже, должна быть в состоянии понять». Ты ведь читала через моё плечо, как я отвечала Твайлайт про её статью~--- разве я в чём-то покривила душой? Да, ей удалось высказать много очень интересных идей{\ldots} но трактовка телепортации действительно может вызвать определённые идеологические неудобства, а интерпретация магии земных пони по-настоящему дурно пахнет, в чём ты имела возможность убедиться лично.

Проблема, как я теперь понимаю, была в том, что Твайлайт только эти три варианта и знает~--- четвёртый, «можно если осторожно», она пока не очень воспринимает. А я хотела сообщить ей именно это и, увы, не преуспела. Девочка отреагировала, как на запрет, со всеми вытекающими последствиями, и хорошо, что рядом с ней оказалась ты.

Кроме шуток, Лулу, я задолжала тебе по меньшей мере одно «спасибо», которое сейчас и говорю. Благодаря тебе всё это не зашло дальше фарса, а ведь могло быть хуже, много хуже, просто поверь мне на слово. Я, как и сказала, поговорю с ней~--- поговори, пожалуйста, и ты. Можешь показать ей это письмо. Постарайся объяснить, что некоторые вещи просто НЕ СТОИТ делать раньше времени, даже если очень хочется и можется~--- уж ты-то знаешь это как никто другой. Я очень рассчитываю на твою помощь и заранее благодарна тебе.

Говорила я тебе или нет, но вдвоём вы замечательно дополняете друг друга, и это одна из причин, по которым вас обеих очень любит твоя сестра~---
\begin{flushright}\textbf{С}.\end{flushright}
P.S. Не было печали~--- теперь ещё и шеф-повар отставки требует после того каменного супа. Ты-то с твоими перекусами разницу вряд ли заметишь, а другие? Придётся требование удовлетворить{\ldots} и перевести его в замок Твайлайт. Скажем, на полгода. Пусть девочку подкормит, ей теперь восстанавливаться надо.

P.P.S. Политых журналистов не бери в голову. Когда перед отправлением в Понивилль один из них заикнулся про «выгодный ракурс, в котором бы и восходящую Луну было видно вместе с экспериментаторами», я сделала круглые глаза и заявила, что это будет самое опасное место. Разумеется, меня стали упрашивать~--- и конечно, я дала себя уговорить. Теперь могут догадываться и писать о чём угодно, их предупреждали при множестве свидетелей. Если что, ссылайся на меня.
\begin{center}
	* * * * * *
\end{center}

--- Во, блин, облом-то!~--- вздохнула Эппл Блум.~--- Я про Твайлайт. Это ж столько всего вытерпеть, и в итоге{\ldots} ни «да», ни «нет», понимай как знаешь.

--- Ну почему,~--- пожала плечами Свити.~--- Когда гипотеза не подтверждена, но и не опровергнута, это всё лучше, чем если её опровергли. Тут-то вопрос открытым остался, может, и найдут ещё какой способ проверить. Идея всё-таки интересная, нельзя не признать{\ldots} хоть и с запашко́м. Странно другое{\ldots}

--- Что?

--- Последнее письмо{\ldots} оно слишком личное. Особенно вот это «я устала». Такое вообще обнародовать можно?

--- Да вроде, норм,~--- Эппл Блум пожала плечами.~--- Мы же его, во-первых, не на помойке нашли, а нам его прислали сама знаешь откуда. А во-вторых, вот это вот «устала», оно ж как раз в основную тему! Небось, уже тогда ясно было, к чему всё клонится.

--- Думаешь, связано?

--- А думаешь, нет? И в-третьих, это же всё равно смотреть будут перед тем как обнародовать.

--- А всё одно это намёк нам,~--- упрямо пробормотала Скуталу.~--- Что в такие игры и доиграться можно.

--- И чё? Мало нам, что ли, про это намекали? Ну ладно, значит, все документы у нас теперь есть. Теперь ясно, чё с этим дальше делать{\ldots}

--- Не все. Вон, там ещё повар в отставку просился.

--- Так это мелочи. Что просился~--- это мы знаем, а стилизацию прошения я и сама напишу. Дескать, прошу меня освободить, больше никого не наказывать, кругом виноват, надо было этой Пинки Пай позволить само́й каменный суп приготовить, а я у неё только рецепт взял{\ldots}

--- Э, стоп! Пинки-то тут при чём?!

--- Может, и ни при чём. Но она против такого упоминания возражать не будет{\ldots} хотя я спрошу, конечно.

--- А так разве можно?

--- Чего нельзя-то, ежли деталь мелкая, несущественная. Называется «реконструкция». Можно посмешнее расписать, чтобы интереснее было{\ldots} главное, настоящих имён не указывать. В общем, это я сама. Тут с этими письмами и работы-то немного, набросать то прошение, да написать между ними вставки, вот примерно как мы щас всё зачитывали и обсуждали.

--- А дальше?

--- Ну, дальше, я так понимаю, был ваш бенефис в Школе.

--- Чего это «наш»-то?~--- поинтересовалась Свити Белль.

--- Так а чей? Не мой же. Кто идейку завучу подбросил? И~сами грифоны потом вас двоих отметили, если что.

--- И чего ж тебе в этой связи,~--- подозрительно осведомилась Скуталу,~--- из-под нас надо?

--- Почти ничего, --- ухмыльнулась Эппл Блум.~--- Завтра утром собираемся, теперь моя очередь вам кой-чего показывать{\ldots}


\chapter*{Глава четвёртая, в которой главные роли играют не только и~не столько Меткоискатели}\addcontentsline{toc}{chapter}{Глава четвёртая, в которой главные роли играют не только и~не столько Меткоискатели}

В это утро сияла Эппл Блум, причём с учётом её масти можно было уточнить: сияла, как начищенный медный таз. Подруги, напротив, были исполнены подозрительности.

--- Если сейчас мне скажут, что про тот урок нужно чего-то писать{\ldots}~--- пробормотала Скуталу.

--- Не писать. Читать,~--- успокоила Эппл Блум.~--- Всё-таки о~полётах ведь. Во, держи,~--- она плюхнула на столик толстую пачку бумаги.

--- Упс{\ldots}~--- Скуталу моментально узнала почерк.~--- Откуда?

--- Твой писал, сама видишь. Я ж на том уроке была, ну потом и попросила свою сделать запись с моей памяти. Она поворчала немножко, что садишься, мол, за машинку и пишешь, чья память-то? Но сделала. Я гляжу --- фигня. Одни слова же, которые там вслух звучали. Не, всякие мелкие пояснения прописать и про эти проверки с подставами добавить я бы и сама смогла, конечно. Но то, шо вы с Сильверстрим{\ldots} я ж в этом ни бум-бум. А моя посмотрела-посмотрела, сгребла все эти бумажки и говорит: ладно, мол, попрошу кого надо, не парься. Ну и вот.

--- Блин{\ldots}

--- Чё?

--- Чё-чё{\ldots} там же про меня будет.

--- А само́й про себя зачесть слабо́?

--- Щазз! Давай сюда. Ещё и от первого лица{\ldots}
\begin{center}
	* * * * * *
\end{center}

Без стука~--- это значит, Старлайт. Все остальные имеют обыкновение стучать при входе. Даже Трикси.

Хотя Трикси вообще-то не входит, она \emph{возникает}. И стукает при этом не она по двери, а дверь по стене~--- когда настежь распахивается. После чего можно смело биться об заклад, что скучно не будет, это ж Трикси. За что её и ценим.

А Старлайт всегда вот так, по-простому, даже если никуда торопиться не нужно. А уж если нужно{\ldots}

Ну вот, накаркал.

--- Слушай, твоя помощь срочно нужна!

--- Что у вас?! Кто-то в Вечнодиком опять потерялся?

--- Чего? А{\ldots} нет, типун тебе на язык. Не это. Хуже!

--- Ну?!

--- У тебя через час лекция в Школе!

--- Чего?! Какая лекция?! В какой ещё школе?

--- В нашей. Школе Дружбы.

--- Не понял. Видимо, я что-то пропустил.

--- Совсем немного. «Ой, Старлайт, у нас тут бла-бла-бла, ты же присмотришь, чтобы из-за этого никакие уроки не пропадали? Вот спасибо, на тебя всегда можно положиться!». Всё как обычно, ничего нового. У них бла-бла-бла, а у меня теперь дырка на третьем уроке.

--- Всё равно не понял. Что, заткнуть не можешь?

--- Затычки кончились! Я уже во всех классах всё что только можно на два раза изложила.

--- Трикси позови. Она мне пару недель назад битый час рассказывала про особенности восприятия у разных народов и про адаптацию фокусов. С демонстрацией. Реально интересно, я аж заслушался. В самый раз тема для вашей школы.

--- Ты такой догадливый! Это же она на тебе свою лекцию и~репетировала. За те две недели уже четыре раза излагала.

--- Ну, эта ваша{\ldots} Дёрпи. Про важность дружбы при косяках в~работе.

--- Ты опять очень догадливый, но это и так постоянная тема. У неё на новый заход ещё примеров не набралось{\ldots} тьфу-тьфу-тьфу, конечно.

--- Ну я не знаю, пускай Свити свои письма зачтёт.

--- Уже.

--- Что, серьёзно?!

--- Нет, блин, у меня сейчас самое настроение шутки шутить! Говорю же, некем заткнуть, кроме тебя. А это ещё и группа особая, которая на контроле сверху{\ldots}

--- И что я им должен излагать?

--- Чтобы \emph{тебе} было нечего рассказать? Ты же где только не бывал и с кем только ни встречался~--- неужели для Школы Дружбы о других народах ничего не сможешь? Насколько помню, ты даже разумеешь по-грифоньи.

--- А в этой группе есть грифоны?

--- Один есть.

--- Давай подробнее. Что за группа?

--- Шестеро. Все подростки среднего возраста. Галлус, грифон. Сильверстрим, гиппогрифина. Сэндбар, земной. Йона, як. Смолдер, дракониха. Оцеллия, чейнджлинг.

--- И чего же в этой группе особенного, что она такая маленькая вдруг на контроле сверху оказалась?

--- Ну{\ldots} всё же пятеро из шести~--- другие расы{\ldots}

--- И эти пятеро рассматриваются как потенциальные агенты влияния?

--- Э{\ldots} ну, вслух не говорится{\ldots}~--- у Старлайт хватило совести чуть покраснеть.

--- Понятно. Четверо крылатых, да плюс я{\ldots} если совсем без подготовки, то можно рассказать об особенностях полёта у разных рас. Вроде как будущие друзья и союзники должны же друг про друга знать-понимать?

--- Ну и вот! Сам сказал, сам и базу подвёл. Уж если кто про полёты и сможет хорошо рассказать, так это ты. Чего ещё нужно?

--- Два условия{\ldots} нет, три. Во-первых, не в классе, а на открытом воздухе{\ldots}

--- Чтобы при необходимости можно было на деле показать? Конечно.

--- Во-вторых, никого лишнего. Я затычка для дырки в расписании, а не приглашённая звезда. Эти шестеро, я{\ldots} ну и ты, наверное.

--- И наши мелкие. Плюс яблочную подружку ещё притащат, можно зуб дать.

--- Какого дискорда ты разболтала Свити, что идёшь ко мне?!

--- Ха! Вообще-то, это \emph{она} меня надоумила насчёт тебя.

--- Пороть. Вот и вся педагогика.

--- Не в этом случае. Они все трое почётные преподаватели Школы. Но кроме них~--- никого. А третье что?

--- Если хоть кто-нибудь, включая тебя, обзовёт меня «профессором»{\ldots}

--- То что?

--- Спроси у Твайлайт, чего было, когда она обозвала меня «светлостью».

--- Так ты по происхождению ещё и повыше «светлости» будешь{\ldots} молчу, молчу. Ладно, пошли на стадион тогда. Подождёшь там, а я их туда приведу и нужное внушение по дороге сделаю.
\begin{center}
	* * * * * *
\end{center}

Скуталу оторвала взгляд от листа и заметила:

--- Этого у тебя в памяти быть не могло. Интересно, а чего он сделал, когда Твайлайт его так назвала{\ldots}

--- Небось недельку отвечал на её вопросы буквально,~--- нетерпеливо отмахнулась Свити.~--- С такого кто угодно взбеленится, сама знаешь{\ldots} Дальше читай!

--- Читаю{\ldots}
\begin{center}
	* * * * * *
\end{center}

Стадион в этой их Школе очень даже ничего. Большой и~оснащён неплохо, тут чем угодно заниматься-тренироваться можно. Кольца для отработки манёвров тоже есть{\ldots} жаль только, что всё это совсем новое. Мне бы это добро пять лет назад, тут бы Скуталу у меня и полегла.

Кстати о птичках~--- троица на-свои-задницы-искателей поступила тактически грамотно и припёрлась на стадион вместе с учебной группой. Сообразили, что приди они сюда заблаговременно, так поимели бы отдельную лекцию{\ldots}

И расположились грамотно, позади всех. Старлайт покосилась на них, на меня, мученически закатила глаза. С некоторым сомнением сообщила:

--- Вот{\ldots} Как я уже сказала, на сегодня в расписании произошли некоторые изменения. Думаю, послушать профес{\ldots} профессионального спасателя будет хорошим уроком. Пожалуйста.

--- Всё-таки речь пойдёт не о том. Мы с вашим завучем поговорили и договорились вот до чего. Вас здесь шестеро, и четверо из вас крылатые. Семеро и пятеро, если добавить меня. Все представляют разные народы{\ldots} в случае со мной и Сэндбаром~--- разные расы. Наверное, тем, кто не умеет летать, полезно будет больше узнать о тех, кто умеет~--- а тем, кто умеет, поближе познакомиться с чужими взглядами на своё умение. Насколько я~понимаю, одной из задач этой школы является лучшее понимание друг друга?\looseness=-1

На что было отвечено кивками и согласным бубнением.

--- Очень хорошо. Давайте посмотрим на эти умения в последовательности от примитивного{\ldots}

--- Кхм-кхм{\ldots}~--- Это Старлайт, конечно.

--- {\ldots}я имею в виду ту последовательность, в которой умение летать было дано природой разным животным, от которых наши народы произошли вследствие различных магических катастроф. От более простых механизмов к более сложным. Когда-то давным-давно дар полёта первыми получили насекомые, и от них этот дар в почти неизменном виде перешёл к вершине их эволюции~--- нынешним чейнджлингам. Оцеллия, можно тебя попросить вот сюда{\ldots}

Она вышла.

--- Разверни, пожалуйста, крылья, чтобы всем было видно{\ldots} да, вот так. Смотрите. Крылья очень тонкие и очень жёсткие. Относительно, конечно. С учётом их маленькой толщины прочность крыльев чейнджлинга просто невероятна, но в абсолютном отношении они очень хрупки. Поэтому в нелетающем состоянии их бо́льшая часть защищена надкрыльями{\ldots} а зачем, по-вашему, природа создала их именно такими?

--- Лёгкость,~--- полуутвердительно заметила Смолдер.

--- Именно. Очень лёгкие крылья имеют малую инерцию, ими можно делать очень частые взмахи по очень сложным траекториям. Как следствие~--- очень точный полёт с возможностью очень замысловатых манёвров телом относительно его центра тяжести. Для чейнджлингов даже полёт задом наперёд не представляет никакой сложности{\ldots}

--- Многие умеют,~--- заметил Галлус.

--- Не говори за многих. Я не знаю точную статистику по другим народам, но среди пегасов летать задом наперёд умеет лишь каждый седьмой, и научиться этому непросто. У вас, грифонов, конечно, побольше.

--- Почему?

--- Об этом чуть позже, когда дело дойдёт до вас. А для чейнджлингов это совершенно рядовое умение{\ldots} кстати сказать, им в~этом смысле очень помогает специфическое круговое зрение их фасеточных глаз. Когда ты видишь сзади себя практически так же хорошо, как спереди, не возникает никаких психологических барьеров. В отличие от других рас. Итак~--- точность и манёвренность. К сожалению, этот механизм полёта имеет и слабые места. Оцеллия?

--- Скорость,~--- тихо сказала она.

--- Да. Полёт был дан чейнджлингам природой для строительства и обслуживания ульев с их сложной архитектурой. Здесь очень важна точность маневрирования и возможность надолго зависать в одной точке, а скорость просто не нужна. Как следствие, в боевом отношении{\ldots}

--- КХМ!

--- {\ldots}которое тоже нужно рассматривать, ибо иногда друзьям приходится заступаться за друзей! Чейнджлинг в своей настоящей форме не представляет сколько-нибудь серьёзной угрозы как отдельный воздушный боец. Представьте себе обычную сороку, охотящуюся на стрекозу, и вы всё поймёте. Однако на их стороне унаследованная от насекомых массовость и слаженность, а также их уникальная способность к трансформации. В кого превратится чейнджлинг, если всё-таки возникнет необходимость принять воздушный бой?

--- В кого выгоднее,~--- хором ответили два или три голоса.

--- Да. По принципу «камень-ножницы-бумага». Но здесь есть нюанс. Трансформируясь в какое-то существо, чейнджлинг приобретает базовые способности, свойственные скопированному виду. Полёт с неявным использованием магии, как например, у пегасов, может быть скопирован, ведь чейнджлинги владеют магией в некоторых пределах{\ldots} но всё-таки это копирование никогда не выйдет за пределы \emph{базовых} способностей вида. Скажем, если Оцеллия превратится в мою копию, то она сможет летать. За счёт того, что она скопирует мою физическую форму~--- летать довольно прилично. Пожалуй, чуть лучше среднего для пегасов уровня. Но мои знания и умения, мой опыт скопированы не будут, и в случае столкновения в небе со мной-настоящим у неё не будет шансов.

--- А если, допустим, в дракона?~--- заинтересованно спросила Смолдер.

--- Аналогично. У Оцеллии-копии-тебя не будет шансов против тебя-настоящей.

--- Нет, во взрослого сильного дракона против вас?

--- Дракон против пегаса~--- ничья. Но если я увижу, что имею дело не с подлинным драконом, а с~трансформированным чейнджлингом, то это даст мне определённое преимущество.\looseness=-1

--- Как увидите?

--- По манере полёта.

--- О! Про{\ldots} --- Старлайт осеклась и поправилась:~--- про узнавание по манере полёта~--- это правда?!

--- Да. То есть, конечно, в суматохе можно и проглядеть, но в спокойной обстановке и при возможности рассмотреть внимательно~--- без всяких проблем.

--- Хм-м-м{\ldots}

--- Хотите эксперимент?

Школота немедленно возбудилась так, что никаких сомнений в положительном ответе не возникло.

--- Пожалуйста. Мне показывают, я опознаю́. Нужно только, чтобы я не видел подготовку.

Старлайт прищурилась и засветила рог. Мне на голову опустился здоровенный колпак из плотной материи. Послышалось шушуканье, что-то вроде «ты{\ldots} нет, ты{\ldots} вот сюда{\ldots} поменяйтесь{\ldots} да какая разница{\ldots}».

Колпак приподнялся и отодвинулся в сторону. Передо мной в воздухе висела Смолдер в количестве двух штук. О чём говорили, то и представили{\ldots}

--- Дракон,~--- я указал на левую от себя и перевёл копыто на правую:~--- Чейнджлинг.

--- Почему?

--- Драконьи крылья не имеют оперения, и траектория взмаха очень хорошо видна в деталях. Взмахи как бы по перекрученной петле для дракона совершенно не характерны, а насекомые именно так и летают. Чейнджлинги, соответственно, тоже. В трансформации это почти незаметно, но если знать, на что смотреть{\ldots}

--- Ещё!~--- заявила Старлайт и снова напялила на меня колпак. Опять раздалось шушуканье. Кого подсунут теперь, гиппогрифа или грифона?

Гиппогриф, две штуки Сильверстрим.

--- Опять Оцеллия справа от меня.

--- Сейчас почему?!

--- Крылья зажаты на взмахах. Посмотрите сами, у Сильверстрим они свободно гнутся, а чейнджлинги гибкостью крыла в~полёте не пользуются, и это даже в трансформации можно заметить.

--- Блин! Ещё!~--- это уже не Старлайт, это хор.

Процедура повторилась. Ну, теперь-то уж понятно~--- грифона придётся опознавать.

На этот раз колпак просто исчез. Надо понимать, в связи с~исчерпанием материала для экспериментов.

--- Так нечестно,~--- сообщил я с интонациями Скуталу. Оная хихикнула из задних рядов.~--- Здесь нет чейнджлинга, только грифон и гиппогриф под наведённой иллюзией. Оцеллия вон там продолжает притворяться, что она не она, а Сильверстрим.

--- Блин, как?!

--- Легко. У них крылья движутся абсолютно синхронно, вплоть до каждого шевеления каждого пёрышка. Так вообще-то не бывает{\ldots} если, конечно, у присутствующего при сём мага не светится рог.

--- Почему под иллюзией именно гиппогриф?

--- А ни на ком другом из присутствующих такой синхронности добиться нельзя, слишком разные механизмы полёта. Разве что на Скуталу, но только она-настоящая могла захихикать на моё «так нечестно» минуту назад.

--- Предполагалось, что различать подопытных нужно только по манере полёта.

--- Предполагалось, что будут показывать пары «чейнджлинг плюс кто-то ещё».

--- Э-э-э{\ldots}

--- Давайте всё же вернёмся к нашим драконам, как раз их очередь. Если все примут свой настоящий облик и Смолдер выйдет вперёд{\ldots} Среди всех крылатых драконы уникальны, причём даже по двум причинам{\ldots}

--- А какая вторая?

--- Смотря что считать за первую. Во-первых, только драконы появляются на свет бескрылыми, обретая крылья при взрослении. Во-вторых, только у них соотношение между размерами детёныша и взрослого выражается таким гигантским числом. Я,~с~вашего позволения, буду говорить о взрослых драконах. Давайте посмотрим на драконьи крылья.

Смолдер без подсказки развернула их.

--- Считается, что предками драконов были какие-то существа, жившие между эпохой расцвета рептилий и появлением первых млекопитающих. Первыми \emph{крылатыми} млекопитающими были летучие мыши, чьи крылья очень похожи на драконьи~--- кожистые и перепончатые, а их размер и форма стали управляемыми в отличие от крыльев чейнджлингов. С~другой стороны, такие крылья гораздо тяжелее, и с учётом размеров относительно тела на них не особенно поманеврируешь. Драконы с их огромной силой могут лететь очень долго и далеко{\ldots} принимая во внимание миграции, занимающие важное место в их жизни, это и есть основное назначение драконьего полёта.

--- А всё-таки про молодых драконов?.. --- попросила Смолдер.

--- Между первой линькой, когда появляются крылья, и началом периода интенсивного роста у молодого дракона есть около трёх десятков лет, когда он или она может наслаждаться полётом со сложным маневрированием. Существует предположение, что это своего рода компенсация от природы, дающая возможность как следует освоиться с новообретёнными крыльями, раз уж от рождения такой возможности не было.

--- О! --- пробормотала Смолдер и сделала какой-то жест. Я~проследил её взгляд --- там Оцеллия яростно строчила в~тетради. Предупредил:

--- Это именно предположение, подтверждений ему нет. Однако и других соображений, которые логично объясняли бы появление и последующее исчезновение способностей, не нужных во взрослой жизни, тоже нет.

--- А в боевом отношении?~--- деловито поинтересовался Галлус. Старлайт закатила глаза, но промолчала.

--- Боюсь, я вас разочарую. Представления о драконах, как о~грозных воздушных бойцах, не имеют под собой оснований. Что является главным оружием дракона?

--- Пламя.

--- Конечно. Пламя, от которого даже магией защититься очень непросто. Взрослый дракон имеет очень ограниченные возможности маневрирования, зато у него есть гибкая шея, позволяющая точно направлять огненную струю. Но эта струя достаточно медленна и инерционна, от неё не так уж трудно уклониться{\ldots}

--- Даже чейнджлингу?

--- Даже. Во всяком случае, приняв форму другого крылатого существа. Драконье пламя великолепно подходит для обороны: можно создать такую огненную завесу, через которую никто не пробьётся и даже не приблизится. Двое драконов против целой толпы сделают это без всякого труда. Драконье пламя представляет собой поистине ужасное оружие против наземных целей~--- надеюсь, вам не доведётся такое увидеть. Но в индивидуальном воздушном бою от него мало пользы. Как я и говорил, одно крылатое существо в чистом небе против одного дракона~--- ничья. Дракон не даст к себе приблизиться, но и сам ничего сделать не сможет.

--- А двое против одного дракона?

--- Против двоих у одного дракона не будет шансов. Один свяжет его скоростными атаками и заберёт на себя всё внимание, другой зайдёт сзади и поразит единственное слабое место, я имею в виду глаза.

--- КХМ!~--- всё-таки не выдержала Старлайт.

--- Давайте всё же продолжим. Следующая ступень~--- гиппогрифы и пегасы. Сильверстрим, встань рядом со мной. Мы получили дар полёта от птиц в виде крыльев с перьями. Перья управляются сравнительно слабыми мышцами, и при этом позволяют менять конфигурацию крыла в довольно широких пределах. Перья позволяют летать бесшумно. Перья, в конце концов, имеют и защитную функцию. Судите сами: там, где чейнджлинг сломает своё крыло и останется инвалидом, там, где дракон получит пробоину в летательной перепонке и какое-то время не сможет летать, пока она не заживёт~--- мы имеем шансы отделаться лишь несколькими сломанными или вырванными перьями. Это неприятно, но летать не помешает. За всё, конечно, нужно платить, и у оперённых крыльев есть слабое место. Сильверстрим?

--- Дождь.

--- Да. Под дождём перья намокают и тяжелеют, тогда как с~крыльев драконов и чейнджлингов вода свободно стекает. А лететь с отяжелевшими крыльями{\ldots} это очень большая нагрузка. Обычно речь может идти лишь о считанных минутах.

--- Подождите!~--- встрял Галлус.~--- Это же всё и про нас тоже!

--- Есть одна существенная разница. Подожди несколько минут, ладно? А пока посмотрите на нас с Сильверстрим. Одинаковые размеры крыльев относительно тела{\ldots} и крылья, прямо скажем, невелики. Мы выкручиваемся двумя способами, один из которых общеизвестен: во всех книгах говорится, что полёты пегасов имеют в своей основе латентную магию{\ldots}

Согласное бубнение подтвердило общеизвестность.

--- Вам наверняка доводилось слышать или даже самостоятельно делать такой вывод: из всех трёх рас пони наиболее хвастливыми и самоуверенными являются пегасы. Это чистая правда, ведь единственным способом управления своей скрытой магией является уверенность в себе и своих силах. Чтобы было понятнее: жеребёнок-пегас, выросший в какой-нибудь глухой деревеньке, ничего сколько-нибудь впечатляющего освоить не сможет. Ему просто неоткуда будет узнать, \emph{как можно на самом деле}. А вот ходить по облакам он сможет без всякого обучения, ибо эта возможность абсолютно общеизвестна и она ни в одном крылатом существе не вызовет сомнения ни на долю секунды. Другим применением этой магии является управление погодой{\ldots} которому, кстати, невозможно научиться по книгам. А лишь одним и только одним способом: «смотри, как я делаю, делай как я!».\looseness=-1

--- Мы это не умеем,~--- заметила Сильверстрим.

--- Хорошее замечание. Если бы детёныш гиппогрифов с~рождения жил среди пегасов, то наверняка научился бы без особого труда. Вы просто не задумываетесь об этом, исторически направив свою латентную магию на другое. Я имею в виду способности к подводной трансформации.

--- То есть, если бы пегасёнок жил среди гиппогрифов?..

--- Да, скорее всего. Никто, конечно, не возьмётся ставить такой эксперимент целенаправленно, но если наши народы будут продолжать дружить, то он рано или поздно может получиться сам собой. Обратите внимание вот ещё на что: гиппогрифы ближе к птицам, чем пегасы, это видно с первого взгляда. Поэтому им легче обучиться некоторым лётным трюкам, но в детстве значительная часть их усилий уходит на обучение трюкам подводным. Это уравнивает нас в небе.

--- Кстати, о птицах,~--- на удивление уместно заметила Смолдер.~--- Птицы бывают разные.

--- Очень хорошо, что упомянула. Да, птицы бывают разные. Ласточки и колибри, коротко говоря. Скорость и порхание. Первое проще, ибо скорость и есть та вторая уловка, которая позволяет нам обходиться сравнительно небольшими крыльями. На высокой скорости они гораздо эффективнее. Научиться порхать сложнее. Много сложнее. И требует больше сил.

--- Летать быстро проще, чем порхать на месте?!~--- с явным недоверием переспросил Сэндбар.

--- Представь себе. Это два совершенно разных стереотипа. Абсолютно разных, вплоть до разной работы мозгов. Скажу даже больше: почти все пегасы способны научиться тому и другому~--- как летать быстро, так и порхать~--- а вот свободно переключаться в полёте от одного к другому~--- лишь две трети. Гиппогрифам должно быть попроще.

--- Тоже есть не умеющие,~--- откликнулась Сильверстрим.

--- Могу привести пример из собственного опыта. В своём первом дальнем самостоятельном полёте я вообще \emph{забыл}, что могу летать этими двумя разными способами и переходить от одного к другому, хотя уже умел. Попытался при сильном ветре сесть так же, как летел, на скорости, и эта забывчивость стоила мне синяка на крупе размером с суповую тарелку. Сидеть потом не мог три дня, зато больше никогда не забывал. Был я тогда на полтора или два года младше, чем вы сейчас{\ldots}

По воздуху подплыл стакан с водой~--- Старлайт озаботилась. Пока я пил, среди слушателей произошло некоторое шевеление.

--- Йона, вовсе не обязательно подпихивать Галлуса. Могла бы и сама спросить{\ldots} а могла бы даже и не спрашивать. Я уже давно понял, что вас двоих больше всего интересует вопросы боевых возможностей. Угадал?

--- Угу.

--- Пегасы и гиппогрифы хоть и не самые сильные воздушные бойцы, но самые универсальные. Совершенные крылья и~высокие скорости позволяют эффективно сражаться в полёте. Управление погодой позволяет пегасам работать и против наземных целей --- молнии из грозовых туч, создание смерчей и так далее. У~гиппогрифов этого нет, зато они могут жить и~драться под водой.

--- А между собой?

--- В воздухе и на земле~--- практически одинаковый уровень, всё определяется только опытом и мастерством. Под водой{\ldots} сами понимаете. Здесь гиппогрифам могут противостоять только чейнджлинги в какой-нибудь подходящей трансформации.

--- А когти и клюв{\ldots}

--- Легко компенсируются навесным оружием. Это лишь небольшой плюс к боеготовности, всё-таки они постоянно при себе.

Я снова отхлебнул. Галлус, не дожидаясь приглашения, вышел вперёд и сменил одноклассницу.

--- Итак, нам осталось рассмотреть лишь грифонов, которые являются лучшими мастерами по лётной части{\ldots}

--- Прямо лучшими?~--- скептически осведомился парень.

--- Да. Что тебя удивляет?

--- Пони презирают нас. Большинство.

--- Мы в Школе Дружбы, а не в армии на пятиминутке ненависти к потенциальному противнику, нет? \emph{К тому же я неправильный пони}.~--- Галлус прищурился: последнюю фразу я произнёс на его родном языке.~--- Презирать вас могут одни глупцы, не понимающие того, что вы превзошли нас отнюдь не только в~полёте и воинственности.

--- В чём ещё?

--- В поэзии, например. Со времён Гааргла Дерзкого ни один из эквестрийских поэтов и близко не подобрался к достигнутым им вершинам.

На физиономии парня настолько явственно нарисовался вопрос «чё?!..», что его одноклассники захихикали. Я вздохнул и~прочитал четыре длинных двустишия-бейта.

Теперь глаза расширились.

--- Это можно найти в библиотеке?~--- осведомился Галлус.

--- Школьной? В оригинале? Вряд ли, не думаю.

--- Яки не понимать грифонский!~--- озвучила Йона.

--- Этого вы точно ни в одной библиотеке не найдёте:

\begin{verse} 
\textsf{Считаться достойным может лишь тот, \mbox{кого не сгибает гнёт,}}\\
\textsf{Или же тот, кто не ведая сна, с гнётом борьбу ведёт.}

\textsf{То не решимость, если в душе нет силы \mbox{на смелый шаг,}}\\
\textsf{То не раздумье, если ему путь заслоняет мрак.}

\textsf{Жить в униженье, покорно глядеть \mbox{в лицо источнику зла~---}}\\
\textsf{Вот пища, что изнуряет дух и иссушает тела.}

\textsf{Низок смирившийся с этой судьбой, \mbox{подл, кто стремится к ней:}}\\
\textsf{В жизни бывает такая жизнь, что \mbox{смерти любой страшней.}}\footnote{Фрагмент стихотворения арабского поэта Х века Абу-т-Тайиба аль-Мутанабби в переводе С. Северцева.}
\end{verse}

--- Почему это не найти ни в одной библиотеке?

--- Потому что это мой собственный перевод. Я его нигде не публиковал, да и вслух читал всего два или три раза.

А ещё эти стихи были написаны как раз против эквестрийской гегемонии. Что, впрочем, совершенно не обязательно говорить вслух.

--- Я, конечно, извиняюсь!~--- вмешалась Старлайт.~--- Это замечательная тема, вполне достойная отдельного урока, но всё-таки стоит вернуться к тому, с чего начинали. По-моему.

--- Действительно, давайте вернёмся. Вот перед нами грифон. Их народ тоже получил дар полёта от птиц. Грифоны родственны гиппогрифам, но к птицам ещё ближе, а~от пони дальше. Очень многое из уже сказанного относится к~ним в~полной мере, как нам заметил тут Галлус. Но есть и~фундаментальное отличие.

--- У нас нет магии.

--- Ну, немножко всё-таки есть, вы ведь умеете ходить по облакам. Но да~--- ваше преимущество в том, что вы летаете одними крыльями, без всякой магии.

--- Это разве преимущество?

--- Безусловно. Посмотрим на это так: мы и грифоны умеем кое-что делать. Но нам для этого нужно ещё нечто дополнительное, а им нет. Очевидно, это означает, что умение грифонов более совершенно. Мы наглядно увидим следствие из только что сформулированного, если ваша завуч сможет сделать что-то, уравнивающее массу Галлуса с моей.

Завуч посопела и сколдовала. На грифоньих лапах появилось что-то вроде браслетов~--- утяжеляющие мешочки-«колбаски» с камнями или песком внутри.

--- Спасибо. Галлус, делай как я.

Я подпрыгнул и завис невысоко над землёй. Через секунду рядом висел Галлус.

--- Видите? Мы сейчас имеем одинаковую массу, но ему для удержания этой массы в воздухе приходится взмахивать крыльями заметно реже, чем мне. Почему?

--- Размеры,~--- озвучила Сильверстрим и уточнила:~--- Площадь.

--- Точно. Даже в абсолютном исчислении крылья грифона больше, а уж относительно размеров тела{\ldots} Подъёмная же сила крыльев увеличивается с их площадью. Грифонам её хватает, и магия для полёта им просто не нужна. Галлус, спускаемся{\ldots} и~утяжеление больше не требуется.

Спустились. Мешочки исчезли с чпоканьем. На физиономии парня была теперь написана откровенная жажда знаний.

--- Грифоны летают чисто по-птичьи. Чтобы управляться с~достаточно большими крыльями, нужна соответствующая сила, и она у них есть. Будучи из всех присутствующих здесь рас наиболее близкими родственниками птиц, они легче всех присутствующих учатся летать{\ldots}

--- Минутку!~--- неожиданно встряла Старлайт.~--- Если всё обстоит именно так, то почему молодые грифоны с давних пор стремятся попасть в эквестрийские лётные школы? Это ведь задолго до Школы Дружбы началось, несколько веков уже.

--- Во-первых, навигация. Мы её систематизировали и превратили в науку, которой глупо не учиться, если ты крылатый. Во-вторых, теория. Мы многое поняли о работе крыльев и о силах, действующих на тело во время полёта. Эти знания помогают шагнуть чуть дальше отпущенного природой через инстинкты и~рефлексы. Ну и есть ещё такая мелочь, совсем безделица: грифона, не окончившего лётную школу, не возьмут работать ни на почту, ни в сколько-нибудь серьёзную курьерскую компанию.

Школота хихикнула. А Старлайт, умница, хихикать не стала~--- сразу узрела в корень:

--- Грифона без бумажки не возьмут, а пегаса?

--- Пегаса возьмут, с испытательным сроком. Есть над чем работать, завуч, не так ли?

--- А гиппогрифа?~--- встряла Сильверстрим.

--- Вот уж не знаю. Просто ни разу не слышал, чтобы кто-то из ваших изъявлял такое желание. Давайте вернёмся к нашим грифонам и озвучим то, что сейчас явно интересует Галлуса. Говоря коротко, в воздухе они являются очень грозными бойцами. При прочих равных условиях в поединке «грифон против кого-то другого, кроме дракона» я поставлю на грифона.

--- Мы медленнее вас,~--- заметил Галлус.

--- Совсем немного. Бо́льшие крылья создают большее сопротивление воздуха. Зато вы маневреннее на низких скоростях, это очень важно. Далее, хорошо развитые пальцы на передних лапах позволяют пользоваться эффективным немагическим оружием. Все, надеюсь, знают, что меч придумали именно грифоны? Вот. Даже грифоний язык имеет значение: команды на нём коротки, исключают двойное понимание и звучат так, что хорошо слышны даже сквозь шум ветра. Как я говорил, при этом он и для стихов прекрасно подходит, да и великий трактат «Искусство войны»\footnote{Трактат, обсуждаемый в этой и следующей главах, дословно или очень близко к тексту повторяет «Искусство войны» Сунь-цзы (написанный между VI и~III~вв. до~н.э.) в переводах Н. Конрада и Ю. Кулишенко.} был написан именно на нём{\ldots} Галлус, я всё сказал, про что ты хотел бы услышать?

--- Наверное{\ldots}~--- пожал тот плечами.~--- У вас как-нибудь потом можно будет ещё спросить{\ldots} в случае чего?

--- Обратишься к завучу Старлайт, она подскажет, как со мной встретиться{\ldots} Значит, мы рассмотрели умение летать от простого к сложному. Остался ещё один случай, так сказать, вне конкурса. Йона, Сэндбар, выходите вперёд.

--- Мы-то какого сена?!

--- Йона бояться высоты!

--- КХМ!!!

Мда, сюрприз удался.

--- Никакой высоты не будет. Будет всего лишь небольшая проверка{\ldots} и вам ничего не понадобится делать.

Вышли-таки.

--- Знаете вы или нет, но существует заклинание, позволяющее на время дать крылья земному пони или единорогу. Вместе с возможностью ими пользоваться, разумеется. Знаете?

Судя по тому, что все дружно обернулись на Старлайт, и ей пришлось кивнуть, это оказалось новостью.

--- Очень хорошо. Значит, придётся честно подумать, чтобы ответить. Вопрос первый: \emph{какие} крылья даст это заклинание?

Теперь все обернулись на Оцеллию. Сойдёт за ответ.

--- Да. Крылья насекомого, как самый простой из существующих вариантов. Не удивлюсь, если это заклинание сродни магии чейнджлингов. А теперь второй вопрос, и я хотел бы, чтобы на него ответил кто-то не имеющий крыльев. Если заклинание реализует самый простой возможный вариант, то почему оно такое сложное? Насколько знаю, сотворить его может лишь маг уровня вашего завуча или директора.

Сэндбар изобразил некий неопределённый жест:

--- Ну{\ldots} Если мне приделать крылья, то вместе с ними надо же приделывать и умение ими махать? И знание, как именно махать? То есть, я хочу сказать, если мне просто приделать ещё одну пару ног, я же ими двигать-то не смогу, а если и смогу как-то, то не факт ещё, что в них не запутаюсь{\ldots}

--- Правильно. В общих чертах всё именно так и есть. Для полёта нужны крылья, для крыльев нужны управляющие ими мускулы, для них нужны управляющие ими нервы, да ещё и обновлённой нервной системе нужны «зашитые» в неё рефлексы, чтобы всё это работало именно так, как надо. Фактически, речь идёт о создании нового существа на основе пони{\ldots} Не знаю даже, подействует ли оно на яка.

--- Йона бояться высоты!!!

Оно и видно. Причём сильно. Как будто её прямо сейчас хотят в полёт отправить{\ldots}

--- Тем не менее, знать о такой возможности нужно, пусть она и довольно экзотическая. Вот теперь я рассказал всё, что хотел рассказать. Надеюсь, это было интересно.

--- Интересно,~--- подтвердил со своего места Галлус. И тут же в чисто национальной манере добавил вредным голосом:~--- В теории{\ldots}

--- Хочешь практики? Ну{\ldots} если присутствующая здесь Скуталу не против размяться{\ldots}~--- должна же быть от неё хоть какая-то польза, раз пришла?

--- Ну, а чё{\ldots}~--- с деланным равнодушием отозвалась мелкая.

--- Меня не учили,~--- тут же отпёрся грифон.~--- Позориться не буду.

--- Я, я, я!~--- запрыгала Сильверстрим.~--- Меня учили! Можно?!

Я вопросительно глянул на Старлайт. Та кивнула. Странно.

Никогда не слышал о гиппогрифах как о воинственном народе. Практически это означает, что за последнюю тысячу лет никто из них на этом поприще ничем не отличился. С другой стороны, завучу виднее.

--- Скуталу?

--- А чё{\ldots}~--- повторила та. Шагнула вперёд:~--- Прямо здесь, что ли?

--- Вон два облачка, по высоте одинаковы, занимайте. Потом вам госпожа завуч чего-нибудь просигналит, и начинайте.

Пока поединщики занимали исходные позиции, я подошёл к Старлайт. Тихонько поинтересовался:

--- Уверена? Кто её там учил-то?

--- У неё отец такой{\ldots} говоря по-нашему, из военно-гвардейской династии. На этой почве там даже семья распалась. Сам-то за свою уверен? Уже ведь почти подростком на крыло встала, поздно совсем, а Сильверстрим{\ldots}

--- Сейчас увидим. Сверкни им, они вон уже на местах.

Госпожа завуч поджали губы и сверкнули. Симпатичная такая вспышка, явно что-то разработанное для Трикси{\ldots}

Обе спрыгнули со своих облаков и буквально качнулись навстречу друг другу~--- небо будто прочертили два гигантских маятника.

Встречная атака. У мелкой чуть быстрее, она ведь по сравнению с противником действительно мельче и шустрее. К некоторому моему удивлению, Сильверстрим очень толково ушла от хитрого финта в момент сближения. Правда, сама при этом ничего выдать уже не смогла.

Чему-то её действительно учили, факт.

Развороты на новое сближение. Кто разворачивается быстрее, тот получает инициативу, раньше видя противника. Можно даже успеть подловить его, пока он ещё сам до конца не развернулся.

Мелкая вошла в вираж чуть ли не вертикально, с разворотом чуть ли не на са́мом кончике крыла.

--- Щас свалится!~--- это Смолдер озвучила факт, очевидный любому крылатому.

Конечно, щас свалится. Чтобы так раскорячиваться, нужно очень энергично тормозиться, а чем потерять скорость на развороте, так уж лучше{\ldots}

Разные народы заканчивают эту поговорку по-разному. Я знаю пять вариантов, и ни один из них приличным не назовёшь. Скуталу, кстати, их тоже знает{\ldots} по меньшей мере три.

Впрочем, какой вариант ни выбирай, а заканчивается такая затея всегда стандартно.

И закончилась.

Мелкую закувыркало в штопоре. Сильверстрим, аккурат в~этот момент закончившая свой разворот (достаточно быстро, ничего не скажешь, как на экзамене) увидела это, перевалилась через крыло и ушла в пике, догоняя.

Уж конечно, не затем, чтобы спасать-помогать: запас высоты достаточен, чтобы три или четыре раза из того штопора успеть выйти.

А выход неизбежно содержит в себе оч-чень неприятный момент. Он, выход то есть, основан на том, что нужно превратить неконтролируемое падение в контролируемое, из собственно штопора выскочить в пикирование, а затем уж и выходить в горизонт. И когда ты выскакиваешь, то совершенно беззащитен~--- ни на что другое просто не отвлечёшься. Избавиться от этого момента беззащитности нельзя, его можно только сократить до некоторых пределов.

Сильверстрим сейчас явно намеревалась этим моментом воспользоваться. И подгадала очень точно~--- когда кувыркание замедлилось, она как раз оказалась на хвосте у мелкой. Как вращение прекратится, так и делай с ней, что хочешь!

То есть, \emph{если} вращение прекратится. Очень важный нюанс.

Потому что Скуталу выходить в пике не стала. Вместо этого её унесло в ещё один сильно расширенный виток, и пока она его там крутила, набравшую скорость Сильверстрим пронесло мимо. А когда виток был докручен, роли коренным образом поменялись, и теперь уже мелкая зависла буквально на хвосте противника.

Оставалось только щёлкнуть зубами, всё остальное сделала разность скоростей.

С неба раздался визг. Я посмотрел на госпожу завуча и шевельнул плечом. Дескать, а чего ждали-то?

Мама, увидь она этот свой приём в исполнении мелкой, была бы довольна. Очень чисто, я бы сам так изящно не смог, крупноват всё-таки.

Скуталу спустилась первой, со смачным «тьфу» выплюнула свидетельство своей победы~--- бело-голубую прядь~--- и подошла к подругам. Сильверстрим приземлилась чуть в стороне и нервно дёрнула покоцанным хвостом. Очевидно, сие означало «не лезьте ко мне сейчас»{\ldots}

Галлус поморщился. Хм, а парень к ней явно неровно дышит.

--- Вот вам практика. Что мы сейчас увидели и узнали? Йона, Сэндбар?

--- Случайность?~--- неуверенно предположил Сэндбар, явно болевший за одноклассницу.

Галлус и Смолдер хмыкнули. Эти-то правильно поняли: тот, кто способен на такой выкрутас, в штопор с виража случайно не валится.\footnote{В нашем мире приём воздушного боя, основанный на имитации сваливания в штопор, был придуман, применён и описан Александром Покрышкиным.}

--- Нечестная победа. Неправильная,~--- поддакнула Йона.

Теперь хмыкнул я. И зачёл по памяти:

--- Война~--- это путь обмана. Поэтому, если ты можешь что-нибудь, покажи противнику, будто не можешь; расставь приманки, чтобы враг соблазнился, изобрази беспорядок и сокруши его; показав свою ничтожность, распали в нём гордыню; нападай на него, когда он не готов. Горе тому, кто обманулся.

Покосился на Галлуса и повторил в оригинале. Глаза парнишки блеснули.

--- \emph{Ещё}!~--- хрипло каркнул он по-своему.

Я немного помедлил, подбирая подходящее место. Продолжил опять для всех:

--- Тот, кто хорошо обороняется, прячется в глубины Тартара, но тот, кто хорошо нападает, разит с высоты небес. Схватываются с противником правильным боем, побеждают же манёвром. Поэтому тот, кто хорошо владеет манёвром, безграничен подобно небу и земле, неисчерпаем подобно морям и океанам. То, что позволяет ветру нести на себе грозу, есть его мощь. То, что позволяет быстрому соколу поразить свою жертву, есть расчёт удара. Поэтому у того, кто хорошо сражается, мощь стремительна и расчёт молниеносен. Мощь подобна взведению катапульты, расчёт же подобен прицельному выстрелу из неё{\ldots}

Из школы раздался звонок{\ldots} как назло, на самом интересном и важном месте. Старлайт, по которой было видно, что она вся как на иголках, подскочила:

--- Так! Урок, как видите{\ldots} то есть слышите{\ldots} окончен, сегодня вы услышали{\ldots} и увидели{\ldots} много нового и обязательно должны это обдумать, дальнейшие занятия по расписанию!

Зыркнула на мелких (включая и наших) так, что те поторопились удалиться, и буквально за их хвостами отгородилась «пологом тишины». Я у неё это заклинание уже на слух узнаю́{\ldots}
\begin{center}
	* * * * * *
\end{center}

--- Тут обрывается, --- сообщила Скуталу. --- Но это явно не конец, дальше должно быть что-то ещё{\ldots} Другой кусок, который мы видеть-слышать тоже не могли.

--- Есть, и самое интересное, --- Эппл Блум кивнула на свою сумку. --- После обеда.

--- И мне копию! Свити?

--- Сделаю. Зачтём остальное, а потом сразу всё целиком и~сделаю. Неужели час-полтора не потерпишь? Странно только, что он фестралов ни словом не упомянул{\ldots}

--- Да понятно,~--- пожала плечами Скуталу.~--- Не было их там, вот и не стал говорить. И так-то времени едва хватило{\ldots}

--- А что это за поговорка про скорость на развороте?

--- Ес­ли я вам ска­жу, ва­ши сёс­тры ме­ня убь­ют.

%TODO 25 апреля здесь



\chapter*{Глава пятая, в которой Меткоискатели узнаю́т, что происходило у них за спинами после того урока}\addcontentsline{toc}{chapter}{Глава пятая, в которой Меткоискатели узнаю́т, что происходило у них за спинами после того урока}

--- Следующие листы давай! --- потребовала Скуталу, едва они вернулись в домик. --- Хм, на чём бишь я остановилась{\ldots} --- она глянула на первый из поданных листов и вернулась к~последним строчкам предыдущего, уже прочитанного, озвучив их ещё раз.
\begin{center}* * * * * *\end{center}

--- Так! Урок, как видите{\ldots} то есть слышите{\ldots} окончен, сегодня вы услышали{\ldots} и увидели{\ldots} много нового и обязательно должны это обдумать, дальнейшие занятия по расписанию!

Зыркнула на мелких (включая и наших) так, что те поторопились удалиться, и буквально за их хвостами отгородилась «пологом тишины». Я у неё это заклинание уже на слух узнаю́{\ldots}

Повернулась ко мне. Вздохнула:

--- Твайлайт меня убьёт{\ldots}

--- Если бы каждый раз, как я это от тебя слышу, мне давали монетку, то я был бы богаче на двадцать три бита. То есть уже на двадцать четыре.

--- Ты не понимаешь. На этот раз она меня \textbf{убьёт}.

--- Тебя-то за что?

--- Так тебя-то небось побоится. И ты в Школу не сам пришёл. А кто тебя в неё притащил?

--- А кто тебя вынудил кого-то притаскивать?

--- «Я начальник~--- ты дурак», слышал такое?

--- Слышал. Если помнишь, я вам ещё в Холлоу-Шэйдс говорил, что самым большим дураком в итоге оказывается тот, кто считает дураками других.

--- Ты ещё посоветуй поцитировать ей твоего любимого генерала Напониона.

--- А что? И поцитируй.

Старлайт вздохнула:

--- Ладно, как-нибудь отбрешусь. Скажи лучше, чего ты к Галлусу прицепился? С середины урока практически всеми речами в него целился.

--- Так он грифон.

--- И что?

--- Допустим, я из всех не-пони их народ лучше всех понимаю. Язык знаю как родной и в их литературе разбираюсь получше их самих{\ldots} ныне живущих, по крайней мере. Большинства.

--- Убедительно. Но далеко не исчерпывающе.

--- Как следствие, мне на них больше всех не наплевать. Родители кое-чем обязаны грифонам{\ldots} считай, что я отдавал старый должок, пытаясь его о чём-то вразумить.

--- А ещё? Ни за что не поверю, что в этой твоей проповеди не было дополнительных смыслов.

Отдать ей должное, она меня неплохо изучила. Что ж{\ldots} хочет дополнительных смыслов~--- будут ей дополнительные смыслы. Вот только за это нас уже не одна Твайлайт убить может. Ещё и~кто повыше.

--- Реку Гато знаешь?

--- Ну. Впадает в море как раз напротив Мэйнхэттена на том берегу пролива.

--- В её устье есть небольшой рыбацкий городок, язык сломаешь выговаривать. А в нём на центральной площади памятник{\ldots} ну как памятник, простая стела. «От благодарных жителей~--- крылатым спасателям». И ниже стихи:

\begin{verse}
\textsf{Ни почестей, ни награды,\\
Никто тебе не поможет,\\
Взлетел, и делай что надо,\\
А сил не хватит --- что должен.}

\textsf{И пусть мозоли кровавы ---\\
Твой труд до седьмого пота\\
Оценит с усмешкой равный:\\
«Хорошая, брат, работа!».}\footnote{Фрагмент стихотворения Ю. Ломова.}
\end{verse}

--- Твои, что ли?

--- Нет, конечно. Сто девяносто восьмой год, я тогда разве что детские дразнилки сочинял. Тебе в этом описании ничего не кажется странным?

--- Кажется. «Ни почестей, ни награды» и «от благодарных жителей»~--- странное сочетание, мягко говоря. С каких это пор Эквестрия оскудела на почести и награды заслужившим?

--- Так. Ещё?

--- «Крылатым спасателям». Как-то пафосно{\ldots} хотя стоп! Если там были не только пегасы, тогда вполне нормально.

--- Правильно мыслишь. Пегасов там было только четверо{\ldots}

--- И все ваши?..

--- Да.

--- Подожди, что-то припоминаю. Устье Гато и сто девяносто восьмой год{\ldots} это же по обычному счёту{\ldots} большое извержение, что ли?

--- Именно. Лавовый поток пошёл на город, земные кое-как соорудили насыпь, единороги укрепили её щитом. На какое-то время поток от города отвели, но он оказался в кольце, и крылатые стали вытаскивать народ по воздуху. Успели.

--- Крылатые~--- кто?

--- В основном грифоны. К которым в Эквестрии сама знаешь какое отношение. Вот и{\ldots} отразили в надписи.

--- К чему ты это сейчас рассказал?

--- Они неплохие ребята. И мне не нравится, что их гнобят практически на уровне государственной политики. С народом, давшим миру таких поэтов, как Гааргл Дерзкий, и таких философов, как безымянный автор «Искусства войны», недолго и доиграться. Про это, кстати, в трактате тоже есть.

--- Минуточку. «Низок смирившийся с этой судьбой, подл стремящийся к ней{\ldots}»~--- это то, о чем я сейчас думаю?~--- Старлайт посмотрела куда-то вверх.

--- Если ты думаешь то, что я сейчас подумал{\ldots}~--- я тоже посмотрел,~--- то да.

--- Вот поэтому~--- Школа Дружбы.

--- Вот поэтому~--- такой урок получился.

--- Вот за него-то нас и убьют.

--- Ты так и не объяснила толком, за что.

--- О, да просто так!~--- опять же, отдать ей должное, сарказм у Старлайт получается просто бесподобно.~--- Подумаешь, ты всего-то превознёс грифонов и их литературу перед другими расами, воспел до небес военное искусство с наглядной демонстрацией{\ldots} на минуточку, в Школе \emph{Дружбы}!.. и настропалил парнишку, не умеющего даже драться, до такой степени, что он вот-вот прибежит к тебе в ученики проситься.

--- Не возьму же. Грифонов нужно совсем по-другому учить, если ты меня слушала. И~делаю вывод, что обсуждаемый трактат ты не читала.

--- Что изменится, если прочту?

--- Спасибо мне скажешь.

--- Можно краткий анонс, за что именно?

--- Чего только некоторые не делают, лишь бы книжки не читать{\ldots} Ну вот. Мощь подобна взведению катапульты, расчёт же подобен прицельному выстрелу из неё{\ldots}

--- Это ты говорил, я помню.

--- {\ldots}Поэтому разумный рассчитывает победу, исходя из мощи соратников, а не требует от них невыполнимого, жертвуя ими. Пользуясь единством сил, он уподобляет себя и соратников камням и брёвнам. Ибо такова природа камней и брёвен, что на ровном месте из них можно строить, но со склона горы они катятся, сметая всё на своём пути. Поэтому разумный, окруживший себя верными соратниками и правильно их оценивший, в~своей мощи подобен высокой горе, с которой достаточно лишь толкнуть камень в нужную сторону, определяемую расчётом.

--- Однако{\ldots}~--- пробормотала Старлайт.~--- Это же очень близко к тому, что нам предполагается{\ldots}

--- Вот-вот. Всё-таки прочти. Не знаю за школьную библиотеку, а в кантерлотской «Искусство войны» есть. Даже в нескольких переводах, я интересовался.

--- Ты сегодня какой цитировал?

--- Свой.

--- Тьфу. Ну, значит, запишешь.

--- У меня в комнате синяя тетрадь на полке.

--- Ты что, заранее к этому готовился?!

--- Нет, конечно. Просто записал, как закончил.

--- Не врёшь? Впрочем, ты-то не врёшь{\ldots} Значит, я сделаю копию.

--- Делай, пользуйся на здоровье{\ldots}

--- Чтобы этим пользоваться, нужно ещё один момент прокомментировать. Сам догадаешься, какой?

--- Легко. Ты хочешь спросить, как народ, додумавшийся до сих замечательных мыслей, оказался в такой дыре. И если преподать их другим народам, не приведёт ли оно их туда же.

--- Это самое.

--- Они слишком поздно поняли, что всё это применимо не только к войне. А \emph{кое-кто} об этом даже и не задумывался. Раз грифоны возвели войну в искусство~--- значит, по-хорошему с~ними и нельзя. Самое ироничное в том, что их загнали в дыру непрямыми действиями, рассуждениям о которых посвящена вся пятая глава того же трактата. У \emph{кое-кого} был слишком большой опыт.

--- Чувствую, мне действительно придётся это прочитать.

--- Синяя тетрадь у меня на полке. Читай, обдумывай, преподавай. Вот тебе и затычки для дырок в учебном процессе.

--- Почему я?

--- А кто?

--- Ты перевёл, ты и преподавай. Время на обдумывание у тебя, очевидно, было.

--- Так завтра выходной.

--- Послезавтра.

--- А послезавтра,~--- ухмыльнулся я,~--- у меня начинается рейсовая декада{\ldots}
\begin{center}
	* * * * * *
\end{center}

--- Вот, значит, как{\ldots} Понятно теперь, чего наверху всё так завертелось, как наскипидаренное,~--- пробормотала Скуталу.~--- Мало ей было чейнджлингов, так наши ещё и грифонов подсиропили{\ldots}

--- Кстати, а у твоего с грифонами чё? Вон, родителей ещё упоминал{\ldots}~--- поинтересовалась Эппл Блум.

--- Без понятия. Но когда нас в почётные грифоны потом принимали, он только сбоку стоял. Меня принимали, Свити и Старлайт как организаторов, даже Твайлайт как директора Школы{\ldots} а он сбоку~--- типа, и ни при чём тут.

--- Дык, наверно, уже.

--- Запросто. Свити, ты ж рассказывала, он в Холлоу-Шэйдс о них хорошо отзывался?

--- Было дело. Мол, они ребята ничего, если им показать, что сам можешь на них начхать, а на себя не позволишь. Практически комплимент. И потом ещё{\ldots} мол, они при всех своих национальных особенностях бояться не любят и не умеют, а страшно в~итоге становится тем, кто вздумает их пугать.

--- Эт’ фигня{\ldots}~--- пробормотала Эппл Блум.~--- А всё-таки, не было ли тут ответного хода в этой их ляганной Игре? Мир-дружба-морковка с грифонами, да трактат переведённый и переписанный на полочке давно лежит, да другая похожая проблема как раз поджимает, а тут такая возможность подвернулась! Они ж давно знакомы, я так поняла, больше века{\ldots}

--- Паранойя,~--- сказала в пространство Свити Белль.

--- Агась. Мне вот прям так и объясняли: сначала совпадение, потом много совпадений, а потом она са́мая. Скутс, а ты-то чё дутая сидишь?

--- Дуюсь потому что,~--- мрачно буркнула та сквозь зубы.

--- С чего?!

--- Да блин{\ldots} Никогда мне его в лужу не посадить. Ни на крыльях, ни на словах.

--- Чё?!~--- Эппл Блум и Свити недоуменно переглянулись.~--- Ты о чём вообще?

--- О том! Перед тем эксдерьментом, когда ты спрашивала, что тут происходит, помнишь? Твоя сестра чего-то сказала про детей, а \emph{она} в ответ разразилась, мол, эти дети хороши пока маленькие, максимум в лужу посадить могут, а как вырастут, так бла-бла-бла, сказочку ещё старую упомянула{\ldots} Вчера вот только читали.

--- Ну и?

--- Мне потом про те дела рассказали кой-чего. Это, оказывается, Твайлайт \emph{её} реально в лужу посадила на уроке, в школе ещё. Я тогда своего и спросила~--- мол, а ты чего делать будешь, если я тебя так посажу когда-нибудь? А он на меня посмотрел как на дуру и сказал, что будет в той луже прыгать от радости. Что не зря со мной возился. Ну, я себя и почувствовала как дура. Вот.

--- Э-э{\ldots}~--- Эппл Блум и Свити опять переглянулись. Свити осторожно спросила:~--- Это ты поэтому витраж и грохнула?

--- И поэтому тоже. А, неважно, забудьте{\ldots} Копию-то мне можно?

Свити Белль засветила рог. Через секунду бумажные листы сложились аккуратной стопкой, их окутало сияние, а когда оно погасло, стопка оказалась в два раза толще. Единорожка быстро разобрала её по листу влево-вправо.

--- Кстать, спросить хотела,~--- заметила Эппл Блум.~--- А чё вы{\ldots} ну, то есть маги, из книжек нужные места переписываете, если можете вот так?

--- Так переписанное в голове остаётся, а копию можно забыть или потерять, и ничего не останется. Дальше-то что?

--- Дальше опять моё пойдёт. В смысле, про себя я сама написала. Будем читать по утрам, там довольно много, а после обеда читанное править буду. И это{\ldots} я хотела попросить, шоб кто-нить из вас читал. Ну, шоб я со стороны своё слышала, мне так удобней.

Теперь переглянулись Скуталу и Свити Белль.

--- Да не вопрос,~--- пожала плечами Свити.~--- На сегодня, что ли, всё?

--- Агась. Мне ещё кой-чего перепроверить{\ldots}



\chapter*{Глава шестая, где повествуется о~том, как некоторое время тому назад обрела наставницу последняя из Меткоискателей}\addcontentsline{toc}{chapter}{Глава шестая, где повествуется о~том, как некоторое время тому назад обрела наставницу последняя из Меткоискателей}

--- Вот!~--- радостно сообщила Эппл Блум, выкладывая на стол пачку листов.~--- Сёдняшняя порция{\ldots} и лучше, наверно, шоб ты читала, Свити.

--- Почему?

--- Дак там дворцовые разговоры. А по части политесов у нас ты{\ldots} ну, в общем, понятно. Увидишь чего не такое, говори сразу. Я, конеш, старалась.

--- Хм-м-м{\ldots}~--- Свити Белль глянула в верхний лист. Текст был машинописным, что сильно упрощало чтение.~--- Подожди, а чего это ты не от первого лица? Про себя писала ведь.

--- И так первых лиц много. А ещё мне сказали, шо настоящему автору должно быть не слабо́ о себе в третьем.

--- Где-то я это уже слышала{\ldots}~--- пробормотала Скуталу.

--- Ша! Читаем, слушаем,~--- скомандовала Эппл Блум.~--- Ежли кто замечает косяк, то грит сразу!
\begin{center}* * * * * *\end{center}

--- Тэ-эксь{\ldots}~--- Эпплджек подняла взгляд с письма на младшую сестру.~--- Докладай, чё на этот раз натворила?

--- Чё сразу натворила-то?!

--- А с чего бы ещё тебя вздрючивать собирались?

--- Чё сразу вздрючивать-то?!

--- Превентивно хотя бы,~--- важно произнесла Эпплджек.

--- Чё?!..

--- На всякий случай,~--- перевёл доставивший письмо Виндчейзер.~--- Твоя сестра хочет сказать, даже если вы ничего не натворили~--- это ещё не значит, что вы ничего не натворили. Потому что либо уже натворили и про это просто пока не знают, либо всё равно натворите самое позднее послезавтра.

--- Эт’ самое,~--- подтвердила старшая.~--- Так что колись. Чё эт’ тебя на вздрючку аж в Кантерлот вызывают?

--- Чё сразу вздрючку-то?! Где тот Кантерлот и где я торчу последние три недели, как в школе четверть началась?!

Эпплджек почесала затылок: это было правдой. Перевела взгляд:

--- А ты-то знаешь, чего доставил?

Виндчейзер кивнул:

--- Знаю. Приглашение Эппл Блум прибыть завтра в десять утра, и далее по мере необходимости, в кантерлотский дворец для разговора с их высочествами о важном деле.

--- И чё тогда было письма расписывать, да аж со всеми подписями-печатями? На словах бы и сказал.

--- На словах попросили сказать, что с собой специально брать ничего не нужно. А эта бумага с подписями-печатями для отмазки от школы вроде как.

--- И отмазать мог бы на словах.

--- В вашей школе от меня шарахаются и пытаются чертить в~воздухе какие-то символы. Зебриканские шаманы такими злых духов отгоняют.

--- Лихо. Чем заслужил?

--- Пару раз отмазал Скуталу.

--- Весело с вами{\ldots} Ну ладно, а чего её туда вызывают, не в~курсе?

--- Знаю только, что это как-то связано с тем экспериментом на задворках вашей фермы. Вы его ещё «эксдерьментом» называете.

--- О как{\ldots}~--- Эпплджек опять почесала затылок.~--- Ну тады ой. В смысле, по этой части вродь как и впрямь дрючить не за что, мелкие там тока сбоку стояли. Кто отвезёт-то её, ты?

--- Нет, зачем? То есть могу, конечно, но проще Старлайт попросить сразу во дворец перебросить.

--- А там её{\ldots}

--- А там с неё, можешь мне поверить, ни на секунду глаз не спустят. И после того, как во дворце побывала Свити Белль, ни секунды лишней не задержат.

--- Блин, да чё там Свити-то наколбасила, хоть бы сказали уже?!~--- потребовала Эппл Блум с интересной смесью жалобных ноток и жадной заинтересованности в голосе.

--- Насколько я знаю, довела до белого каления всю дворцовую прислугу со стражей и кокнула вазу.

--- Всего-то?

--- Любимую вазу Со{\ldots} Селестии.

--- Вау{\ldots}~--- прошептала Эппл Блум. На этот раз с лютой завистью.

--- Слышь,~--- предостерегла её сестра,~--- ты, если вздумаешь превзойти, то имей в виду. Шо никакого переходящего приза за энто дело не получишь, а получишь кой-чего другого, усекла?

--- Тож за приз сойдёт, ради такого дела можно и потерпеть{\ldots} ладно, ладно, шучу.

--- Получишь так, что не сойдёт долго, я без шуток предупредила. А назад её как?

--- А про это пусть у них самих головы и болят. Вот же тебе официальная бумага: пригласили. Сами пригласили, сами пусть и огребают. Небось, после Свити должны понимать, на что идут.

--- Мне-то чё делать?~--- вернула разговор в прежнее русло Эппл Блум.

--- Ничего особенного. Завтра к полдесятому приходи в за́мок Твайлайт, мы тебя встретим и Старлайт перебросит. Там тоже встретят и объяснят, что дальше. Желательно не опаздывать, конечно.

--- Да я ж теперь не опоздаю просто потому, что не усну{\ldots}
\begin{center}$\triangleleft\star\triangleright$\end{center}

Из вредности Эппл Блум побилась с собой об заклад, что не придёт к за́мку раньше половины десятого~--- и, к немалому собственному удивлению, выиграла. В замке этим фактом тоже были слегка удивлены, но вслух высказывать ничего не стали и сразу перешли к краткому инструктажу:

--- Старлайт перебросит тебя в мою комнату. Просто выглянешь наружу, там кто-нибудь будет ждать, и тебя проводят куда надо.

--- Стоп, эт’ чё? Это ради меня там кому-то полчаса на посту придётся торчать?

Виндчейзер и Старлайт переглянулись и рассмеялись.

--- После твоей подруги,~--- пояснила отсмеявшаяся первой Старлайт,~--- я не удивлюсь, если этот пост выставили уже через минуту, как только вчера было дописано письмо с приглашением. Причём тот, кто на нём стоит, попал туда не как-нибудь, а в~наказание.

--- Ещё и дразнятся,~--- буркнула Эппл Блум.~--- Так нечестно вообще-то, мне ж сестра запретила Свити переплёвывать{\ldots}

--- А слабо́ переплюнуть так, чтобы не влетело?

--- Тебе-то что?

--- Так ведь это, если ты помнишь, давняя традиция моей семьи~--- её величество радовать.

--- Хватит!~--- остановила Старлайт.~--- Нечего её подстрекать{\ldots} она, если приспичит, и без подстреканий обойдётся. Ну, готова?

--- Агась.

Хлоп!
\begin{center}$\triangleleft\star\triangleright$\end{center}

Эппл Блум огляделась. Судя по всему, её выбросило в типичные гостевые апартаменты дворца~--- нетипичной здесь была разве что огромная карта в половину стены, какие-то хитрые инструменты на столе (опознавались только циркули) и преизрядный запас бумаги с карандашами. Больше ничего необычного не нашлось. Кобылка пожала плечами и вышла в коридор.

Прямо напротив двери сидя спал стражник-единорог. Его копьё было прислонено к стене.

--- Буп!!!~--- весело воскликнула Эппл Блум и дотронулась копытцем до носа стражника.

--- А-а-а!!!~--- спросонья взвизгнул тот, подскакивая. Задетое им копьё свалилось и древком съездило шутницу аккурат по крупу. Та, впрочем, нисколько не впечатлилась: дело было житейское и довольно привычное. Так же весело сказала:

--- Привет! Вродь как, это меня вы тут ждёте.

--- Э{\ldots} по описанию подходишь,~--- признал стражник, встал и~поправил на себе амуницию. Подхватил копьё.~--- Пойдём тогда.

Пару минут они шли молча, потом провожатый осторожно поинтересовался:

--- А ты секреты хранить умеешь?

--- Сколько?~--- деловито поинтересовалась мелкая, весьма искушённая в таких делах.

--- Двадцать бит, если будешь молчать, что я там уснул.

--- Пф! Двадцать бит я б те сама дала за инфу о том, где стоит любимая ваза принцессы Луны. А поскольку ты там уснул, теперь скажешь бесплатно.

--- А тебе зачем?

--- Затем, что любимую вазу принцессы Селестии уже кокнула моя подруга. Сечёшь?

--- Боюсь, это секретная информация.

--- Бойся. Потому что я твой сон на посту секретить не собираюсь.

--- Да ври сколько хочешь!~--- вдруг возвысил голос стражник, распахивая телекинезом дверь. За дверью обнаружились их высочества Селестия и Луна.~--- Чтобы тебе поверили, придётся придумать враку поубедительнее сна на посту или надписи на двери туалета!

Эппл Блум вошла в комнату, стражник закрыл за ней дверь снаружи.

Селестия изогнула бровь:

--- Что ещё за надпись?

--- «Не колдовать и не телепортироваться!»~--- хмуро ответила Луна. Встала со своего места, подошла к двери и наполовину высунулась:~--- Гвардеец! По окончании этой миссии засту́пите на пост у входа в дворцовый парк.

--- Ваше высо!..

--- Два дня, завтра и послезавтра.

--- С-с-с{\ldots}лушаюсь.

Луна вернулась на своё место.

--- Эт’ чё щас было?~--- полюбопытствовала Эппл Блум.

Селестия рассмеялась:

--- В парк постоянно ходят экскурсии, детишки из школ и садиков. Угадай, что́ им всё время оказывается нужно?

--- В сортир, чего ж ещё. Детишки ведь.

--- А теперь угадай, кому приходится их туда-обратно сопровождать.

--- А, понятно.

--- Теперь ещё и надпись дверную им объяснять. Всё же два дня на этом посту слишком сурово.

--- А по-моему, всё честно,~--- отозвалась Луна.~--- Один день за надпись, один за то, что уснул.

--- Вы чё, знаете эту фишку, когда себя отмазывают, заворачивая правду на чужое как бы враньё?!

--- Девочка,~--- вздохнула принцесса,~--- этой, как ты сказала, фишке тыща двести лет. Причём это не преувеличение, а скорее наоборот. Мы с сестрой ей сами пользовались, когда в твоём возрасте были{\ldots} и чаще всего неудачно.

--- Давайте всё же к делу,~--- остановила ностальгию Селестия.~--- Присаживайся. Чай, кофе, печенье?~--- она кивнула на середину стола.

--- Не, спасибо,~--- мотнула головой Эппл Блум. --- У сестры с бабулей всё одно вкусней всех, после их стряпни всякую городскую каку в рот-то брать противно{\ldots}

Луна вдруг скрючилась и закудахтала от смеха. Селестия же подскочила и выпрямилась, как будто ей ткнули шилом в круп.

--- Погодьте, эт’ чё{\ldots}~--- дошло до мелкой.~--- Это получается, шо я вашу{\ldots} как бы{\ldots} ой{\ldots}

--- Не бери в голову,~--- суховато сказала Селестия.~--- Мне с самого начала следовало понимать, чего можно ожидать от сестры Элемента Честности{\ldots} к тому же, дома у вас и впрямь намного вкуснее, ни секунды ни сомневаюсь.

--- Прощения просим{\ldots}

--- Я же сказала, не бери в голову. Собственно, это одна из причин, почему ты здесь.

--- Печеньки?!

--- Честность!~--- принцесса перевела дыхание, явно успокаивая себя.~--- Скажи вот что. Ты в курсе, что мы с сестрой следим за{\ldots} э-э{\ldots} успехами твоими и твоих подруг?

--- Агась. Мы ж когда метки получили, так вам сюда нашу фотку с половиной города отсылали. Тут уж трудно не понять было, что следите.

--- Да. Сначала Твайлайт много писала про вас, ей и нам это было интересно, опять же ваша дружба, потом оно просто вошло в привычку. Потом у твоих подруг появились наставники, причём такие, что многие бы обзавидовались. \emph{Ты} им не завидуешь?

--- Не, с чего бы?! --- удивилась Эппл Блум.~--- Меня ж они всё одно не смогли бы научить ни летать, ни колдовать, хоть из шкуры выпрыгни.

--- Не о том речь,~--- уточнила Луна.~--- Тебе не завидно, что у~них \emph{есть} наставники?

--- А{\ldots} да не. То есть, им-то учиться реально нужно, особенно Скутс. А меня по ферме всё одно никто лучше брата с сестрой не научит. А ежли послушать чего интересное, так меня и так завсегда зовут.

--- Говоря о «послушать интересное»{\ldots}~--- вновь перехватила инициативу Селестия.~--- Помнишь ту внеплановую лекцию полторы луны назад?

--- Когда Скутс и Сильверстрим стыкнулись? Агась.

--- Было интересно?

--- Ещё бы. Правда, Старлайт потом неделю причитала, шо её за это убьют. Ну так не убили же вроде.

--- Нет. Конечно, нет. Сказать по правде, Твайлайт собиралась устроить взбучку, но я с ней категорически не согласилась. На са́мом деле наставники твоих подруг оказали мне огромную помощь, за это только благодарить нужно{\ldots}

--- Какую помощь?

--- Касательно отношений с грифонами. Я прекрасно понимала необходимость каких-то шагов, это давно назрело{\ldots} и всё никак не могла решиться. Сомневалась, ходила вокруг да около{\ldots} а после той лекции была просто вынуждена. Поняла, что иначе уже никак.

--- И чё за шаги?

--- Если тебе интересно, то со следующего учебного года двери эквестрийских военных училищ будут открыты для грифонов. Это для начала.

--- О как. А не боитесь, что вскорости они могут{\ldots} того? Судя по той лекции{\ldots}

--- Того, что скоро туда придётся принимать одних грифонов, ты это хотела сказать? Не боюсь. Первые несколько лет эта проблема перед нами ещё не встанет. А потом, судя по той же лекции, для грифонов нужно будет открывать отдельные факультеты со своими программами обучения, и этим как раз займутся первые их выпускники. В любом случае, это лишь начало задуманного, потом придётся корректировать по обстановке, и нам кажется, что ты сможешь сыграть в этом немаловажную роль{\ldots}

--- В ситуации с грифонами?!

--- Я сейчас говорю в широком смысле. Речь уже не только о~них. Нам представляется, что ты могла бы помочь в ситуации с чейнджлингами.

--- А с ними-то чё не так?

--- С ними всё даже хуже. Представь, что какому-то конкретному пони показали конкретного грифона и сказали, что с завтрашнего дня им придётся работать вместе. Что он сделает? Закатит глаза, попеняет, что вот счастье-то привалило, посетует, что придётся привыкать к этому грубияну{\ldots} Но в итоге они наверняка как-нибудь сработаются и уживутся. Согласна?

--- Ну, пожалуй, да.

--- А если покажут чейнджлинга и скажут то же самое? Добрая половина откажется сразу, больно уж непонятный народ. И~это при том, что грифонов откровенно недолюбливают, а~чейнджлингов на словах считают друзьями.

--- Но у нас в Понивилле{\ldots}

--- Это у вас. У вас там ко всему уже привыкли и вообще Школа Дружбы под боком. Поверь, что в других местах совсем не так.

--- Ну ладно{\ldots} а я-то при чём?

--- Когда твоя подруга Свити Белль обзавелась наставницей, она почти сразу оказалась втянутой в приключение с чейнджлингами, помнишь?

--- Ещё бы. Мы ж её потом чуть не убили, что она участвовала, а мы нет. И чё?

--- Мы просим тебя написать об этом книжку.

--- Чё-ё-ё-ё-ё-ё?! Меня?!

--- А что такое?

--- Почему?!

--- Во-первых, ты хорошо знаешь всех участников тех событий, и тебе не составит труда собрать всю нужную информацию. Во-вторых, у тебя нет никаких предубеждений против чейнджлингов, а книжка нужна такая{\ldots} позитивная. Даже весёлая. И при этом честная. Ты с этим справишься лучше других. В-третьих{\ldots}~--- Селестия улыбнулась,~--- за тем приключением стои́т кое-что несколько большее, чем кажется. Обещаю, тебе будет очень интересно узнать об этой первой. Ну и в-четвёртых, за тебя говорят две рекомендации.

--- Какие ещё рекомендации?!

Селестия посмотрела на сестру. Луна вздохнула, собираясь с мыслями.

--- После того{\ldots} понивилльского эксперимента{\ldots} ты написала о нём письмо своей кузине. И показала его своей подруге Свити Белль, которую просила кое-что у меня уточнить. Так?

--- Угу. Черновик показывала.

--- Свити упомянула об этом в своём письме ко мне. Я заглянула в сны той девочки, хотя их пришлось долго искать среди прочих. Судя по увиденному мной, твоё описание произвело на Бэбс Сид весьма яркое впечатление.

--- Ну и{\ldots} а что второе?

--- Ты показывала черновик не только подруге. И он посоветовал тебе подумать о писательской карьере.

--- Да блин, это ж в шутку!

--- Не скажи. В устах такого рассказчика подобные слова значат многое. Это очень серьёзный комплимент. Вчера я специально переспросила и услышала то же самое.

--- Блин{\ldots} ну как-то это{\ldots}

--- Неожиданно?~--- усмехнулась Селестия.~--- Понимаю. Но писательство~--- это примерно такая же работа, как работа на ферме. И одно другому совершенно не мешает, кстати говоря.

--- А этот, ну, талант?

--- Тебе же только что сказали.

--- Да у меня даже по сочинениям в школе выше четвёрки сроду не бывало.

--- Видишь ли, хорошо рассказанная история имеет весьма мало общего с хорошо написанным школьным сочинением. Чаще даже как раз наоборот.

--- Хм{\ldots} ну, вообще да, дядя Виндчейзер однажды примерно так же сказал. И добавил, что авторам учебников по литературе стоило бы отор{\ldots}

--- Кхм!~--- остановила её Луна.~--- А про учебники истории он ничего не говорил?

--- Не-а, не говорил. Сказал, что такие слова при детях говорить нельзя. Это при нас то есть.

Луна ехидно посмотрела на сестру. Та сделала вид, что не заметила. Продолжила как ни в чём не бывало:

--- Ну так что скажешь?

--- Блин, ну я не знаю. В смысле, я же правда ничего не знаю о том, как эти книжки пишутся!

--- А если тебе помогут?

--- Кто?

--- Тебе что-нибудь говорит такое имя~--- Квик Чаптер?

--- М-м-м{\ldots}~--- Эппл Блум сосредоточилась на кончике носа и честно постаралась вспомнить.~--- Не припоминаю. А чё, должно говорить?

Луна вытащила телекинезом откуда-то из-под стола стопку книжек и разложила их веером на столешнице:

--- Взгляни. Может, что-нибудь видела?

Все книги были в мягких обложках с красочными рисунками. На всех стояло только что названное имя.

--- О, вот эту видела!~--- Эппл Блум показала на книжку под названием «Настоящие, или У страсти на поводу».~--- У Свити{\ldots} в смысле у сестры ейной.

--- Значит, всё-таки говорит.

--- Так это ж вроде бабские романы?

--- Дамские,~--- поправила Селестия.~--- Да, это коронная тема Чаптер, она пишет их в качестве хобби и для заработка. Но при этом она очень хорошая журналистка и соавтор нескольких весьма серьёзных пьес. Что ещё более важно, я ей полностью доверяю{\ldots} и она может кое-что рассказать тебе про связанные с той историей обстоятельства. Если ты возьмёшься, она тебе поможет. Возьмёшься?

--- Ну{\ldots} э{\ldots} я таки не уверена.

--- Но попробовать-то можно? А?

--- Ну, можно{\ldots}

--- Замечательно. Тогда иди, до дома Квик Чаптер тебя проводят. Она в курсе ситуации и позаботится о тебе дальше.

--- Пока{\ldots} ой, то есть, до свидания!

Эппл Блум вышла из комнаты, провожаемая улыбками принцесс.

В коридоре тот же стражник опять сидел напротив двери~--- впрочем, теперь он хотя бы не спал. Коротко поинтересовался:

--- Вести́?

--- Агась.

--- Ну, пошли. Не туда, направо{\ldots}

Свернув по коридору во второй раз, он деловито огляделся и уголком рта негромко сказал:

--- Что ты там хотела знать про любимую вазу принцессы Луны? Это не совсем ваза, но она стои́т{\ldots}

--- Не надо,~--- гордо отмахнулась Эппл Блум.~--- Я уже сделала круче, подруги теперь лопнут от зависти!
\begin{center}* * * * * *\end{center}

--- Ну и?!~--- слегка агрессивно потребовала Эппл Блум. --- А~где косяки-то?

--- Так вроде нету,~--- пожала плечами Свити Белль.

--- «Даже если мы ничего не натворили~--- это ещё не значит, что мы ничего не натворили{\ldots}»~--- задумчиво сказала Скуталу.~--- Классный девиз. А что, ты реально её стряпню назвала ка́кой прямо ей в глаза?

--- Ну да. Тока ж я, блин, знать тогда не знала!

--- Если верить Старлайт,~--- заметила Свити,~--- попробуй ты те печеньки, так ещё и покрепче бы выразилась. Зная выпечку твоей бабули{\ldots} охотно верю.

--- Неважно. Так нету косяков?

Свити опять пожала плечами.

--- Коли так, можно после обеда и дальше почитать. Я пару мест заметила, над которыми подумать, но это недолго.

--- Не, я пас,~--- помотала головой Скуталу.~--- Надо до Кантерлота метнуться, кое-что проверить.

--- Ну, значит, до завтра.


\chapter*{Глава седьмая, об экзамене и~первом задании, выпавшем на долю последней из Меткоискателей}\addcontentsline{toc}{chapter}{Глава седьмая, об экзамене и~первом задании, выпавшем на долю последней из Меткоискателей}

--- Нынче кто читать будет?~--- Эппл Блум шлёпнула на стол очередную порцию бумаги.

--- Я, наверно,~--- пожала плечами Скуталу.

--- Я в этом куске не шибко уверена, насколь читателю зайдёт. Она ж мне натурально экзамен учинила, а экзамены{\ldots} оно сами знаете. Их же взрослые придумали, шоб детей мучить{\ldots}

Скуталу и Свити Белль расхохотались в один голос.

--- Щас-то смешно, ага. А пока вас всерьёз учить не взялись{\ldots} сами-то? Экзамены с контрольными~--- отстой, школа~--- фигня{\ldots}

--- Так отстой,~--- авторитетно заявила Скуталу.~--- И именно потому, что в школе фигня, не воспринимать же её всерьёз{\ldots}

--- Тебе-то с трояком по математике не стрёмно ходить?

--- Да хвост покласть. После настоящего экзамена по навигации никто на него смотреть не станет, а я это хоть сейчас в любом училище сдам. А тебе с трояком по литературе?

--- Та ж фигня. Кому оно интересно, когда у меня «Игра» уже двумя изданиями вышла, да крупных статей газетных штук семь.

--- И нечего на меня так смотреть, --- сообщила Свити. --- У~меня оценки для сестры, сто раз уже говорила. Оно не трудно, а~жить проще. Читать-то будем?

--- Будем, будем{\ldots} особенно если кое-кто мне будет страницы телекинезом перекладывать{\ldots}
\begin{center}* * * * * *\end{center}

Идти от дворца оказалось недалеко, каких-то десять минут. Дорога, правда, пролегала по улочкам с извилинами и неожиданными поворотами, но для Эппл Блум, помнившей каждую тропинку на немаленькой семейной ферме, запомнить этот путь не составило никакого труда. Очень скоро сопровождавший её стражник стучался в дверь небольшого уютного домика.

На стук открыла единорожка бледно-салатовой масти с лохматой рыже-красной гривой и такого же цвета хвостом.

--- Ага,~--- хмыкнула она.~--- Та самая Эппл Блум{\ldots} ну, проходи. Устраивайся. Эй, приятель, а ты-то куда?

--- Как куда?!~--- удивился стражник. --- Велено присматривать, не спуская глаз, а по окончании сопроводить обратно во дворец.

--- Вот отсюда и не спускай. Из моего дома другого выхода нет. И~потом сопровождай себе на здоровье.

--- А присесть?

--- Вот же целый тротуар. Присаживайся.

--- А внутри никак?

--- Никак. Допуском не вышел, парень.

--- А «пологом тишины» закрыться?

--- А что, я обязана его уметь? --- ехидно поинтересовалась хозяйка.

--- Я умею.

--- Допуском не вышел. И вообще, надоел уже{\ldots} --- хозяйка телекинезом захлопнула дверь.

--- Чё меня сопровождать-то{\ldots}~--- буркнула Эппл Блум, успевшая устроиться на стуле.~--- Типа я дорогу во дворец не найду. Второй поворот направо, третий налево, там мимо фонтана{\ldots}

--- Нисколько не сомневаюсь, что найдёшь{\ldots}~--- Квик Чаптер уселась на стуле напротив.~--- Дело-то не в тебе.

--- А в ком?

--- А тебя кто в Кантерлот приглашал? Вот то-то и оно. Пока ты здесь, они{\ldots} --- журналистка ткнула копытом куда-то в потолок, --- за тебя отвечают лично. Пойдёшь ты одна, подерёшься с~кем-нибудь, навесят тебе фонарь, и что? Их высочествам за твой фонарь лично объясняться и извиняться?

--- Кому б ещё навесили{\ldots}~--- гордо фыркнула Эппл Блум.

--- И вообще, у меня на этого долдона есть планы. Не сама же ты потащишь машинку во дворец.

--- Какую ещё машинку?

--- Пишущую. Вон ту{\ldots} --- журналистка мотнула головой в~сторону, где стояла старенькая машинка. --- Хочу тебе отдать.

--- Эт’ зачем ещё?!

--- Как зачем? Меня попросили помочь тебе написать книжку. Я согласилась. Вот, это часть помощи и есть.

--- Да нафига́ она мне?

--- О-о{\ldots} Запомни, девочка: для того, кто всерьёз занимается тем, что выражает мысли словами, возможность как можно быстрее фиксировать эти мысли очень важна. А пишущая машинка пока что является самым быстрым способом, ничего лучше ещё не придумали. Вот к примеру{\ldots} э-э{\ldots} ну-ка, назови мне какого-нибудь писателя или поэта из прошлого?

--- Нарифмундл Пропиитский.

--- Тьфу!!!~--- смачно плюнула Чаптер и ловко испарила плевок на лету.~--- Я тебе серьёзно, а ты этого шута горохового поминаешь, хорошо хоть не к ночи{\ldots} Серьёзно, ну?

--- Ну, этот{\ldots} Гааргл Дерзкий.

Результат превзошёл все ожидания: журналистка уставилась на Эппл Блум очень большими и круглыми глазами.

--- Девочка,~--- медленно и с расстановкой проговорила она,~--- ты хоть поняла, что́ сейчас сказала? Тебе же не то что стихи его знать не положено, тебе не положено знать даже то, что грифон с таким именем вообще жил на свете! Откуда?!

--- Э{\ldots} слышала в Школе,~--- не менее ошарашенно призналась Эппл Блум.

--- Ты что, хочешь мне сказать, что в вашей понивилльской школе проходят грифонского поэта, полвека обкладывавшего лично Селестию такими стихами, с которых она только чудом от злости не лопнула?

--- Не, в Школе Дружбы вообще-то.

--- Ещё интереснее. С такими уроками дружбы не надо никакой вражды{\ldots} впрочем, это не моё дело. Ну хорошо, ты сама назвала это имя. Получай: будь у Гааргла Дерзкого в те времена пишущая машинка, так сейчас Эквестрия прогибалась бы под грифонами, а не наоборот, и я говорю абсолютно серьёзно.

--- Ого{\ldots}

--- Пример получился не совсем в ту тему, но без всяких шуток: вот это{\ldots}~--- Чаптер ткнула копытом в сторону машинки,~--- самое настоящее оружие. И если хочешь по-настоящему жечь глаголом, то придётся им овладеть.

--- Да блин, мне сестра просто все уши прожужжала, шо подачки принимать негоже{\ldots}

--- Во-первых, я уже сказала, что это никакая не подачка. Во-вторых, если это тебя напрягает, притащи в следующий раз баночку молнияблочного варенья, она как раз на здешнем рынке столько и стоит. В-третьих, эта машинка счастливая, у меня когда-то с неё первые книги в тираж пошли. Может, и тебе повезёт с ней.

--- А может, ничего и не получится вообще.

--- Давай посмотрим.

--- А как?

--- Ну, когда мне рассказали про тебя и я согласилась с тобой пообщаться, то придумала несколько тестов. Два ты уже успела пройти.

--- Эт’ как?!

--- Всего несколько минут назад тебе удалось меня удивить. Причём удивить очень сильно. Это, знаешь ли, достижение не из малых.

--- Да я у того Гааргла всего несколько строчек{\ldots}

--- Неважно. В тебе есть больше, чем кажется на первый взгляд, и ты способна преподнести сюрприз. Мне этого достаточно.

--- А чё второе?

--- Второе{\ldots} тут нужно сказать вот что. Для хорошего автора очень важна этакая{\ldots} --- Чаптер покрутила копытом в воздухе, --- скажем так, безбашенность. В нормальном смысле. Готовность к смелым высказываниям без долгих раздумий и прикидок, кто как воспримет, да как бы чего не вышло. Ты сегодня доказала, что и это качество у тебя есть.

--- Как?

--- Обхаяв лично её величеством принцессой Селестией приготовленное печенье прямо ей в глаза.

--- Блин, вы-то откуда{\ldots}

--- Через пару минут после окончания вашего разговора мне прилетело письмо. Что ты согласилась, что сейчас придёшь{\ldots} ну и подробности всякие.

--- Да если б я только знала{\ldots}

--- Чем ты слушаешь? Для меня важно, что ты прямо сказала в присутствии высочайших особ то, что было у тебя на уме. Не пытаясь ничего выяснить, прикинуть и рассчитать.

--- Меня теперь этим печеньем{\ldots}

--- Не бери в голову. Можешь поверить, что до стихов Гааргла тебе всё равно далеко. \emph{Пока} далеко, во всяком случае.

--- Ну ладно, это, значит, было два. А чё ещё?

--- В качестве третьего теста я хотела бы прочесть что-нибудь, написанное тобой не для меня. Мне упоминали, что ты написала для своей родственницы целый репортаж про тот понивилльский эксперимент{\ldots} позволишь ознакомиться?

--- Дак я-то не против, а как? Ежли вы думаете, шо я его наизусть помню, то не помню же{\ldots}

--- Помнишь,~--- улыбнулась Квик Чаптер.~--- Конечно, помнишь. Вот здесь{\ldots}~--- она дотронулась до лба Эппл Блум,~--- хранится всё, что ты когда-нибудь сказала или написала. Не всегда легко вытащить эти слова из памяти, но можно воспользоваться специальным заклинанием. С твоего позволения, конечно.

--- А, ну да, Тва{\ldots} принцесса Твайлайт говорила. Но оно же вроде сложное?

--- Не самое простое, да. Но я журналистка{\ldots} и уж тебе-то значение слов «особый талант» объяснять не нужно?

--- Агась{\ldots}~--- Эппл Блум немного оживилась, почувствовав себя в своей стихии.~--- А вот про это можно спросить?

--- Что именно?

--- Ну, вот писательство и журналистика. У них ведь какие символы? Перо там, чернильница, карандаш, свиток, бумага, книга, та же пишущая машинка. В общем-то, и всё. А у вас метка совсем не такая.

--- А, ты об этом. Видишь ли, перечисленное тобой{\ldots} это всё-таки в первую очередь символы писательства. А для журналиста очень важны ещё два качества~--- въедливость и настырность. Готовность всюду совать свой нос в поисках нужной информации, ни перед чем не отступая ради того, чтобы её раздобыть. По-моему, моя метка неплохо под это подходит, как думаешь?

--- Точно, эт’ самое то.

--- Так ты не против того, чтобы я восстановила из твоей памяти и прочитала то письмо?

--- Не{\ldots} а чё делать надо?

--- Тебе~--- почти ничего. Минутку подумать, как ты писала письмо: в какое время суток, в какой обстановке, с каким настроением, всё такое. Чтобы заклинание зацепилось. И потом просто находиться в этой комнате в течение следующего часа. Сейчас я~активирую, и можешь делать что угодно. Пытаться вспоминать содержание не нужно{\ldots} помешать не помешает, но и не поможет ничем.

Рог журналистки засветился. Она телекинезом заправила лист бумаги в машинку (не в ту, что собиралась отдать, а в другую, поновее), затем от кончика рога отделилось ярко-зелёное облачко, поплыло к голове Эппл Блум и окутало её, быстро рассеявшись. Через несколько секунд машинка неторопливо застучала.

--- Вот и всё. Самое большее через час весь текст будет восстановлен.

--- Вы говорили, это третий тест,~--- напомнила Эппл Блум.

--- Да. В течение этого часа как раз займёшься четвёртым, а заодно машинку опробуешь. Хочу, чтобы ты кое-что написала специально для меня{\ldots} в качестве проверки ещё одного нужного навыка.

--- За час-то я много не понапишу{\ldots}

--- Много и не надо, надо оригинально.

--- И про что?

--- Сначала кое-что тебе расскажу. Есть ещё одно умение, которым должен обладать хороший автор{\ldots} точнее, это две стороны одного явления. Уметь \emph{увидеть} необычное в обыденном и \emph{показать} обыденное с необычной стороны. Понимаешь, о чём я?

--- Не совсем. Ну, типа, вот стол. Самый обычный, такой в~каждом доме есть, ну, или примерно такой. Как ты его вообще покажешь необычным? Разве поставить вверх ножками и на каждую ножку чего-нить надеть{\ldots} ну так это не то, наверно?

--- Не то. Ты взяла самый сложный пример, хоть и правильно угадала суть: чем оно обыденнее, тем сложнее раскрыть в нём необычное для себя или других. Хм{\ldots} Ну, давай рассмотрим классику. Допустим, ты читаешь газету и видишь там большими буквами: «Принцесса Селестия, принявшая участие в соревнованиях по спринту, заняла второе место. Фаворит состязаний Рэйнбоу Дэш пришла к финишу предпоследней». Что ты скажешь?

--- Да быть такого не может. Брехня какая-то.

--- Исключено. Это авторитетная спортивная газета с официальными результатами. Вранья там не может быть, потому что не может быть никогда.

--- Ну{\ldots} э{\ldots} если подстава на самих соревнованиях тоже исключена?..

--- Исключена. Всё было по-честному. Думай.

Эппл Блум задумчиво посмотрела на кончик носа. Положила на столешницу оба копытца, несколько раз по-всякому их переставила между собой. Неуверенно сказала:

--- Они там что{\ldots} только вдвоём и участвовали?

--- Именно! Оцени, как мастерски описан совершенно очевидный и тривиальный результат. Такое кто угодно захочет прочитать, даже и не любитель спорта. Говорю же~--- классика.

--- И чё, мне надо за час что-то такое придумать?!

--- Конечно, нет. У тебя будет более конкретное, а потому и~более простое задание. Возьмём какое-нибудь всем известное событие, которое ты знаешь лучше многих других{\ldots}

--- А такие есть?!

--- Есть. Странно, что это спрашиваешь \emph{ты}, живущая в Понивилле. Ты вообще в курсе, что ваш город считается местом, где конец света происходит чуть ли не по расписанию?

--- Агась.

--- С чего это началось?

--- Да с того, шо к нам принце{\ldots} то есть, тогда ещё не принцесса Твайлайт переехала.

--- А потом было возвращение Найтмэр Мун. Насколько понимаю, ты была тогда несколькими годами младше, но вполне уже сознательного возраста жеребёнком. Должна была видеть и~запомнить. Так?

--- Угу.

--- Ну и вот. Тогда-ещё-не-принцесса Твайлайт Спаркл написала письмо своей наставнице принцессе Селестии о древнем пророчестве и связанной с ним угрозе. Против всяких ожиданий была послана в Понивилль учиться дружбе, но внезапно это оказалось именно тем, что нужно. Когда всё закончилось, опять-таки против многих ожиданий осталась в Понивилле. По-моему, общеизвестная история, нет?

Эппл Блум молча кивнула, вопрос был риторическим.

--- Твоя задача заключается в том, чтобы за этими общеизвестными событиями постараться увидеть какой-то скрытый{\ldots} нет, не то слово{\ldots} тщательно скрываемый тайный смысл. Который выставляет всю историю в совершенно другом свете. Таком, что понадобилось это скрывать. Без особых подробностей, но общая схема должна быть понятной и не противоречить ни известным фактам, ни самой себе.

--- Подождите, а разве{\ldots}

--- Нет. Насколько я знаю, нет. Возможно, официальная версия не называет какие-то тонкости и дополнительные обстоятельства, но за все прошедшие годы никому не удалось найти ничего такого, что прямо бы ей противоречило. Это просто игра с~заданием, и смотри на это именно как на игру. Твоя задача упрощается тем, что ты можешь в известных пределах придумывать дополнительные обстоятельства~--- их проверка уже выходит за рамки игры. Лишь бы всё звучало хоть сколько-нибудь правдоподобно.

--- А зачем вообще в такие игры играть, так ведь недолго и до шизы доиграться?

--- Ну, знаешь, бывает всякое. Произошло странное, а~причины неизвестны. Если это преступление, то им, конечно, будет заниматься пониция, но если нет, то это может привлечь внимание журналистов. В любом случае, расследование строится именно так: придумываются версии, которые могли бы объяснить случившееся, потом они проверяются с учётом открывающихся обстоятельств. Что-то отвергается, что-то корректируется, могут появиться и новые версии. Если в конце концов остаётся только один вариант, который как-то может всё увязать между собой, то с большой вероятностью именно он всё и объясняет. Есть даже специальный термин~--- «журналистское расследование». Ну всё, давай, за дело. Смотри, бумага в~машинку заправляется вот так{\ldots}

--- А как писать-то? Если как в школе требуют, правильными словами, так я ж за час вообще ничего{\ldots}

--- Как тебе удобнее. Это же просто упражнение, здесь содержание важнее формы.

В комнате воцарилась тишина. Эппл Блум задумчиво смотрела на белый лист бумаги, который ей предстояло заполнить какой-то историей, а Квик Чаптер перебралась к своей машинке и читала появляющийся там текст по мере его возникновения.

Через несколько минут послышались странные булькающие звуки.

--- Э{\ldots} с вами всё в порядке?!~--- встревожилась Эппл Блум, но оглянулась и поняла, что это всего лишь сдерживаемый смех.

--- Почти,~--- хихикнула журналистка.~--- «Секс-та́натос»{\ldots} мда{\ldots} Знаешь, меня там хоть и не было, но я тебе даже так могу сказать, что эта штука вообще-то называется «секстант»! Не то чтобы я сама умела им пользоваться, конечно{\ldots}

--- Ага. Скутс говорила. Просто я никак не могла запомнить, есть там на конце «т» или нет. Тогда она сказала, шо это можно запомнить вот так, и я запомнила, но потом уже само слово забыла, а переспрашивать постремалась.

--- Забавно. Наверняка это какая-то старая пегасья хохма, а~ещё тебе придётся привыкать пользоваться словарями и вообще всякими книжками. Не отвлекайся.

Ещё через несколько минут в комнате застучали уже две машинки. Эппл Блум довольно быстро приспособилась к клавишам и оценила, насколько права была журналистка~--- так и впрямь оказалось намного удобнее. Постепенно она увлеклась и с некоторым удивлением обнаружила, что лист бумаги уже подходит к концу.

За спиной раздалось деликатное «кхм-кхм».

--- Ну, как там у тебя?

--- Щас{\ldots}~--- Эппл Блум добила последнюю фразу.~--- Вроде всё. И что письмо?

--- Здорово, особенно про «параболу ветвями вниз». Хм{\ldots} То-то дворцовый садовник вечером рассказывал жене, как её величество изволили весь день дурью маяться: «воду из фонтана в~фонтан по во-от такенной дуге перекидывала, всех рыбок распугала{\ldots}».

--- Чё, серьёзно, что ль?!

--- Расслабься. Конечно, нет. Просто прикидываю, как можно было бы растащить эту историю на анекдоты. Мне понравилось, правда. Считай, тот тест ты тоже прошла, осталось посмотреть, что сейчас написала.

--- А как бумагу вытащить?

--- Так же, как и заряжать: крути вот эту ручку дальше. Ну-ка, что там у тебя{\ldots} «Жила-была единорожка Твайлайт Спаркл, она была такая{\ldots}»

Квик Чаптер вдруг скрючилась в какой-то клубок и буквально возрыдала. Эппл Блум открыла было рот спросить, всё ли в~порядке, но быстро распознала смеховую истерику и~приготовилась ждать.

К чести её, журналистка прохохоталась довольно быстро. Распрямилась, встала обратно на ноги, встряхнула головой. Дочитала до конца, несколько раз ещё всхрюкнув.

--- Мда-а-а{\ldots}~--- задумчиво протянула она.

--- Чё, совсем фигово?! --- заволновалась Эппл Блум.

--- Вовсе нет, с чего ты взяла?

--- Ну{\ldots} вы так ржали{\ldots} тётя Рэрити сказала бы, что неприлично даже.

--- Не обращай внимания. Я ржала как читатель, этому радоваться нужно.

--- Пойдёт хоть?

--- Пойдёт ли? Девочка, даже если бы ты написала только первые две фразы, уже этого было бы более чем достаточно!

--- То есть я{\ldots}

--- Да. Думаю, мы сработаемся, и всё у тебя получится. При некотором старании должно получиться.

--- А мне во дворце говорили, вы что-то такое про это дело знаете{\ldots}

--- Знаю, но с этим пока придётся потерпеть. Видишь ли{\ldots} раз уж писать будешь ты, я не хочу давить на тебя своим ви́дением ситуации. Сначала у тебя должно сложиться \emph{своё} представление и \emph{свой} план того, как рассказать всё читателю. Когда моя информация уже не сможет серьёзно повлиять на твой взгляд, а лишь добавит к нему детали~--- вот тогда и расскажу. Понимаешь?

--- Кажется, да.

--- Ну вот мы и начали работу над книгой. Смотри. Если не вдаваться пока в детали, то ты хочешь рассказать читателям о~событии, в котором приняла участие твоя подруга, и в котором ты сама не отказалась бы поучаствовать. С чего имеет смысл начинать рассказ?

--- Так ведь Свити там не одна была. Наверно, надо сказать о~тех, кто вообще участвовал.

--- Допустим. И кто?

--- Кроме Свити~--- Старлайт Глиммер и Трикси. Ещё этот чейнджлинг, Фаринкс, но он к ним потом присоединился{\ldots} наверно, его с самого начала упоминать не надо? Чтобы эта была{\ldots} как её{\ldots} интрига.

--- Так. Старлайт Глиммер и Трикси хорошо известны, их представлять не нужно. В конце концов, это ведь именно они помогли преобразиться чейнджлингам, и их за это даже наградили орденами. А каким боком тут твоя подруга?

--- Ну так ведь она как раз за день до этого стала ученицей Старлайт{\ldots} о! То есть, видимо, надо начинать с того, как она в~ученицы и попала.

--- Ты сама-то это знаешь?

--- В общих чертах. Свити так не терпелось рассказать про чейнджлингов и жук-пука, шо остальное она совсем мельком. Вроде как, Рэрити шо-то там неосторожно ляпнула, Старлайт подхватила, а Твайлайт припечатала, и стало так.

--- То есть тебе для начала нужно толком восстановить эту историю ученичества. У кого будешь выяснять?

--- Да у этих троих и буду. Ну, ещё у Свити, само собой. У кого же ещё.

--- Я бы на твоём месте ещё и Трикси расспросила. Лишним точно не будет.

--- Хм, ну да. Пожалуй.

--- Вот тебе и домашнее задание. Расспроси их всех и разберись, как всё было. Машинка у тебя теперь есть, и лучше сразу же печатай всю добытую информацию. В следующий раз мы встретимся{\ldots} давай через четыре дня. Я буду ждать тебя дома, приходи в любое время, но лучше утром. Школу ради этих встреч можешь пропускать, с таким приглашением тебе никто слова не скажет{\ldots} но по возможности всё-таки не злоупотребляй. Будет хорошо, если ты успеешь написать связный рассказ по тому, что узнаешь, но это не обязательно. Главное, чтобы была информация, а рассказ всё равно потом придётся править и переписывать. Всё понятно?

--- Агась. А как я отсюда и сюда?

--- Отсюда~--- через дворец. Во дворце всегда есть дежурный маг, владеющий телепортом, а за́мок принцессы Дружбы уж точно входит в список обязательных контрольных точек. А сюда{\ldots} ну, ты же всё равно сестре расскажешь о своём поручении. Она договорится с Твайлайт или со Старлайт, чтобы тебя сюда доставляли. В крайнем случае поездом, если никого из них на месте не окажется.

--- И чё, тут за мной теперь всегда дворцовый долдон по пятам ходить будет?!

--- По дворцу~--- будет. А по городу{\ldots} это уж от тебя зависит,~--- усмехнулась журналистка.~--- Если убедишь сестру упросить принцессу Твайлайт написать принцессе Селестии письмо и~поручиться за тебя{\ldots} Кстати о долдонах, давай-ка нагрузим нашего, и пойдёшь с ним потихоньку. Тебе уже обедать пора{\ldots} я, конечно, могу тебя покормить, но если уж ты стряпню её величества обхаяла{\ldots}

Эппл Блум вздохнула и повозила копытцем по полу. Квик Чаптер подхватила телекинезом старую машинку и понесла к~двери. Взглядом попросила открыть её.

Вместе с дверью открылась дивная картина: стражник действительно сидел на тротуаре и опять спал, обняв копьё.

--- Тс-с-с!~--- прошептала журналистка своей новоявленной ученице, аккуратно опуская машинку. Вытащила из дома фотоаппарат и установила на крыльце.

Эппл Блум, глядя во все глаза, была вынуждена признаться само́й себе, что наставнице удалось-таки ответно её удивить. Зелёная телекинетическая аура окутала копьё, аккуратно высвобождая его из ослабевших объятий, а затем три события слились в одно.

Остриё копья ткнулось в круп стражника.

Стражник заорал и взвился в воздух как заправский пегас, даром что был единорогом.

Затвор фотоаппарата щёлкнул.

--- Утречка,~--- как ни в чём не бывало поприветствовала Квик Чаптер, напрочь игнорируя свои же слова минутной давности про «обедать пора».~--- Принимай объект под охрану, сопровождай во дворец. И вот тебе ещё в нагрузку{\ldots}~--- она кивнула на машинку.

--- Какую, к дискорду, нагрузку?!~--- злобно прошипел страж, пытаясь изогнуться и осмотреть уколотое место.

--- От слова «груз». Донесёшь ей.

--- Чего это я?!

--- А кто? Я?

--- А я обязан?

--- Понятия не имею. Про свои обязанности будешь разговаривать с командирами{\ldots} после того, как я снимочек опубликую,~--- журналистка приподняла фотоаппарат в воздух и~покрутила перед носом.

--- Он ещё мне пытался рассказывать, где стоит любимая ваза принцессы Луны! Шоб я её кокнула на зависть подругам,~--- радостно наябедничала Эппл Блум.~--- И двадцать бит предлагал!

--- Правда, что ли?~--- воздела бровь Чаптер.

--- Агась! Пытался и предлагал. Правда, сначала я сама спрашивала, и двадцатку он предлагал в другой связи{\ldots} но про это упоминать не обязательно.

--- Способная ученица! --- умильно проворковала наставница. --- Умная девочка, прямо на лету всё схватывает.

Стражник, бормоча себе под нос что-то явно ругательное, кое-как взгромоздил машинку себе на спину и похромал в сторону дворца. Эппл Блум двинулась за ним.

--- Так помни! Через четыре дня я тебя жду с информацией{\ldots}

{\ldots}Проводив ученицу взглядом, Квик Чаптер вернулась в~дом. Села за стол, положила перед собой лист с напечатанным заданием и перечитала его ещё раз.
\begin{quotation}
Жила-была единорожка Твайлайт Спаркл. Она была такая офигеть умная, что у неё все мозги ушли в рог, и на нормальную жизнь уже не оставалось. Поэтому в школе от неё шарахались, но как-то терпели. Потому как всё же умная была. Ну, хотя бы по части книжек ихних.

Однажды эта самая Твайлайт Спаркл раскопала в древней книжке эту самую муть про возвращение Найтмэр Мун и накатала письмо принцессе. Но в~школе решили, что она совсем с кукушкой раздружилась, и положили под сукно. А принцессе ещё и~настучали: вот, мол, умная-умная, а дура.

А у принцессы как раз тогда голова болела, что через неё какую-то блатную в школу пропихнуть пытались. Поэтому она решила две яблони одним пинком околотить и велела эту Твайлайт Спаркл из школы куда-нибудь сплавить. Типа, раз она такая умная, значит, уже всё превзошла и больше ей тут делать нечего. А на её место ту блатную взять. Так оно всё и~сделали.

А потом Найтмэр Мун взяла и вернулась. А эта самая Твайлайт Спаркл взяла и с ней справилась. И~тут уж все скумекали, что как-то стрёмно это всё получилось, и это ещё очень даже большой вопрос, кто тут самая большая дура. И принцесса скумекала первее всех.

Поэтому в газетах прописали, что, мол, и было всё  совсем-совсем не так, и эта самая Твайлайт Спаркл вроде как с самого начала была у принцессы наилюбимейшей ученицей, и вся эта фигня была совсем не с того, что на её место блатную пропихивали, а такой хитрый план. И все, конечно, сразу поверили. Потому как вслух держать принцессу за самую большую дуру, это же надо самому с головой совсем не дружить.

Правда, эту самую Твайлайт Спаркл обратно в~школу запихнуть уже не получалось, потому как место занято было. Поэтому её оставили жить там, куда из школы выпихнули, и опять сделали вид, что это тоже такой хитрый план. Но это уже другая история.
\end{quotation}
Журналистка хмыкнула, подвесила этот лист в воздухе телекинезом, зарядила в свою машинку чистый и быстро застучала по клавишам{\ldots}
\begin{center}* * * * * *\end{center}

--- И что она напечатала, когда тебя проводила? --- поинтересовалась Свити Белль. --- Ты это видеть не могла.

--- Почему не могла? Она ж сама мне показала потом. Ну, то есть не факт, шо прям сразу уселась, но оно и не шибко важно. Рассказик. Небольшой, на пару страниц, аккурат по той идее, шо с меня наэкзаменовала. Называется «Распределение».\footnote{Рассказ «Распределение» см. на стр. \pageref{raspredelenie}.}

--- Покажешь?

--- А разве не показывала? Хорошо, поищу при случае{\ldots}


\chapter*{Глава восьмая, в которой Меткоискатели обсуждают хитрости процесса обретения наставников}\addcontentsline{toc}{chapter}{Глава восьмая, в которой Меткоискатели обсуждают хитрости процесса обретения наставников}


--- Сёдни лучше, шоб ты, Скутс, читала,~--- распорядилась Эппл Блум.~--- А то там про Свити дофига.

--- Лучше, так лучше{\ldots}

--- Свити, а ты учти. Шо я там всё прописала точно как вы с~Рэрити мне рассказывали, а не как в «Игре» прописано было. Ты~ж слушай, и если чё надо скрыть, то говори.

--- Да нечего там уже скрывать, почти три года прошло. А~так, как ты расспрашивала, и без согласия публиковать можно, с высочайшего-то поручения. Если уж тебе сестра всё выдала по первому слову{\ldots}

--- Ну, всё равно. Скутс, читай.
\begin{center}* * * * * *\end{center}


--- Ну чё тут скажешь{\ldots} круто, конешно, --- констатировала Эпплджек, выслушав рассказ младшей сестры и разглядев бумажку. Разумеется, всё это случилось не раньше, чем они пообедали.

Бумажку выдали во дворце перед отправкой домой, и~Эппл Блум ей сразу же страшно возгордилась. Там было красивыми завитушками написано, что «все вопросы, задаваемые предъявительницей сего Эппл Блум, задаются по высочайшему поручению и во благо Эквестрии», внизу стояли внушительные подписи с печатями, а если задержать взгляд на словах «предъявительницей сего Эппл Блум», то над бумагой будто всплывало изображение метки~--- щит с яблоком.

--- Буду всех расспрашивать,~--- гордо сообщила младшая.~--- Ты первая.

--- И чё те с меня надо знать?

--- Как Свити в ученицы к Старлайт попала. Давай, делись, чем знаешь.

--- Эхма{\ldots} --- вздохнула старшая. --- Как бы те сказать-то{\ldots} Я~ж о том деле знаю всего ничего, да и то рассказывать неудобно.

--- Элемент Честности,~--- напомнила Эппл Блум.

--- Дак я ж не отпираюсь. В общем, как оно, это ученичество, началось, меня потом три дня, если даж не больше, все тюкали за язык мой. Рэрити, конечно, шибче всех, но и остальные не шибко отставали. Дескать, какой пример подаю жеребёнке, вон Старлайт её с радостью взялась делу учить, а я?..

--- Не поняла. А что ты?

--- Да вишь какая история{\ldots} Аккурат когда подружка твоя в~ученицы просилась, Твайлайт при ней ругалась то ли едрёным сеном, то ли как-то ещё, просто дым из ушей. В моём духе, в общем, ну ты ж меня знаешь. А Свити-то эти словечки в диковинку, ну она и заинтересовалась~--- шо они означают, да всё такое. Там Старлайт была, так она на меня сразу все стрелки и загнула{\ldots} дескать, Эпплджек так часто выражается, она должна знать, у ей и~спроси. Удружила, конешно, хотя с другой стороны, чего ей ещё было делать-то? Словечки-то и впрямь с моего языка. Ну и~вот. Молодец, типа, шо так ловко выкрутилась, на́ тебе, учи девчонку. Назавтра они в тот свой поход отправились, а~как вернулись, так и началось. Рэрити тюкала, Старлайт тюкала, Твайлайт тюкала{\ldots} хотя по-хорошему её саму надо было, это ж она язык распускала. Даже принцесса Луна мораль читала! Такие дела. Я,~конеш, свой язык-то с тех пор при детях стараюсь ещё больше придерживать, а тока всё одно стыдно. Уж не знаю, поможет это тебе чем или нет.

Эппл Блум озадаченно хмыкнула. Отодвинула от себя тарелку, придвинула новенький блокнот с карандашом. Немного погрызла карандаш и записала:

\emph{«Твайлайт матюгалась при Свити едрёным сеном и ещё всяко что дым из ушей, Старлайт взялась её научить как следует, загнула стрелки на Эпплджек, правильно сделала»}.

Потом чуть подумала и пририсовала сбоку большой вопросительный знак.
\begin{center}$\triangleleft\star\triangleright$\end{center}

--- Эм{\ldots}~--- Свити Белль поковыряла землю копытцем.~--- А тебе как рассказать? Как мы обычно рассказываем, типа наша семейная история, или как на са́мом деле было?

--- А есть разница?~--- насторожилась Эппл Блум.

--- И ещё какая.

--- Ну, давай тады сначала семейную историю.

--- Ты при начале-то сама присутствовала. Помнишь тот неудавшийся пикник, когда на нас эта насекомая дрянь напала и мы потом в пещере байки слушали?

--- Ну да.

--- Я тогда перепугалась и с перепугу сферу создала. Это вообще-то довольно крутой магией считается, сестра не умеет. Она, сфера то есть, у меня, конечно, и пары секунд не продержалась, но сестра заметила и отвела меня к Твайлайт. У той меня учить времени не было, зато Старлайт предложила себя, ну и~вот.

--- И всё, что ли?~--- разочарованно спросила Эппл Блум, царапая в блокноте: \emph{«Свити как бы: пикник, насекомые, сфера, повели учиться, Твайлайт была занята, вместо неё Старлайт»}.~--- А едрёно сено?

--- Так Твайлайт им ругалась. Как раз про то, что времени не было. Ругнулась и куда-то скакнула телепортом. Я и спросила, чего это такое. А меня все сначала наперебой к твоей сестре посылать стали, а потом намекать, что про это вообще лучше не спрашивать.

Эппл Блум подчеркнула слова «была занята» и провела от них стрелочку вверх к словам «едрёно сено».

--- Так, а если по правде?

--- Если по правде{\ldots} --- Свити Белль снова ковырнула землю,~--- то меня к Твайлайт отвели не из-за сферы. То есть из-за сферы, это смотря как посмотреть, но тут всё довольно сложно{\ldots}

--- А всё-таки?

--- Эм{\ldots} ну, сферу-то я тогда поставила, да, но сестра на неё внимания не обратила. Мне это потом ночью ещё приснилось, и вроде я во сне даже поняла, как её нужно удерживать. Сферу то есть. Наутро, как проснулась, попробовала у себя в комнате, но чего-то пошло́ не так. В общем, я в комнате устроила погром, и прибежала сестра, а я испугалась и свалила всё на кошку. Типа, это она на меня книжку свалила, когда я спала, а со мной от неожиданности магическая икота приключилась. Ну, оно и~правда такое бывает, я где-то читала. Это когда икаешь и~при этом из рога всякой дурной магией шмаляет{\ldots}

--- Агась{\ldots}~--- Эппл Блум настрочила в блокноте: \emph{«По правде: расфигачила сферой комнату, закосила под волшебную икоту»}.~--- Дальше?

--- Дальше я позавтракала и опять сферу создать попробовала. Только на этот раз из рога по-настоящему магией шмальнуло, и прямо в платье, которое сестра три дня шила на заказ. А~платье~--- фшух!~--- и все фитюльки на нём в кактусы превратились. И тут как раз сестра в дверях стоит. И я такая стою и понимаю, что сейчас меня убьют{\ldots}

--- Так не убили же?~--- ценная информация в блокноте дополнилась строчкой \emph{«платье тоже расфигачила кактусами{\ldots}»}.

--- Нет. Я, конечно, опять про икоту, а сестра посмотрела, и~вдруг давай меня хвалить. Вроде, с кактусами даже лучше стало, эксклюзив получился, вот. Ну, и дальше как я говорила: повели к Твайлайт учиться.

После «кактусами» появилась приписка \emph{«{\ldots}к лучшему»}.

--- Мда. Всё?

--- Не совсем. Было ещё странное{\ldots} вроде, сестра сначала не хотела меня к Старлайт в ученицы отдавать, то ли она недостаточно крутая, то ли ещё что. Но тут как раз Твайлайт вернулась, тоже телепортом, и ка-ак список её заслуг назвала. Я прямо обзавидовалась, нам в жизни столько не наворотить.

В блокноте появился жирный вопросительный знак и рядом фраза \emph{«Старлайт недостаточно крутая? Рэрити думала»}.

--- И чё за заслуги?

--- Ой, там много. Кого-то там поработила, что-то отобрала{\ldots} зомбировала, телепортировала, заразила{\ldots} Я потом по памяти постаралась записать, могу для тебя найти.

--- Агась. Я к тебе всё равно ещё буду приставать, как оно всё дальше было{\ldots}~--- и Эппл Блум завершила этот протокол расспросов словами \emph{«Список стрясти со Свити!»}.
\begin{center}* * * * * *\end{center}


--- Так, минутку!~--- остановилась Скуталу.~--- Что-то мне смутно помнится{\ldots} вроде, где-то тут у нас экземпляр «Большой Игры» валялся?

Эппл Блум махнула копытом в угол. Свити пролевитировала оттуда книжку на стол.

--- Было ближе к началу{\ldots} ага, вот!~--- Скуталу перелистала страницы и прочла:~--- «Заразила своим гневом Селестию и Луну, телепортировала их в деревню, и они там всех поработили, а~потом устроили уличную драку на пять реальностей, пока ты меняла время и мыла мозги подругам Твайлайт за то, что они поменялись отобранными метками и потеряли стол с банкой, в~которую их сложили? И кого-то там зомбировала{\ldots}». Ты что, \emph{вот этот} список и представила?!

Свити Белль и Эппл Блум расхохотались.

--- Нет, конечно! Там же настоящий список страницей раньше идёт, а это хохма. Если ты не заметила, я в этой книжке вообще года на три младше описана: мне тогда тринадцать было, а по тексту где-то десять, самое большее одиннадцать.

--- Хм{\ldots} да, пожалуй. А смысл?

--- Мне так посоветовали,~--- пожала плечами Эппл Блум.~--- Сказали, шо тогда оно на более широкую аудиторию пойдёт. Почему бы и нет? Ты читай давай, там самое интересное впереди.
\begin{center}
	* * * * * *
\end{center}

--- Великой и могущественной Трикси нечего скрывать!~--- гордо провозгласила оная. И деловито поинтересовалась:~--- Ты чего знать-то хочешь?

--- Правду, как Свити в ученицы попала.

--- Ха! Влипла Стар, вот тебе и вся правда. Потом, правда, привыкла. Ну, то есть я-то там не была, потому как аккурат за день до того наша Прынцесска Дружбы что-то про чейнджлингов задвигала. Мол, какие они бедные-несчастные, литературы у~них нету, с искусством у них туго, потому как тоже нету, надо типа им культурно помогать, надо их всяко дружить, вот это вот всё такое. Я у ей спрашиваю: что, и фокусы они не знают? А она такая: ага. И я такая сама себе: ага! Раз не знают, так надо ж первой быть! Сразу пошла гастроль готовить и название придумывать. Классно так придумалось, вот, оцени: «Эксклюзивное Представление Великой и Могущественной Трикси для Возродившихся к Новой Жизни Чейнджлингов, Которых Она Избавила от Тирании Королевы Кризалис». Скромненько, коротко, но со вкусом{\ldots}

--- Момент!~--- попросила Эппл Блум, записывая в блокнот.~--- Ага, дальше?

--- Ну, дальше я всё собрала, фургон упаковала и пошла за великой и могущественной ассистенткой. Прихожу, а она перед шибко заумной книжкой сидит и по столу обтекает. Спрашиваю~--- говорит, всё, ученицу ко мне прицепили мелкую. Я такая, не будь дура: а поехали к чейнджлингам на гастроль! Хоть в себя придёшь немного за пару дней. Она, конечно, обрадовалась, а~потом пошла к этой вашей моднице про мелкую разговаривать{\ldots} ну и как дура. Задом наперёд, совсем наоборот: вместо чтоб мелкой с нами ехать запретили, так её ещё и~благословили в дорогу! Ну, Стар мне подруга всё-таки{\ldots} я ещё покумекала и~подкинула ей идейку: ты с мелкой-то позанимайся маленько, а потом сплавь её тихонечко в кантерлотскую Школу! Не может же быть, чтоб её с такими связями не приняли! И~главное, идея-то была хорошая, но я ж потом на тех гастролях сама напортачила как дура. В итоге вместо гастролей вышло прикольное приключение, а после того уж мелкая ни о какой Школе и слышать не хотела~--- нафига ж ей та Школа сдалась, если здесь прямо с первых дней такая веселуха? Ну вот. Мы сначала в шоке были, потом попривыкли. Как-то так. Про гастроль тоже рассказывать?

--- Пасиб, тётя Трикси! Про гастроль я потом ещё спрошу{\ldots} а~вы про едрёно сено чего-нить знаете?

--- Чего-о? --- Трикси удивлённо вытаращилась на Эппл Блум, потом честно задумалась и решительно помотала головой: --- Не. Про то ничего сказать не могу. Если и было, то не при мне.

--- Агась. Так я вас про гастроль потом ещё найду!

В блокноте появилась ещё одна запись:

\emph{«Твайлайт сказала Трикси что чейнджлинги в фокусах не шарят, та решила замутить и взяла с собой Старлайт чтобы отдохнуть от Свити, а та лоханулась и Рэрити её отпустила. Потом лоханулась Трикси, вышло прикольно, и Свити про Шк.Од.Ед. уже слышать не хотела, так в ученицах и осталась»}.

Нарисованная стрелочка вела от слова «замутить» к записанному чуть выше названию несостоявшегося шоу.
\begin{center}$\triangleleft\star\triangleright$\end{center}

--- Хм{\ldots} Квик Чаптер, говоришь?~--- Твайлайт задумалась.~--- Сказать по правде, не слышала. Ну, книжки её тебе показывали, таких у меня в за́мковой библиотеке нету. С периодикой тоже вряд ли что-то смогу выяснить, здесь только научная. А вот что касается пьес{\ldots}

Она подошла к каталожному шкафу и вытянула один ящик. С~негромким «шурх-шурх» быстро перебрала в нём карточки.

--- Ага! И правда, есть!~--- Телекинетическая аура выудила два картонных прямоугольничка и положила их на стол перед Эппл Блум.~--- Вот, две пьесы, обе в соавторстве. Названия мне ничего не говорят, но издано в известной серии. Возьмёшь почитать?

--- Не, спасибо,~--- мотнула головой та, переписывая в блокнот информацию.~--- Просто интересно было проверить.

--- Не думаешь же ты,~--- хмыкнула Твайлайт,~--- что тебе во дворце могли дать неверные сведения?

--- Не-а. Просто любопытно. Ну и потренироваться. Мне там вот ещё чё дали,~--- Эппл Блум вытащила из сумки и положила рядом с карточками свой мандат.

--- Ого! Впечатляет. Будешь спрашивать?

--- Буду. Как Свити попала в ученицы к Старлайт?

--- Внезапно. Так получилось, что в этой истории мне кое-что пришлось пропустить{\ldots} пойдём-ка в тронный зал, дело было там.

В тронном зале Твайлайт подошла к столу с магической картой:

--- Вот, я в то утро как раз здесь стояла и о чём-то думала. Кажется, что-то с чейнджлингами связанное, но уже не вспомню. Пришла Рэрити с сестрой, у меня сразу из головы выскочило. Поговорили о всякой ерунде{\ldots} немного посплетничали. Тут входит Спайк с письмом из Кантерлота. Я открываю, читаю~--- батюшки, неотложное королевское дело, ещё вчера надо было сделать, всё такое. Я все разговоры со сплетнями сразу из головы долой и прямо телепортом туда{\ldots}

--- Едрёно сено.

--- Ну да, выругалась, было дело. Неудобно получилось, конечно, при Свити-то{\ldots} да и Рэрити такие выражения шибко не одобряет{\ldots} Да ещё и на пустом месте, как оказалось: Луна, когда письмо писала, ошиблась и не ту дату поставила. У неё бывает такое~--- то она забудет, что полночь прошла и число сменилось, то, наоборот, перестрахуется и два раза это учтёт{\ldots} В общем, когда всё выяснилось, я прыгнула обратно сюда. А~тут в моё отсутствие Старлайт и Рэрити уже успели рогами сцепиться~--- одна рвётся Свити Белль учить, другая кричит, что не с её-то прошлым. Я, конечно, объяснила, что подруги себя так не ведут, а~идея хорошая. Рэрити ещё немного поупиралась, но согласилась. Вот и всё.

--- Всё?

--- Ну, я же сказала, что часть истории пропустила. Потом комнату им для занятий выделила, посоветовала Старлайт книжку про обучение, это уже мелочи.

--- Хм{\ldots}

--- А если ты узнаешь, чего там они поругались в моё отсутствие, так мне само́й интересно будет.

Протокол этой беседы гласил: \emph{«Твайлайт часть истории пропустила. К ней пришли Рэрити и Свити, сплетничали, пришёл Спайк с письмом, телепортнулась в Кантерлот с едрёным сеном. Когда вернулась, Рэрити полаялась со Старлайт. Успокоила и разрулила».}

\begin{center}$\triangleleft\star\triangleright$\end{center}

--- Да ничего особенного,~--- пожала плечами Старлайт.~--- Ну, неожиданно, конечно, получилось, я сначала просто в шоке была. Потом привыкла{\ldots} сейчас вот вообще в завучах хожу{\ldots}

--- А как получилось-то?

--- Ну, захожу я утром в тронный зал. Смотрю, Рэрити со Свити стоят, а Твайлайт от них с руганью телепортируется. Свити как услышала, так сразу заинтересовалась~--- что это значит, да к чему оно{\ldots}

--- Едрёно сено?

--- Ну да. Рэрити сама обалдела, как услышала, ну ты же её знаешь. И вместо ответа только «э-э-э{\ldots}». Я подхватила, мол, Эпплджек должна знать{\ldots} извиняюсь, конечно, но сама понимаешь, откуда такой лексикон.

--- Агась, она до сих пор со стыда угорает{\ldots}~--- Эппл Блум вернулась к самой первой записи в блокноте и поверх вопросительного знака поставила жирную галочку.

--- «Сгорает», ты хотела сказать. Ну вот. Я, конечно, чаю предложила, спросила, как дела{\ldots} должен же кто-то был проявить вежливость? А Рэрити мне рассказала про магическую икоту у Свити{\ldots} у меня у само́й в детстве такое было, я в ответ поделилась. Вот, говорят, как раз пришли посоветоваться с Твайлайт, чего тут делать, да как учить мелкую себя контролировать{\ldots} А я возьми, да и скажи, что сама могла бы её научить.\looseness=-1

--- Из вежливости? --- проницательно поинтересовалась Эппл Блум.

--- Ты знаешь, не только. Мне как раз перед тем ночью снилось, что у жеребят какой-то урок вела{\ldots} кстати, даже сбылось в~итоге, спустя время{\ldots}

Эппл Блум резюмировала в блокноте: \emph{«Старлайт рассказали версию как-бы. Ей снилось что вела урок, предложилась не только из вежливости»}.

--- {\ldots}вот. А Рэрити как упёрлась рогом, вроде не гожусь я, и на прошлое моё намекает. Ну, я сейчас-то уже довольно спокойно про это, а тогда{\ldots} Тоже упёрлась, в общем. Вообще, сама удивляюсь, по сути-то всё правильно. Не иначе, тот сон сказался. Тебе о прошлых делах, небось, тоже нужно?

--- Неа, спасибо, я у Свити уточню, если что{\ldots}~--- Эппл Блум подчеркнула слова «список стрясти».~--- А дальше?

--- Дальше Твайлайт вернулась совсем сильно не в духе. Даже не знаю, что за муха её укусила~--- на Рэрити наехала, мол, кто ты такая критиковать, на меня чемодан компромата вывалила, так что хоть со стыда сгорай, и в итоге всё за нас решила. Вот, говорит, тебе ученица, радуйся, в кои веки что-то умное придумала! Рэрити ещё попробовала что-то вякнуть, да куда там{\ldots}

Новая строчка в блокноте выражала удивление: \emph{«Твайлайт укусила муха? Матюгалась при Свити, Старлайт в компромате обваляла, Трикси к чейнджлингам послала!»}.

--- Тётя Трикси рассказывала, была идея Свити в кантерлотскую Школу сплавить.

--- Была, точно. Сама же Трикси и предложила. Только эта идея долго не продержалась~--- какая там кантерлотская Школа, когда тут и дома можно весело приключаться, да принцессе Луне письма писать!

--- А вот про письма?

--- А то ты сама не догадываешься, у кого она эту идею позаимствовала. Хочу, говорит, письма принцессе писать о выученных уроках, и всё тут! Ну, у меня по сравнению с Твайлайт труба пониже и дым пожиже, сама понимаешь. Кстати, Твайлайт её как принцесса не устроила~--- сказала, что ненастоящая! В том смысле, что когда она сюда приехала, то принцессой же ещё не была.

--- Интересно,~--- пробормотала Эппл Блум уголком рта, записывая: \emph{«Письма принцессе --- труба пониже и дым пожиже. Твайлайт типа ненастоящая принцесса, сначала не была!»}. --- А у вас из тех писем чего-нить осталось?

--- Так это ты в Кантерлоте и спрашивай. Если уж тебе во дворце такое поручение дали и такую бумагу выдали, наверняка покажут что-нибудь.

--- Ой, правда ведь! Пасибочки, тётя Старлайт!

Страница блокнота завершилась подчёркнутой фразой: \emph{«Про письма спросить во дворце!»}.

\vspace{5mm}
\begin{center}* * * * * *\end{center}
\vspace{5mm}

Скуталу опять прервала чтение.

--- Та-ак, вот тут тоже интересно. Если ты сдвинула ей возраст на три года младше, то письмо в конце книжки{\ldots} ну, которое про первый усвоенный урок с восторгами?..

--- Основано на реальном письме,~--- сообщила Эппл Блум.~--- Маленько изменено, ага. Художественная, блин, правда.

--- А написано, что приводится в авторской орфографии и~пунктуации{\ldots}

--- В авторской,~--- подтвердила Свити.~--- Моё же письмо было, я для книжки сама детскую версию и написала. Добавила восторгов и наивностей, а по сути всё как есть.

--- Ты дальше читай!~--- поторопила Эппл Блум.~--- Тут щас самое прикольное будет!
\begin{center}* * * * * *\end{center}

--- О-о-о!..~--- простонала Рэрити, едва глянув на бумажку и~выслушав вопрос.~--- Как же некстати{\ldots} я уже так надеялась, что всё это в прошлом{\ldots}~--- она телекинезом подтянула к себе любимую кушетку и привычно плюхнулась на неё в драматической позе.

--- Да я как бы уже почти про всё в курсе,~--- осторожно сказала Эппл Блум.~--- Мне Свити уж объяснила, шо у вас по семейной истории одно числится, а по правде другое было.

--- Она знает, что я знаю, что она мне про ту магическую икоту врала?

--- Э{\ldots} вродь как нет{\ldots}~--- над словом «закосила» появилось лихорадочно нацарапанное \emph{«Рэрити просекла!»}.

--- Намекни ей, пожалуйста, дорогуша.

--- А что ж вы сами{\ldots} не того?

--- Да знаешь ли ты, что такое «стыдно»?~--- надрывно вопросила Рэрити.~--- Ничего ты не знаешь, девочка!

--- Э-э{\ldots}~--- умно прокомментировала Эппл Блум.

--- Почему, по-твоему, я так надеялась, что эта история забыта? Я же была в одном шаге от величайшего позора своей жизни! Я~--- я, Рэрити!~--- оказалась творчески бессильной! Мне заказали платье, а я что?!

--- Что?~--- осторожно спросила Эппл Блум.~--- Свити же говорила, вроде какое-то платье там было{\ldots}

--- Пффф! «Платье»! Быдланский балахон! Бездарная банальщина! Безвкусная{\ldots}~--- модельерша помахала копытом в воздухе, явно подыскивая ещё одно слово на ту же букву, не нашла и небрежно отмахнулась:~--- Платье, дорогуша, должно подчёркивать уникальность и индивидуальность заказчицы! Говорить о них всему миру! А я три дня ломала голову и ничего не придумала. Ни-че-го! Сшила ерунду по выкройке из книжки времён своей бабушки~--- думала, может хоть как-то сойдёт за оригинальность, мода иногда возвращается. Просто чтобы хоть что-то было. Поэтому, когда я увидела, что сестра вот-вот чего-нибудь натворит, я ей сразу кристалл подставила под импульс, чтобы импульсом тем по моей халтуре прилетело!

--- Не поняла. Чё, обгорелые лохмотья разве лучше, что ли?!

--- Конечно! Намного, намного лучше! Ну как же ты не понимаешь: проваленный заказ~--- это позорище. А работа, погибшая из-за чрезвычайных обстоятельств~--- это совсем другое, это уже сочувствие! Ну в самом деле, не винить же ребёнка с психотравмой из-за испуга. Ну, вернула бы аванс, и все дела.

--- Опять не поняла. А сразу сказать про погибшую работу{\ldots}

--- Дорогуша!~--- Рэрити выпрямилась.~--- Я всё-таки ещё не дошла до того, чтобы делать крайней свою сестру вообще без всякого повода! Ты себе не представляешь, как я обрадовалась, что тот дурацкий импульс решил проблему и врать вообще не пришлось.

Эппл Блум почувствовала, что у неё ум начинает заходить за разум от этих морально-модельерных дилемм, и решила немного сменить тему:

--- А чё эт’ за заказ-то был, шо кактусы кстати пришлись?

--- О, платье для одной журналистки. Ведущая светской хроники, обожающая скандалы. Смотри, как хорошо получилось: платье бабушкиного фасона с живыми кактусами. То есть светские скандалы стары как мир, но мастер всегда углядит в них новые острые моменты. Или вот ещё так: корни свежих скандалов уходят в бабушкины времена{\ldots}

--- Случайно, не Квик Чаптер?

--- Ты её знаешь? Нет, не она, Чаптер работает по сенсациям вообще, а не по светским скандалам{\ldots} Ну вот, теперь тебе известно всё.

--- Не всё. Тётя Рэрити, а почему вы сначала не хотели, чтобы Старлайт Глиммер учила Свити?

--- О-о-о!..~--- тётя Рэрити приняла на восемьдесят процентов более драматическую позу.~--- Ну зачем, зачем тебе нужно вытаскивать на белый свет ещё и это моё позорище?! Впрочем, какая теперь разница{\ldots} Девочка, я просто не была уверена, может ли она \emph{научить}. Уметь и учить~--- это же совершенно разные вещи! Возьми хотя бы меня: никто не посмеет сказать, что я не умею шить, но научить другого{\ldots} да ты на сестру мою посмотри, и это за столько лет!

--- Вы хотите сказать{\ldots}

--- Да! Я пыталась намекнуть Старлайт, что у неё в прошлом никакого опыта наставничества не было, и стоит ли браться сразу с ребёнком~--- а она подумала, что я имею в виду совсем другие моменты её прошлого. А мне даже в голову не приходило, что это может прийти ей в голову! То есть до тех пор, пока не вернулась Твайлайт и не огласила эти моменты вслух, как бы от моего имени! Ей, похоже, тоже кое-что в голову не приходило, тактичность я имею в виду{\ldots} Никогда бы я сама такого не сказала! И вот стоим мы там~--- я вот-вот сгорю со стыда, Старлайт вот-вот расплачется, и Твайлайт вся такая довольная, что меня перед ней подставила и всё за нас решила!

--- То есть вы не возражали?

--- Ну{\ldots} сказать по правде, мне эта идея тогда не очень нравилась, но не вот так же в лоб! Это теперь-то понятно, что получилось как нельзя лучше~--- личная ученица завуча Школы Дружбы, протеже принцессы Луны, как-никак!~--- а тогда я себе всю голову сломала, как бы перед Старлайт извиниться. И опять ни одной идеи, прямо как с тем платьем{\ldots} Ну, конечно, сказала, что извиняюсь, когда она ко мне на следующий день приходила Свити Белль отпрашивать в поездку, и сразу же разрешила{\ldots}

--- Э-э{\ldots} --- осторожно сказала Эппл Блум. --- А вы вообще в~курсе, что тётя Трикси и тётя Старлайт надеялись, что \emph{не} разрешите? Ну, чтобы тётя Старлайт за ту пару дней от шока отошла и маленько с идеей пообвыклась?

Ответом ей был отчётливый стук, с которым Рэрити сделала фэйсхуф.
\begin{center}$\triangleleft\star\triangleright$\end{center}

Выйдя из «Карусели», Эппл Блум почувствовала, что ум зашёл за разум окончательно. У неё уже не оставалось никакой уверенности в том, что за следующую пару дней она успеет свести шесть услышанных рассказов в единое целое, и она решила просто зафиксировать их по отдельности.

А пока заметка «Твайлайт укусила муха» пополнилась ещё одним примечанием: \emph{«Рэрити перед Старлайт подставила!»}.

\vspace{2mm}
\begin{center}
	* * * * * *
\end{center}
\vspace{2mm}


--- Офигеть,~--- заметила Скуталу, переведя дыхание.~--- Сколько вокруг вас накрутили-то{\ldots} с одной Твайлайт в такую интригу влезла, что до сих пор вопросы вылезают, другой наставницу аж из самого́ Кантерлота сосватали{\ldots} А я-то думала, что у меня всё сложно.

--- Мож, и не ошибалась,~--- заметила Эппл Блум.~--- Смотрю я на твоё ученичество, и чё-то как-то{\ldots}

--- Не поняла. Чё и как тебе не так?

--- Ну, сколь я тот свиток о твоих весёлых стартах на пинковой тяге помню, он же тебе тогда всё замкнул на комплекс, как бы его в чём-нить уделать? Вон, ты нам третьего дня об этом плакалась.

--- И что с того?

--- То, шо он это замкнул на себя ещё \emph{до того}, как у вас речь зашла о каком-то там ученичестве!

--- Фигня. Раз ему первому удалось меня взлететь, на себя и~замкнул. Понятно, что я бы такое всё равно век не забыла, хоть там ученица, хоть нет.

--- Можно я?~--- встряла Свити.~--- А тебе не странно, что тебя кто только учить не пытался, и всё без толку? Мне вот не верится, что среди них одни лишь неумехи попадались! С учётом того, что он псионик{\ldots} ой, не одними только словами и пинками тебя на крыло поставили!

--- Да и плевать. Поставил же.

--- А ещё?~--- вернула себе инициативу Эппл Блум. --- По твоему́ опять же рассказу, он потом в Кантерлоте насчёт «замкнуть круг» задвигал, когда мирились. И тебе в том «круге» тоже место нашёл!

--- Это четыре-то года спустя?

--- А проблеме сколь годов? Было время заранее подумать и~придумать.

--- Не поняла, вы до меня чего донести-то хотите? Что всё может быть не просто? Да и хвост покласть! Главное~--- во!~--- Скуталу развернула крылья и вызывающе уставилась на подруг.

--- А и правда, чего мы упёрлись{\ldots}~--- пробормотала Свити.~--- Ладно, если там ещё чего и спрятано, то всплывёт. А тут{\ldots}~--- она кивнула на листы с текстом, собирая их в аккуратную стопку,~--- я так понимаю, самое главное осталось?

--- Агась. Две главы ещё{\ldots}



\chapter*{Глава девятая, о том, как последняя из Меткоискателей первой столкнулась с Большим Откровением}\addcontentsline{toc}{chapter}{Глава девятая, о том, как последняя из Меткоискателей первой столкнулась с Большим Откровением}

--- Две главы осталось,~--- напомнила Эппл Блум.~--- Первая небольшая, эт’ где меня в нестыковки носом натыкали и сюрпризом шарахнули. Вторая поболе, там самое главное. Думаю, шо можно их вместе зачесть, ежли вы потянете.

--- Ну так давай я одну, а Свити другую, --- предложила Скуталу. Попробовала взмахом крыла сдвинуть верхний лист из стопки и чуть не сдула её со стола целиком. Торопливо прижала копытцем. --- Тьфу, блин{\ldots} Ладно, поехали.
\begin{center}* * * * * *\end{center}

--- А я про вас в библиотеке спрашивала!~--- с порога сообщила Эппл Блум.~--- То есть, ой, здрасьте!

--- Что нашла? Тебе тоже не болеть.

--- Две пьесы в соавторстве с какой-то Санни Скайс.

--- Было дело. Читала?

--- Не-а. Пьесы тяжко читать, они ж ведь так пишутся, что им ещё постановщик нужен, который по-своему показывает. Да и времени не было.

--- Ну, тут ты не совсем права. Есть много замечательных пьес, которые прекрасно читаются с листа. Впрочем, в данном случае постановка действительно значит всё{\ldots} А что по делу?

--- Кажись, до меня допёрло, зачем нужны эти официальные версии.

--- Уже неплохо. Ещё что?

--- Во!~--- Эппл Блум выложила на стол шесть листов бумаги с~шестью отпечатанными рассказами.~--- Думала, как всё это можно вместе свести, но тут офигеешь, пока сведёшь.

Квик Чаптер окинула листы буквально одним секундным взглядом~--- казалось бы, за такое время прочитать их было невозможно, однако следующая реплика свидетельствовала об обратном:

--- Мда. Семейные тайны Свити Белль, интрижка по её сплаву со стороны Старлайт Глиммер и Трикси{\ldots} и везде это твоё «едрёно сено»{\ldots}

--- Просто я про него первым делом в этой связи услышала,~--- пояснила Эппл Блум.~--- Ну и вродь как за него зацепилась.

--- Забавная деталь, да. А что скажешь про главное?

--- Не знаю даже, блин. Чё-то мне эти семейные тайны неохота по секрету всему свету растрёпывать. Свити же мне подруга всё-таки.

--- Так ведь тебя никто и не заставляет. Ты автор, только тебе и решать, до какой степени откровенничать в будущей книге.

--- И как эту степень определить?

--- Чутьём. Постепенно оно у тебя появится. А пока{\ldots} если тебя интересует моё мнение?..

Эппл Блум кивнула.

--- Пожалуй, я бы не стала раскрывать семейную тайну, тут ты правильно решила. Вариант с магической икотой вполне нормален{\ldots} он правдоподобен, достаточно забавен сам по себе и всё равно не имеет никакого отношения к тому, ради чего задумана эта затея. А вот едрёно сено я бы оставила. Интрижку по сплаву{\ldots} если твоя подружка о ней знает, то тоже оставила бы. Но девять шансов из десяти, что знает. Сейчас это уже ничего, кроме смеха, не вызовет. Странно другое{\ldots}

--- Что?

--- Ты поняла, \emph{как} всё начиналось. Но из собранных тобой материалов совершенно невозможно понять, \emph{почему}! А ведь именно это является нашей конечной целью{\ldots} тебе сказали, что я знаю кое-что, чего не знают другие, но даже если бы не знала, то насторожилась бы. Есть такое слово~--- «туфта»{\ldots}

--- Где туфта?!~--- Эппл Блум уставилась на листы с подозрением.

--- Смотри сама. Ты опросила шестерых, и пятерым из них твои расспросы не доставили особого удовольствия. Ладно твоя сестра~--- она тут только очень косвенно при чём. Ладно Старлайт Глиммер и Трикси~--- им скрывать уже нечего. Но вот твоей подруге и её старшей сестре очень даже есть что скрывать~--- а они по первой просьбе всё выложили. И только один рассказ в~этом смысле отличается от других{\ldots} причём по странному совпадению, он ещё и самый короткий, не считая рассказа Эпплджек. И это полностью согласуется с тем, что известно мне{\ldots}

Эппл Блум положила копытце на лист, в верхней части которого было напечатано «Пр. ТВАЙЛАЙТ СПАРКЛ».

--- Да. У тебя никаких рабочих заметок не сохранилось? Из них иногда можно выудить важные детали.

Из недр сумки на стол был вытащен блокнот.

--- Похвальная предусмотрительность{\ldots} Ну, давай посмотрим, что ты там набросала по ходу своих расспросов{\ldots}

Взгляд журналистки пополз по кривым строчкам~--- на сей раз куда медленней и внимательней, чем по печатным листам.

--- Стоп! Это что?!~--- Квик Чаптер указала на запись «часть истории пропустила».

--- А, ну это принцесса Твайлайт так сказала. Дескать, разговор про то ученичество без неё начался, она уже потом к нему подключилась, когда вернулась.

--- Вот именно так и сказала, что часть истории пропустила? Этими словами? Сама?

--- Агась.

--- Ну-ка вспоминай, когда именно: до или после рассказа?!

--- М-м-м{\ldots} а так и так. Она это дважды говорила.

--- Ну надо же! Твайлайт Спаркл научилась правдивой лжи второго рода. Ай да принцесса Дружбы!

--- Чего?

--- Слушай и учись сама. Есть два основных способа солгать, не сказав ни слова неправды. Первый довольно простой: нужно вкладывать в свои слова абсолютно буквальный смысл. Или, если отвечаешь на вопрос, отвечать именно про то, что было спрошено, а не про то, что спрашивающий на самом деле хотел узнать. Про всё, что сверх этого~--- молчать, если не спрашивают.

--- А, ну это знаю. Это как на всяких конкурсах юных талантов награждают всех. Типа, первое-второе места по правде, а третье всем остальным. И родители потом такие: «аж сорок участников было, а наш-то третье место занял, вот диплом!».

--- Именно. Вот диплом за третье место, и всего было сорок участников. А что из этих сорока с такими дипломами ушли тридцать восемь, говорить вовсе не нужно. Да ты же в нашу прошлую встречу сама такой трюк провернула, когда сказала, что тот долдон из дворца пытался рассказывать тебе про вазу и~давал двадцать бит.

--- Агась.

--- Это правдивая ложь первого рода. Второй род гораздо сложнее, тут нужно так сформулировать свои слова, чтобы в~них всё было правдой, но услышали совсем не то, чего было вложено. Вот она тебе сказала перед началом рассказа, что часть истории пропустила~--- ну правильно, разговор-то начался без неё! А потом рассказала и повторила. И вот придёшь ты к~ней с~претензией, что не всё было рассказано, а она посмотрит на тебя большими удивлёнными глазами: конечно, не всё! Я же прямо два раза говорила, что часть истории пропустила! Ты ничего не переспрашивала, я и подумала, что тебе не нужно! Откуда мне было знать-то?!

--- Бли-и-ин{\ldots} С этими вашими приёмчиками точно шизиком станешь{\ldots}

--- Параноиком, а не шизиком, если уж быть точным. И кто бы говорил: в прошлый раз ты мне написала эссе, как кого-то по блату пристраивали в Школу для одарённых единорогов, а теперь приносишь материал, из которого следует, что именно так предполагалось поступить с твоей подругой.

--- Ой, блин{\ldots} Не, это просто совпадение.

--- Вот-вот. Сначала совпадения, потом слишком много совпадений, потом паранойя{\ldots} Привыкай, раз в эти игры ввязалась.

--- А она, принцесса Твайлайт то есть, точно не всё рассказала?

--- Точно-точно. Скажи-ка ей при случае вот что{\ldots} э, лучше запиши: «Тётя Твайлайт, вот вы в прошлый раз часть истории пропустили, а можно всё-таки её услышать? Про ту ночь, когда Спайк ходил во сне, а вам не спалось, и чего вы из той ночи вынесли?».

--- А спросит, откуда я это взяла?

--- Так и скажи: от того, кто в курсе этих снов.

--- Ой, это ж опять тот второй род, да?

--- Точно. Ты этим никого конкретно не назовёшь, но подумано будет понятно что.

--- А вы-то откуда в курсе?

--- О-о-о, это вторая половина истории. Но давай сначала с~первой закончим. С учётом того, что́ я тебе пару минут назад сообщила{\ldots} попробуй-ка найти в этом ученичестве некое общее обстоятельство, которое здесь присутствует в нескольких местах разом{\ldots} и которого тут что-то слишком много?

--- Да сны же! Свити приснилось, как эту сферу ставить, Старлайт приснилось, что она жеребят учит, Спайк, я так поняла, что во сне ходил, а Твайлайт, наоборот, не спала чего-то?

--- Молодец!

--- Оба-на! А ведь Твайлайт ещё говорила, что как раз тогда получила письмо от принцессы Луны с перепутанной датой!

--- Даже так? Дважды молодец! Вот это очень важная деталь, нужно было в печатную версию обязательно включить.

--- Да я не подумала. Ну, было письмо, которое типа важное, а на самом деле нет, это-то я напечатала.

--- В журналистском расследовании лишних деталей не бывает, запомни это раз и навсегда. Ну что ж, теперь тебе понятна задача на время до нашей следующей встречи.

Эппл Блум подтащила блокнот обратно к себе:

--- Свести шесть рассказов в один без этих семейных тайн?

--- Так. Без них тебе будет попроще.

--- Спросить у принцессы Твайлайт вот это, чего сегодня с ваших слов записала?

--- Так.

--- Поговорить с принцессой Луной про те сны и то письмо?

--- Да. Скажешь во дворце, когда тебя будут отправлять домой, тебе устроят эту встречу в один из следующих визитов.

--- Ой, а она ответит?

--- Конечно. Вспомни, от кого исходило предложение написать книжку. Я тебе даже больше скажу: именно здесь кроется корень всей идеи.

--- А ещё больше сказать~--- никак? Ну, пож-жалуйста!

--- Хм. Скажем пока так: принцесса Твайлайт Спаркл придумала весьма нетривиальную затею. Это было довольно сложно, и она обратилась за помощью. Ей помогли, но при этом{\ldots} хм{\ldots} её затея оказалась частью другой затеи. А в самом центре затей оказался тот факт, что некие Свити Белль, Старлайт Глиммер и~Трикси сошлись вместе.

--- А эти их гастроли у чейнджлингов{\ldots} тоже затея?

--- О да.

--- На что мне обратить внимание?

--- А тебе факты известны?

--- Ну{\ldots} Свити все уши прожужжала, да я сейчас ещё кое-что поняла.

--- Давай послушаем, что именно ты знаешь.

--- Вроде как они отправились на ту гастроль, Старлайт их перебросила телепортом. Недалеко от улья начали репетировать или как-то ещё готовиться. Потом на улей напал насекомый монстр, они его жук-пуком называют и всё время зачем-то добавляют, что это рабочее название. Я так понимаю, они этого жука от улья прогнали, своим телепортом опять же, но Трикси чего-то напутала, и он вместо дискордовых куличек оказался в Понивилле. Ну, они-то там, в улье, это знать не могли, конечно~--- пока они там отдохнули, да пока их там наградили, да пока в обратный путь двинулись{\ldots} о представлении-то речь уже не шла. Приезжают в Понивилль, и здрасьте~--- тут опять этот жук-пук! Прогнали его опять, на этот раз вроде как надо, их тут снова наградили, и аккурат после этого Свити уже и слышать не хотела про Школу одарённых единорогов. Как-то так.

--- Всё правильно. Ну тогда я повторюсь: попробуй найти общее обстоятельство, которое присутствует в нескольких местах, и которого тут почему-то слишком много.

--- Хм-м-м-м{\ldots} --- Эппл Блум задумалась, но на этот раз ей ничего не приходило в голову.

--- Трижды,~--- подсказала Квик Чаптер.

--- Три раза? Блин{\ldots} не вижу, вот хоть убейте.

--- Тогда, наверное, сто́ит отложить это до следующего раза{\ldots}

--- Нет! Ну, подскажите ещё, пожалста!

Журналистка поджала губы.

--- Насекомые{\ldots}~--- с явной неохотой проговорила она.

--- Эм-м{\ldots} а почему трижды-то? Этот монстр нападал на улей и на Понивилль. Где третий раз?

--- Неудавшийся пикник,~--- Квик Чаптер постучала по листу бумаги с записью рассказа Свити Белль.

--- Упс. Точно.

--- Ты переключилась на дальнейшие события и перестала обращать внимание на предшествующие. Нельзя так делать, всё взаимосвязано{\ldots} по крайней мере, нужно исходить из того, что всё \emph{может быть} взаимосвязано.

--- Поня-атненько{\ldots}

--- Раз уж мы об этом заговорили, хочу спросить у тебя вот что. Ты сама-то при той атаке на Понивилль присутствовала?

--- Агась. Ну, не в само́м городе, я это с фермы видела, чуток со стороны.

--- Сколько та атака длилась по времени?

--- Что-то между тремя и четырьмя часами. Ой, чё там было! Принцессы Твайлайт с подругами нету, у них опять что-то срочное случилось, тётя Старлайт с тётей Трикси на ту свою гастроль укатили, тётя Бон-Бон с тётей Лирой в Кантерлоте на концерте тёти Октавии{\ldots} Чего делать, никто не знает, магия-то на этого жукопука не действует. Потом кто-то допёр поднять в~небо всех городских пегасов, чтоб они эту вонь ветром отгоняли. Ну, они и~отгоняли все те три часа, ухайдакались вусмерть{\ldots} Но всё ж продержались, пока гастролёры не вернулись. Самое-то смешное~--- перед тем, как того вонючку вышвырнули, он крайний раз навонять ещё успел, а пегасы-то рты на такое зрелище с вышвыриванием раззявили, ну город и накрыло. Ну, то есть, это нам на ферме смешно было, а в городе-то, поди, не очень.

--- Весьма интересно,~--- хмыкнула журналистка.~--- Весьма{\ldots} Раз уж ты всё это в красках описала, давай ковать железо, пока горячо. Отвлекаясь от всего остального, тебе конкретно в этих двух атаках жук-пука ничего не кажется странным? Сначала на улей, а потом на Понивилль?

--- Да вроде нет, а чё? Ну, лажанулась тётя Трикси, она ж сама признаёт.

--- Три-четыре часа атаки, по твоим же собственным словам. А сколько времени занимает дорога от чейнджлингов до Понивилля? А сколько времени они ещё провели в улье после того, как тётя Трикси лажанулась? Ты об этом тоже говорила несколько минут назад.

--- Оба-на{\ldots}

--- Вот именно. Эти атаки не сходятся по времени, между ними прошли почти сутки. У чейнджлингов и в Понивилле был не один и тот же монстр. И тётя Трикси здесь совсем ни при чём.

--- Бли-и-и-ин! А может, он просто где-то отлёживался после того, как его от улья телепортнули? Отлежался, пришёл в себя и~напал?

--- Вообще да, такое объяснение логично и вполне возможно. Но не в этом случае{\ldots} видишь ли, я просто \emph{знаю}. Знаю совершенно точно.

--- Откуда?

Журналистка вздохнула и встала из-за стола. Прошлась туда-сюда по комнате, зачем-то выглянула в окно. Сообщила в~пространство:

--- Я так надеялась, что с этим можно будет подождать хотя бы до третьей встречи{\ldots}~--- и сколдовала на себя какое-то заклинание, окутавшее её ярко-зелёным облаком.

Когда аура рассеялась, перед обалдевшей Эппл Блум, выпрямившись во весь свой немалый рост, стояла королева чейнджлингов.
\begin{center}* * * * * *\end{center}

--- Обед?~--- коротко поинтересовалась Эппл Блум.

--- Нафиг,~--- столь же коротко отозвалась Скуталу.

Свити Белль согласно кивнула:

--- Дальше!

Перед ней на стол легла ещё одна стопка бумажных листов{\ldots}



\chapter*{Глава десятая, в которой детские игры Меткоискателей сменяются взрослыми}\addcontentsline{toc}{chapter}{Глава десятая, в которой детские игры Меткоискателей сменяются взрослыми}


Эппл Блум отреагировала не самым достойным, но абсолютно естественным для кобылки её возраста способом~--- завизжала во всю мощь своих лёгких.

Кризалис стоически выдержала этот звуковой удар, дождалась паузы на перевод дыхания и сообщила:

--- На улице ничего не слышно. Заклинание против подслушивания.

Продолжения визга не последовало, зато следующая реакция была уже куда более осмысленной и похвальной~--- Эппл Блум соскочила со стула и бросилась на королеву, нагнув голову.

За мгновение до контакта раздался хлопок. Кризалис исчезла со своего места и появилась на пару шагов правее. Кобылка едва не врезалась в шкаф, но её придержало магической хваткой за хвост, и она лишь чуть стукнулась плечом. С верхней полки свалилась и разлетелась вдребезги тарелка.

--- Королева Кризалис, конечно, великая злодейка,~--- прозвучал насмешливый комментарий,~--- но зачем же посуду-то бить?

Третьей реакцией Эппл Блум было броситься в окно~--- Скуталу однажды объясняла, как это нужно правильно делать. Выбивать не головой, а выставленными вперёд неё копытами, в~случае маленького окна целиться ими не в стекло, а в переплёт рамы, и обязательно складывать крылья перед ударом, плотно прижимая их к телу.

Крыльев, правда, в наличии не было, зато были сильные задние ноги. Толчок удался на славу{\ldots} но едва эти самые ноги оторвались от пола, вокруг кобылки с негромким «чпок!» возник зеленоватый упругий пузырь, на чём прыжок и закончился.

--- Считай, тебе удалось меня удивить ещё раз,~--- сообщила Кризалис.~--- Не знала, что ты умеешь эти пегасьи штучки.

Эппл Блум яростно лягнула пузырь изнутри. Сначала он легко поддался, но сразу же погасил удар своей упругостью. Кризалис вздохнула:

--- Придётся немного подождать. Думаю, речь идёт самое большее о четверти часа{\ldots} и если ты ещё не поняла, то тебе абсолютно ничто не угрожает. Равно как и твоим друзьям. Извини, что пришлось вот так, но вопли и погромы в центре Кантерлота мне совершенно ни к чему.

Она передвинула пузырь на середину комнаты и развернула так, чтобы в поле зрения его узницы оказались настенные часы.

Эппл Блум задумалась. Голос королевы не был злым, смотрела она тоже без злости. Орать и дрыгаться, очевидно, не имело никакого смысла, так что оставалось лишь ждать.

Ждать, как и было обещано, долго не пришлось. На девятой минуте в комнате сверкнула яркая вспышка, из которой возник свиток. Кризалис подскочила к нему и ловко толкнула дырчатым копытом так, что он влетел в пузырь и остановился там. Кивнула и сделала приглашающий жест.

Свиток представлял собой лист бумаги, который удерживался в свёрнутом состоянии двумя печатями на торцах. Изображение такой печати было известно всей Эквестрии по школьным учебникам{\ldots} собственно, кроме как на картинке в~учебнике, это почти никто и никогда не видел. Зато по тем же учебникам было прекрасно известно, что сломать такую печать может лишь тот, кому послание адресовано, а любая попытка прочесть его в обход печатей приведёт к тому, что свиток моментально рассыплется в пыль.

Кризалис снова кивнула. Эппл Блум сжала копытцами одну из печатей~--- та с негромким звоном исчезла. За окном мигнуло солнце, как будто его на долю секунды закрыла собой здоровенная птица.

Со второй печатью произошло то же самое. Личность отправителя удостоверялась самым надёжным способом, какой только был возможен во всей Эквестрии.

Текст письма гласил: «Дорогая Эппл Блум! Я понимаю, что ты сейчас шокирована, и очень извиняюсь за это, но ты и сама видишь~--- проблему отношений между пони и чейнджлингами действительно нужно решать. Причём не просто решать, а начиная с себя. Я обещала, что ты узнаешь кое-что новое, чего пока не знает никто другой, и теперь тебе известно: дружба между нашими народами началась гораздо раньше, чем принято считать. Это ещё не всё, тебе предстоит узнать и другие вещи, про которые заверяю тебя своим словом, что они развеселят тебя, а~не шокируют. Очень надеюсь, что ты не бросишь начатое из-за того, что сейчас на тебя свалилось! Твоей наставнице, я имею в~виду мою подругу и соавтора королеву Кризалис, можно и нужно доверять~--- верь, и тебе проще будет выполнить задачу».

Последняя фраза, в которой имя называлось прямым текстом, не оставляла никаких сомнений. Эппл Блум подняла взгляд от письма и посмотрела со смесью интереса и~настороженности.

--- Прочитала?~--- поинтересовалась Кризалис.

Кивок.

--- Успокоилась?

Кивок.

--- Орать и громить ещё будешь?

Эппл Блум чуть не кивнула в третий раз, потом спохватилась и помотала головой.

--- Сейчас{\ldots}

Пузырь опустился, так что находящаяся в нём кобылка оказалась сидящей на полу, и распался на зелёные клочья, быстро развеявшиеся в воздухе.

--- Что дальше?~--- спросила королева.

--- В смысле, «что дальше»?

--- У тебя сейчас есть выбор: уйти или остаться. Если уйдёшь, унесёшь с собой две тайны, я имею в виду личность Квик Чаптер и знание о том, что за дружбой с чейнджлингами кроется нечто большее, включая ученичество твоей подружки. У тебя хватит ума держать язык за зубами, не сомневаюсь. Если останешься{\ldots} узнаешь кое-что ещё, и придётся над этим работать.

--- А эта Квик Чаптер{\ldots}

--- Перед тобой. Единорожки с таким именем нет и никогда не было. Я никого собой не подменяла, если ты об этом. У этой Квик Чаптер даже инициалы мои, если ты не заметила.\footnote{«Quick Chapter» и «Queen Chrysalis».}

--- А как-нить по-другому нельзя было?

--- Как? Ну подскажи на будущее. Я же говорю, что думала открыться тебе не раньше следующей встречи, но ты меня буквально вынудила сделать это сейчас.

--- Чё?!

--- Чё слышала. После всех сегодняшних намёков и подсказок ты бы задумалась, откуда мне это известно. Сопоставив это с~другими фактами обо мне, легко могла бы догадаться сама. Так уж лучше, чтобы ты узнала правду здесь от меня же, чем додумалась дома и{\ldots} я не знаю, вскочила бы среди ночи с воплями об измене!

--- Какими фактами-то? Чё я о вас знала-то, пока мне во дворце не сказали?!

--- Зато после того, как сказали, много чего узнала. Начать хотя бы с жука на метке{\ldots} меня, конечно, предупредили, с кем я~буду иметь дело, и я заготовила объяснение{\ldots} но оно, по-моему, получилось корявым.

--- Нормальное объяснение,~--- буркнула Эппл Блум.~--- После того, как однажды оказалось, что метка с черепушкой означает талант археолога{\ldots} А что, рисунок изменить никак?

--- Никак. Если метка есть, то она отражает внутреннюю сущность, и хоть ты тресни. А гримировать её опасно и ненадёжно, Старлайт Глиммер подтвердит. Далее, моя магическая аура{\ldots}~--- Кризалис подхватила ей блокнот и повертела перед носом Эппл Блум, демонстрируя.~--- Ни у одного единорога нет ярко-зелёной магии, она вам вообще чужда! Из ваших только аликорны могут, и то не без труда. И заклинания работы с памятью~--- это высшая магия.

--- А это-то мне откуда?!..

--- От подружки. Как правильно выбивать собой окна, небось, усвоила?

--- Блин. Надо будет внимательнее слушать Свити про магию{\ldots}

--- Девочка, лишних знаний не бывает. От слова «совсем».

--- Блин, вы прям как дядя Виндчейзер{\ldots} А ещё что?

--- Твои же записи!~--- Блокнот раскрылся.~--- Две пьесы! В соавторстве с Санни Скайс! Неужели в Эквестрии ещё есть кто-то, кто не знает, что это псевдоним Селестии?!

--- Я не знала! И принцесса Твайлайт, кажись, тоже!

--- Допустим. А названия?! «Королевская свадьба» и «Похищение с превращением»~--- они что, тебе совсем ни о чём не говорят?!

--- О, бли-и-и-ин!!! А там чё, и правда{\ldots}

--- Нет. Это нормальные пьесы для театра, которым мы дали такие названия{\ldots} говоря твоими словами, чисто по приколу. Ну, то есть, парочка моментов там использована{\ldots} сама прочитаешь, если захочешь.

--- Дальше!

--- Дальше я всё-таки очень хочу знать, что ты будешь делать.

Эппл Блум посмотрела на Кризалис. На письмо. На дверь. Чуть подумала.

Потом подошла к столу и залезла на стул.

--- Почему?~--- лаконично спросила королева.

--- Интересно,~--- столь же лаконично ответила кобылка.

--- Готова мне доверять?

Вместо ответа Эппл Блум кивнула на письмо, так и оставшееся лежать на полу.

--- Можно?..

Та кивнула ещё раз. Кризалис подняла бумагу, пробежала глазами. Заметила:

--- Вообще, письма из-под таких печатей надо уничтожать сразу после прочтения.

--- Делайте.

Письмо вспыхнуло ослепительным пламенем, после которого не осталось даже пепла.

--- Хочешь, я перекинусь обратно? Мне всё равно.

--- Не надо. Буду это{\ldots} начинать с себя.

--- Как скажешь. Чтобы закрыть вопрос о доверии{\ldots} На пишущей машинке, что я тебе подарила, висят два моих заклинания. Одно очень сложное, с ним на этой машинке почти невозможно сделать опечатку, другое довольно простое. Позволяющее мне знать всё, что на ней печатают. Наверняка ты заметила, что я на твои бумаги посмотрела чисто символически.

--- Да, а зачем?

--- Всё-таки речь идёт о вопросах важных и тонких. На всякий случай. Ну и для экономии времени. Могу снять одно или оба, как скажешь. Потом покажи машинку принцессе Спаркл или Старлайт Глиммер, они подтвердят результат. Или сами могут снять. Вопросов это не вызовет, на журналистские машинки в редакциях следилки вешают сплошь и рядом ради оперативности.\looseness=-1

--- М-м-м{\ldots}~--- задумалась Эппл Блум.

--- А вот корректорское заклинание я бы тебе посоветовала оставить. Штучная работа, большинству магов такое не под силу.

--- Пусть остаются оба. По этому делу мне от вас вродь как секретить нечего, а если для школы чего понадобится напечатать{\ldots} вы же сможете потерпеть?

--- Смогу{\ldots}~--- Кризалис тоже взгромоздилась на стул, что в её настоящем виде было несколько сложнее.~--- Теперь спрашивай.

--- «Дружба между нашими народами началась гораздо раньше»,~--- процитировала Эппл Блум по памяти.~--- Когда?

--- Когда моя мать привела наш рой на земли Эквестрии. Около пяти с половиной веков назад.

--- А вы тогда уже{\ldots} э-э{\ldots}

--- Да. Это на моей памяти, я уже была, хоть и совсем малявкой. Пророчество о преображении было известно уже тогда, но его держали в тайне, передавая от королевы к королеве.

--- Почему? И чего было столько времени скрываться?

--- Смутные времена, вы же сами называете их Упадком Гармонии. Выживание роя было слишком важным вопросом, чтобы отказываться от скрытности и боевой формы.

--- И это чё, дружба называется? А нападение на Кантерлот?! А похищение принцесс?!

--- Это были пьесы про королевскую свадьбу и похищение с превращением. Если ты всерьёз думаешь, что я с парой сотен солдат могла бы захватить столицу или в одиночку повязать четырёх аликорнов{\ldots} это, конечно, лестно. Но не более того.

--- Ни фига́ себе пьесы!!! Нафига́?!

--- Про первую лучше спроси во дворце, это не моя идея. Если коротко, то предполагалось проверить в деле и окончательно подготовить будущих правителей Кристальной Империи, возвращение которой ожидалось в са́мом скором времени. Мне такая идея не понравилась, но меня упросили.

--- Санни Скайс?

--- Она са́мая. Большие девочки играют в большие игры. Очень большие.

--- А вторая? «Похищение с превращением»?

--- Предполагалось, что она открыто подружит пони и чейнджлингов.

--- \emph{Всех} чейнджлингов?

--- А ты молодец. Да, всех.

--- И чё пошло́ не так?

--- Я засомневалась. В самый последний момент, когда роль была уже почти отыграна. Можешь спросить Старлайт Глиммер, она расскажет, как это было. На кону стояло слишком многое.

--- В Эквестрии только один ваш рой?

--- Именно поэтому. Как показало дальнейшее, я сомневалась не зря.

--- Почему?

--- Сколько новых чейнджлингов появилось на свет после того преображения?

--- Да мне-то почём знать!

--- А я тебе скажу. Ни одного за последние два с лишним года, и никто~--- никто!~--- в преображённом рое об этом даже не задумывается. Понимаешь, что это значит?

--- Не-а.

--- Это значит, что мы должны появляться на свет вот такими, какими и были прежде. Лишь повзрослев, мы \emph{можем} преобразиться{\ldots} но это преображение только для духовного развития, вопросов выживания оно никак не решает. Вам придётся смириться с тем, что «плохие» чейнджлинги и «хорошие» чейнджлинги неотделимы друг от друга, а нам{\ldots} то есть \emph{мне}{\ldots} да, \emph{мне} придётся торчать посередине и улаживать проблемы на три стороны. И как-то по-новому воспитывать своих детей, конечно.

--- Вы{\ldots} не можете измениться, как другие?~--- тихо спросила Эппл Блум.

--- Не так, как другие, --- уточнила с невеселой усмешкой Кризалис. --- Мне ничего не стоит принять любой облик, но он будет ненастоящим. Потому что только вот в этом виде{\ldots}~--- она развела копытами, --- я могу делать то, что является моей сутью. Давать жизнь новым чейнджлингам, чтобы рой продолжал существовать. Видимо, это как ваши метки: если ты занимаешься выживанием --- ты такой, если творишь и развиваешь --- такой. Выживание первично. И хоть ты тресни. И любой чейнджлинг видит настоящее обличье любого другого{\ldots} для этого чувства просто нет слов, но обмануть его невозможно.

--- Ой{\ldots}

--- Тогда было неясно{\ldots} если бы преображённый рой имел возможность воспроизводиться и размножаться~--- значит, дело однозначно во мне. Злодейка, неспособная к настоящей любви, и всё такое. Но эта возможность исчезла с преображением~--- значит, дело не во мне, просто мы так устроены. Тогда меня могли обвинить в обмане~--- дескать, не может, не хочет, притворяется, врёт{\ldots} А так у нас ещё есть время и возможность что-то на этот счёт придумать. Правда, времени осталось уже совсем мало, потому тебя и призвали.

--- Почему совсем мало?!

--- В отсутствие феромонов{\ldots}~--- Кризалис досадливо мотнула головой и перешла на речь попроще:~--- Моё тело уже довольно долгое время не получает внешних сигналов, что с роем всё в порядке. Что он окружает меня. С точки зрения природы это означает только одно~--- рой погиб! В этой ситуации главной и единственной задачей королевы является немедленное создание нового роя. Природа не дура, это срабатывает независимо от желания и почти не контролируется. Выживание первично! Можно лишь чуть-чуть подстроиться под обстоятельства, немножко ускорить или замедлить{\ldots} чем я сейчас и занимаюсь.

--- И сколь у вас{\ldots} нас того времени?

--- Следующее поколение чейнджлингов появится на свет через две, самое большее через три луны. После этого я уже не смогу свободно распоряжаться своим временем. Ещё через две-три луны они смогут самостоятельно покидать новый улей.

--- Ой, бли-и-ин!!!

--- Это мне ещё повезло, что ваша{\ldots} Санни Скайс{\ldots} хоть сейчас соизволила почесаться и осознать, что проблемы отношений пони с другими расами надо всё-таки решать до конца! Кто бы там ни ткнул её шилом в круп насчёт ситуации с грифонами, он уже заслужил низкий поклон от чейнджлингов{\ldots}

--- Дядя Виндчейзер. А вы его знаете?

--- Встречались. Та-а-ак, кажется, начинаю понимать! Это что, он читал в вашей Школе Дружбы грифонские стихи?!

--- Агась.

--- Дожили. Паладин Найтмэр Мун преподаёт в Школе Дружбы{\ldots} похоже, нас прокляли гораздо сильнее, чем я~думала{\ldots}\looseness=-1

--- Чё?!..

--- Старое зебриканское проклятье. «Чтоб тебе жить в интересные времена!». Если ты не заметила, они становятся всё более и более интересными.

--- Заметила{\ldots}~--- вздохнула Эппл Блум. Потом немного помолчала и тихо сказала:~--- Тётя Кризалис, только это ведь, наверно, невозможно. Ну, чтобы за две-три луны, пусть даже четыре-пять, написать книжку, которая бы всё объяснила, разрулила и уладила. Я-то уж точно не того{\ldots}

--- Конечно. Девочка, никто и не требует от тебя невозможного. Это же только первый шаг. От тебя хотят, чтобы ты написала книжку о приключении своей подружки для своего поколения. Из которой можно было бы понять, что вот есть Большая Игра, и в рамках её есть много чего интересного. Хохмы, совместные приключения с теми же чейнджлингами, новые знания{\ldots} Самое-то главное~--- что принцессы и правители тоже иногда делают глупости!

--- Не поняла. А кто там из принцесс чё сглупил?

--- Не хотела говорить прежде времени, но чего уж теперь{\ldots} Вся эта история с ученичеством и последующим приключением была интрижкой принцессы Спаркл, вздумавшей собрать свою команду героев для спасения Эквестрии в чрезвычайной ситуации.

--- Бред же какой-то.

--- Вот и я сначала так думала. Но если вспомнить историю{\ldots}

--- То?

--- Когда-то принцесса Луна собрала круг из двенадцати Детей Ночи, ты должна знать эту историю ничуть не хуже меня. Принцесса Селестия собрала Шестёрку Дружбы. Теперь вот принцесса Спаркл озаботилась{\ldots} похоже, это у ваших принцесс какая-то особенность.

--- Так, минутку, я извиняюсь. А где тут принцесса Кэйденс?

--- Принцесса Еды{\ldots}~--- с откровенно насмешливыми нотками фыркнула Кризалис.~--- Не знаю, можно ли ей засчитать за попытку её собственную свадьбу с капитаном гвардии и братом без пяти минут другой принцессы, но на большее она всё равно не способна. Вот уж кто ненастоящая принцесса.

--- Почему?!

--- Потому что та же Спаркл практически всего добилась сама. Её направляли, но вперёд она шла \emph{своими} ногами и препятствия прошибала \emph{своим} лбом. А эта{\ldots} один раз сделала что-то настоящее, за что и была вознесена{\ldots} потом за неё всё делали другие. Роль в пьесе про свадьбу она провалила, всё вытащила опять-таки Спаркл, и то мне пришлось импровизировать, чтобы дать ей такую возможность. Стыд и позор, я пять минут тупо пялилась в окно, распевая какую-то дурацкую песенку и старательно делая вид, что не вижу ничего за спиной{\ldots} И ведь кто-то мог подумать, что я впрямь такая дура{\ldots} ладно, не бери в голову.

--- А вы с ней{\ldots}

--- А вот это,~--- Кризалис расплылась в ухмылке,~--- не твоя проблема. И даже, хвала Праматери, не моя. Сей сюжет сочинила моя дорогая соавтор, ей и расхлёбывать. В чём она мне поклялась страшными клятвами. Даже знать не хочу, как она будет выкручиваться перед своими протеже{\ldots} а со Старлайт Глиммер я объяснюсь. Здесь особых сложностей не предвижу. Не отвлекайся от дела.

--- Интрижка по сбору команды.

--- Да. Наша третья совместная пьеса. Ты уже почти обо всём догадалась. Принцессе Спаркл помогли выполнить задуманное{\ldots} кое-кто обеспечил исполнителям правильные сны, я~обеспечила инсектоидные декорации{\ldots}

--- Какие-какие?!~--- Эппл Блум потянулась к блокноту.

--- Ну, насекомые. А кое-кто с са́мого верха всё это утвердил и одобрил. Мы немного поспорили, стоило ли вообще всё это затевать{\ldots} Тия оказалась права~--- стоило. Когда случилась заваруха в Холлоу-Шэйдс, с ней разбиралась именно эта команда в~чуть обновлённом составе, ну ты тоже в курсе. Другое дело, что это творение сумрачного гения Спаркл получилось \emph{слишком} хорошим~--- по недостатку опыта она сама не сумела толком его оценить, и достижение уплыло из её копыт в чужие.

--- Чё? Вот это вообще не поняла.

--- Холлоу-Шэйдс. Как нынче в книжках называют команду, впервые столкнувшуюся с культом?

--- Лунная пятёрка{\ldots} о!

--- Вот именно. Учитывая уже известные тебе обстоятельства формирования команды, просто грех было не прибрать такой шедевр себе. Рановато ещё вашей принцессе Дружбы в такие игры играться, хоть тут я не ошиблась.

--- Кстати, про обстоятельства. Тут уже два намёка было на что-то такое, что должно меня позабавить.

--- Нет, --- с улыбкой, но довольно твёрдо сказала Кризалис.~--- Не буду портить тебе впечатления и удовольствие. Сама раскопаешь, я тебе уже подсказала, что́ нужно спросить.

--- Нет, так нет,~--- вздохнула Эппл Блум.~--- Мне ещё чего-нить надо про эту вашу игру знать?

--- \emph{Нашу}. Добро пожаловать.

--- А мне, типа, не рановато?

--- Ну, пока ещё не на таком уровне, конечно. Я тут тебе приготовила копию одного своего письма{\ldots} думала отдать позже, но теперь уже неважно.~--- Рядом с шестью печатными листами лёг ещё один.~--- Зачаровала на тебя, никто другой текст не увидит. Но всё-таки осторожнее с такими письмами, ладно?

--- Агась{\ldots}~--- рассеянно отозвалась Эппл Блум, скользя глазами по строчкам.~--- Ой, а вот здесь: «Мы устроены слишком по-разному, чтобы я смогла адекватно передать словами свои ощущения от чрезмерной задержки на нынешней стадии{\ldots} однако полагаю, что выражение „я вся дико чешусь“ поможет тебе в какой-то мере их представить{\ldots}»?

--- А, это{\ldots}~--- Кризалис пожала плечами.~--- Всё-таки учти, что я писала год назад. Те признаки, что я принимала за близящееся преображение, оказались на самом деле предвестниками появления нового роя. Откуда мне было знать, раньше со мной такого не случалось{\ldots}

--- Понятно. Потом ещё перечитаю,~--- Эппл Блум стала складывать свои бумаги в сумку.~--- И когда нам теперь встречаться в~следующий раз?

--- Как будешь готова. Следующие две луны{\ldots} полторы уж точно я в твоём распоряжении. Просто напечатай на машинке, что хочешь встретиться, время выбирай сама. И не торопись{\ldots} сейчас тебе нужно многое обдумать. Иди, я буду ждать.

Кризалис телекинезом распахнула дверь.

--- Э{\ldots}~--- Эппл Блум глянула на дверь, потом на хозяйку дома.

--- Я не стану высовываться наружу,~--- усмехнулась та.~--- Оттуда не заметят. И там всё равно сейчас свои.

Эппл Блум выглянула и увидела возле крыльца Виндчейзера. Чуть подумала и прокомментировала в пространство:

--- Слово «свои» слишком двусмысленное. Какие стихи ты читал тогда на уроке?

--- «Считаться достойным{\ldots}». Плохой контрольный вопрос: на уроке была Оцеллия.

--- Как ты вообще на том уроке оказался?

--- Свити надоумила Старлайт.

--- Я тут минут десять назад говорила,~--- сообщили изнутри дома,~--- что нас прокляли гораздо серьёзнее, чем казалось.

--- Ты даже не представляешь, \emph{насколько} серьёзнее,~--- ответил Виндчейзер внутреннему голосу и обратился к Эппл Блум:~--- Пойдём, что ли?

--- «Насколько серьёзнее»~--- что?~--- тут же заинтересовалась та.

--- Твайлайт, кажется, раскопала ещё одну тайну прошлого. Даже не поленилась отыскать меня в рейсе, чтобы спросить моё мнение{\ldots} но с этой историей вроде не связано.

--- А ты ж говорил сестре, что не знаешь, зачем меня в Кантерлот вытащили?

--- Тогда не знал. Мне сказали после того, как ты согласилась. Про ту интрижку я в курсе{\ldots} в общих чертах, правда, но если услышишь чего двусмысленное~--- можешь обращаться. Интересно, что ты это спросила{\ldots}

--- А чё такого?

--- Я как раз хотел тебе кое-что сказать, ради чего и пришёл. Я рассказывал вам троим про своих родителей и упоминал тётю Гэри, помнишь?

--- Которая писала хронику? Ну да.

--- Не хронику, скорее дневники. Впрочем, не суть важно. Были там такие слова: «Придёт время, все узнают, зачем всё это, для чего эти страдания, никаких не будет тайн…». Вот.\footnote{В нашем мире эти слова были сказаны героями чеховской пьесы «Три сестры» (1900)}

--- И{\ldots} что?

--- Тебе выдалась возможность поработать на это. По-моему, ради такого сто́ит постараться, нет?

--- Агась. А куда мы идём?

--- На вокзал. Посажу тебя на поезд, тебе ещё нужно как следует обдумать, что и как будешь дома рассказывать про сегодняшнюю встречу. Насчёт держать язык за зубами, надеюсь, ничего объяснять не нужно?

--- Не нужно{\ldots}~--- буркнула Эппл Блум, заранее не испытывая восторгов от того, что держать язык за зубами придётся в~том числе от Скуталу и Свити Белль. Потом, однако, ей пришла в~голову неплохая идея, и она слегка успокоилась.~--- Только мне это{\ldots} с принцессой Луной надо бы поговорить. В следующий раз или там в послеследующий.

--- Хорошо, это передам{\ldots}
\begin{center}
	* * * * * *
\end{center}

--- Как-то оно обрывается,~--- заметила Свити Белль, перевернув только что прочитанный лист и убедившись, что на его обороте ничего нет.

--- Не решила пока с концовкой. Думаю, просто письмо своё добавлю, да и хватит.

--- Какое письмо?

--- То, шо написала как домой вернулась. Напечатала, вернее. На той са́мой машинке.

--- А нам?

--- Да пжалста{\ldots}~--- Эппл Блум прищурилась и проговорила по памяти:

\emph{«Тётя Кризалис!}

\emph{Я пока в поезде домой ехала, маленько подумала и кое-что придумала. Надо будет всё написать так, чтобы сначала было это ученичество и приключение с хохмами, и только потом уже вот это всё про большие игры, чего Вы мне сегодня говорили. Но без особых подробностей, а просто с намёками, и в самом конце добавить то Ваше письмо, и пусть кто не дурак, тот уже сам дальше соображает. И мне бы ещё раздобыть первое письмо Свити про ученичество, которое она после того приключения написала. Я себе добавила в список вопросов и обязательно узнаю.}

\emph{А ещё вот что. Когда появятся маленькие чейнджлинги, можно мне будет на них посмотреть, как только будет можно? Даже не мне, а подругам. Потому что я-то язык за зубами держу, но они в конце концов всё равно же узнают. И наверное, меня убьют за то, что я в этом участвовала, а они нет. А если им такое пообещать, то может ещё и не убьют. Потому что иначе фигу они такое увидят. Пожалуйста?}

\emph{А этот лист бумаги я сейчас сожгу, как допечатаю, я же понимаю. Про это не беспокойтесь».}

--- Ну и нормально. И кстати, из твоего собственного рассказа вытекает, что параноишь ты зря. Вот же, она знать не знала о~том уроке, только от тебя же о нём и услышала. Врать тебе у~неё никаких резонов не было.

--- Ни о чём не говорит,~--- заметила Скуталу.~--- Он не дурак, просчитать ситуацию и сделать такой ход вполне мог лично от себя.

--- Так Старлайт его спрашивала сразу после урока, и он ей прямо ответил «нет».

--- Пф! Вообще ни о чём! Что́ она там спросила~--- не готовился ли он к этому заранее? Ну так заранее не готовился, сделал ход экспромтом, вот тебе и «нет».

--- С вами свяжешься, так точно параноиком станешь{\ldots}~--- буркнула Свити Белль.

--- Ша!~--- восстановила порядок Эппл Блум.~--- Мы тут делом вообще-то занимаемся. По сути чё скажете?

--- Да всё здорово. Прикольно вообще получилось.

--- Вот именно! Прикольно, и{\ldots} и всё. Я того я боялась.

--- Не поняла!~--- хором сказали Скуталу и Свити.

--- Блин, ну как вам{\ldots} Во. «Большую игру» уже два раза издавали, и все про неё то ж самое грят: прикольно! И некоторые ещё: а чё там за фигня-то в конце добавлена, зачем бумагу-то на неё изводить было? Шо прикольно, то видят, а ради чего оно написано было{\ldots} Блин.

--- Ну{\ldots} э{\ldots} тебя же сразу предупреждали, что это только первые шаги,~--- заметила Свити Белль.

--- И чё? Всё равно обидно, блин. Это как если б тя в ту же Школу пригласили про магию рассказать, а ты на табуретку влезла и детский стих им: «я научилась колдовать --- не надо к~маме приставать{\ldots}». Или чё там в единорожьих садиках учат. Прикольно, да? По первому разу, считай, пшик вышел, и щас тоже так же будет, я чувствую{\ldots}

--- «Мелко, гнусаво, чуть слышно{\ldots}»~--- задумчиво пробормотала Скуталу.

--- Чё?

--- Так{\ldots} ещё один стих вспомнила.

--- А как, шоб не мелко и не гнусаво?!

--- Да есть мысль вообще-то. Четвёртый день голову ломаю{\ldots}

--- И?

--- Плохо. Одними крыльями там никак не извернёшься. По-любому магией добавлять нужно.

--- Ну так{\ldots}~--- начала Свити.

--- Телепорт,~--- безжалостно припечатала Скуталу.~--- Надо, чтобы меня в полёте пробросили вперёд с сохранением скорости и направления. Через критический участок, а дальше я уже сама.

Свити Белль увяла на глазах.

--- А чё не сразу в нужное место?~--- поинтересовалась Эппл Блум.

--- Туда, во-первых, нельзя телепортироваться в принципе, а~во-вторых, там нужна именно скорость на финише.

--- Совсем никак не прошмыгнуть?

--- Совсем. Может, мой смог бы извернуться, и то ещё не факт. А другие{\ldots}~--- Скуталу махнула копытом.

--- А его попросить{\ldots}

--- Нет! Сама должна!

--- А Старлайт попросить?~--- подала голос Свити.

--- Чего?! Ты серьёзно вообще?!

--- Да, а что?

--- Это с какого же сена завуч Школы Дружбы будет помогать нам наколбасить неслабый такой скандалище?

--- Ну, во-первых, она сама не столь уж давно \emph{такое} колбасила, что мы против неё сопля́чки!

--- А во-вторых?

--- Если для чего настоящего~--- уговорю!~--- буркнула Свити.

--- Настоящей некуда! Это будет такой урок дружбы{\ldots}




\chapter*{Эпилог, в котором Меткоискатели преподают урок дружбы и оказываются за это жестоко наказанными}\addcontentsline{toc}{chapter}{Эпилог, в котором Меткоискатели преподают урок дружбы и оказываются за это жестоко наказанными}


За последнюю пару лун тронный зал дворца в Кантерлоте изменился сильнее, чем за последнюю пару лет. Недавно разбитый витраж про одоление Найтмэр Мун едва успел смениться новым, про открытие Школы Дружбы, и вот{\ldots}

По меткому выражению одного из советников, зашедшего в~зал с какой-то мелкой надобностью и видевшего всё своими глазами, «никогда такого не было, и тут опять!».

Через три оконных проёма от обновлённого витража зияла дырища, в которой посвистывал ветер и виновница происшествия. Нельзя не признать, советник правильно выразился: она именно что была тут опять, и даже ухмылялась, прислушиваясь к доносившимся снаружи громким и экспрессивным обещаниям начальника караула.

Минутой позже в разгромленный зал вошли ещё четверо, а~ещё через пару минут в дверях показалась правительница. Советник бросил на неё один только взгляд, снова понял всё правильно и почёл за лучшее тихо исчезнуть.

--- Предмет этого разбирательства кажется мне знакомым,~--- ледяным тоном произнесла принцесса Селестия, глядя на Скуталу. За спиной у той стояли Эппл Блум и Свити Белль, а по бокам от них Старлайт и Виндчейзер.~--- И что же случилось сейчас? Ещё один досадный инцидент на тренировке?

--- Нет. Нарочно грохнула. И тогда тоже.

--- И как такое оказалось возможным сегодня, после принятых в тот раз мер? Я знаю только одного пегаса, который смог бы проделать этот трюк, но тебе до него пока далеко.

--- Мне помогли.

Сзади послышалось шевеление: Старлайт и Свити сделали полшага вперёд.

--- Вы понимаете, в чём сейчас признались? Организация и~проведение атаки на дворец с целью причинения ущерба.

--- А кто организовывал то, что было изображено здесь?~--- Скуталу мотнула головой в сторону пустого оконного проёма, окаймлённого осколками.~--- Так называемое нашествие чейнджлингов на Кантерлот?

--- Допустим. Во втором случае~--- допустим. Но если вернуться к первому случаю?.. У пословицы «Кто старое помянет, тому глаз вон» есть окончание, о котором многие почему-то предпочитают умалчивать: «А кто забудет~--- тому два!».

--- Про тех, кто всё время тычет этим старым в глаза другим, там ничего не говорится?

--- Твои слова~--- большая дерзость!

--- «Жил как умел, а иначе не вышло{\ldots}»~--- процитировала Скуталу. Сзади опять послышалось шевеление.

--- Я помню, кто написал эти стихи,~--- сухо заметила принцесса.~--- Что ж, попробуем подойти с другой стороны. Ты~--- я сейчас обращаюсь именно к тебе, как главному исполнителю~--- не думала о том, что эти витражи были созданы трудом лучших мастеров, потративших на это недели?

--- Тьфу на такое мастерство. Которым унижают.

--- Они были посвящены не унижению~--- заслугам. В том числе ваших старших сестёр. Сможете смотреть им в глаза?

--- Как-нить посмотрим, они ж на то и сёстры. Главное, шо наставникам точно сможем{\ldots}~--- теперь полшага вперёд сделала Эппл Блум.~--- И Галлусу, и другим.

--- Достойные ученицы, это я без иронии{\ldots} Могу понять ваши чувства, но давайте теперь посмотрим на всё с позиции разума. Всем вам, здесь присутствующим, прекрасно известно, что так называемое нашествие на Кантерлот было частью некоего плана. Следовательно, вы должны понимать и то, что обсуждаемое изображение предназначалось для его поддержки.

--- Есть такое слово~--- пропаганда.

--- Именно. С изменением ситуации меняется и пропаганда, а ситуация меняется{\ldots} и вы, призванные помочь с её изменением, не станете с этим спорить. Разве трудно было понять, что после первого подходящего повода этот витраж сменился бы на что-то другое?

--- Ну так вот же он, повод!~--- фыркнула Скуталу, указав крылом.~--- Пожалуйста, меняйте на чего подходящее. Теперь-то всяко менять придётся.

--- А просто поговорить со мной об этом было нельзя? Неужели, по-вашему, я не прислушалась бы к разумному слову?

--- Нельзя.

--- Почему же?

--- Про нашествие слышали все. Значит, надо, чтобы и про это тоже. Ну, не все, так хоть многие.

Эппл Блум сделала ещё полшага, будто невзначай качнув висящим на шее фотоаппаратом. Сообщила в пространство:

--- Фотик от засветки зачарован{\ldots}

Селестия шумно вздохнула.

--- Начинаю понимать,~--- заметила она,~--- почему Твайлайт сделала вас почётными преподавателями Школы Дружбы. Глядя на то, как вы сейчас пытаетесь преподать соответствующий урок \emph{мне}{\ldots} Но ваша выходка требует наказания, и вы трое \emph{будете} наказаны.

Свити Белль и Эппл Блум сделали ещё несколько шагов вперёд, встав рядом со Скуталу.

--- Когда-то давно я получила ещё один очень важный урок. Про то, какими опасными могут быть собственные слова, превращённые в ловушку, и вы должны знать эту историю. Настала пора применить урок на практике. Ты сказала, что даёшь мне повод поменять один из витражей на более подходящий? Быть по сему. Витраж, который займёт место разбитого тобой, будет изображать вас троих!

--- Э-э-э-э!..~--- взвыла троица Меткоискателей.~--- Так нечестно!!!

--- Старлайт?..~--- произнёс Виндчейзер, перекрыв своим голосом это возмущение.

--- М-м-м?

Он описал копытом некую округлую замкнутую кривую на полу, затем махнул им сверху вниз, будто кидая что-то в этот круг.

--- Уверен?

--- Да.

Старлайт пожала плечами и засветила рог. На пол из ниоткуда хлынул небольшой водопадик.

Виндчейзер демонстративно повернулся задом к Солнечному трону и уселся в образовавшуюся лужу.

--- И только попробуй сказать, --- заорала Скуталу на двадцать процентов громче,~--- что меня больше нечему учить!!!


\chapter*{Интерлюдия, в которой двое игроков пьют чай и говорят о Меткоискателях}\addcontentsline{toc}{chapter}{Интерлюдия, в которой двое игроков пьют чай и говорят о~Меткоискателях}

--- Многая лета солнечному сиятельству, да не запятнается оное новыми конфузами. Зван бых и приидох.

--- Проходи, садись,~--- вздохнула Селестия, отодвигая телекинезом стул.~--- Угощайся, если хочешь{\ldots} выпечка не моя, заказывала в лучшей кондитерской. Чай, кофе?

--- Спасибо. Лучше чай~--- я слышал, в нём ты действительно знаешь толк.

Под носик чайника подплыло облачко какого-то заклинания, и струйка полилась в чашки через него.

--- Ты правильно слышал, по части кофе мне до сестры далеко. Возможно, тебя позабавит тот факт, что она научила Твайлайт, а вот мне раскрывать свои секреты из вредности отказывается{\ldots} Угодила?

Виндчейзер отпил и с уважением кивнул:

--- Да, более чем. Прекрасный чай.

--- Если не возражаешь, совместим чаепитие с разговором. Когда я приглашала тебя свободно приходить во дворец, то сказала, что без явной необходимости не буду навязывать тебе своё общение{\ldots}

--- Помню. Очевидно, такая необходимость возникла?

--- Да. Вчера я позволила себе сказать лишнее в твоём присутствии и очень об этом сожалею.

--- Прости, не понимаю. Поясни, пожалуйста.

--- Я процитировала пословицу. «Кто старое помянет, тому глаз вон, а кто забудет~--- тому два!». С моей стороны это было очень глупо. Говорить такое при тебе{\ldots} после того нашего разговора о примирении и прощении{\ldots} Правда, я сказала это не в~твой адрес, а твоей ученице, будучи изрядно разозлённой её выходкой, но всё равно. Будь такое возможно, я сейчас сделала бы всё, чтобы эти слова вчера не прозвучали.

--- То есть ты хочешь взять эти слова обратно?

--- Да. Более того, я ещё и рявкнула на Скуталу после того, как она вполне справедливо, хоть и не особенно вежливо, заметила мне, что постоянно тыкать прошлым в глаза другим не есть хорошая идея. О чём тоже сожалею.

Виндчейзер пожал плечами, чуть разведя крыльями:

--- На самом-то деле ничего не случилось. Процитированная тобой пословица{\ldots} не буду скрывать, она вполне согласуется с моими собственными принципами. Сказав это, ты никак не оскорбила и ничем не спровоцировала меня.

--- Никто не забыт и ничто не забыто?

--- Да. Другое дело, что мы пережили это, и это не должно мешать нам жить дальше. По-моему, всё уже было сказано. С тех пор я не попрекал тебя прошлым, и после вчерашнего тоже не собираюсь попрекать.

--- Спасибо. Если хочешь, в качестве ответного жеста я откажусь от своей идеи с заменой витража.

--- Зачем?! Вот уж это было придумано просто замечательно. Они во всех смыслах заслужили такое наказание.

--- Надо же!~--- Селестия хмыкнула.~--- Ты не просто согласился с моим решением, а ещё и одобрил его{\ldots} Позволь полюбопытствовать: а ты смог бы в одиночку проделать то, что вчера учинили они?

--- Посреди бела дня прорваться в тронный зал через один из витражей? Смог бы{\ldots} при помощи обстоятельств. Дождавшись подходящего облака на нужной высоте в нужном месте и в нужное время. Но этот вариант потерял актуальность, ведь после вчерашнего уже наверняка приняты дополнительные меры. Тебе придётся подождать несколько дней, если хочешь услышать ответ про нынешние условия.

--- Пожалуйста, сообщи мне, когда придёшь к каким-то выводам. Говоря о витражах{\ldots} а какую сцену стоило бы изобразить на новом?

--- Забавно. Вообще-то я хотел задать тот же самый вопрос тебе.

--- Ну, всё-таки ты знаешь эту троицу гораздо лучше меня.

--- Извини, ответа не будет. Я согласен с твоей идеей и даже одобряю её, но сам в этом наказании участвовать не хочу.

--- Потому что в равной мере одобряешь сделанное ими?

--- Конечно.

--- Ты поэтому и уселся в лужу назло мне?

--- Не только. Скуталу выросла{\ldots} я никогда не скажу ей, что больше ничему не смогу её научить, но она пережила детскую подначку, на которой я смог заставить её взлететь. В последнее время она так искала моего одобрения и подтверждения, что с~ней возились не зря{\ldots} я бы даже сказал, яростно искала. Первый витраж она разбила именно поэтому, а вместо одобрения получила разнос: выпендрёж не заслуживает похвал. Но вчерашнее было по-настоящему и не ради выпендрёжа.

--- Знаешь, а ведь тебе удалось меня уязвить своей демонстрацией.

--- И в мыслях не было.

--- Верю. Но ты не постеснялся сам усесться в лужу перед ученицей с её подругами и без всяких слов выразить им своё уважение в моём присутствии.

--- Да. И что тебя уязвило?

--- Так{\ldots} Вспомнила Твайлайт и те интриги, что я накрутила вокруг неё.

--- Если тебя интересует моё мнение{\ldots}

--- Интересует.

--- Нет ничего недостойного в интригах. Но недоверие к ученику или ученице{\ldots} Ты зря так долго скрывала обстоятельства поступления в Школу, это очень сильно уязвило \emph{её}. Если у тебя есть ещё одна подобная тайна{\ldots}

--- Честное слово, больше нету. Но должна признать, она очень достойно поквиталась со мной за поступление. Я потом две луны места себе не находила от сомнений.

--- Та записка насчёт «время не перехитришь»? Это был я. Считай, я так отомстил за попытку шантажировать её мной.

--- Я уже говорила, что больше не буду ни при каких условиях. Но раз уж между нами прозвучало слово «интриги», я хотела бы спросить ещё.

--- Слушаю тебя.

--- Риторический вопрос: ты, конечно, читал «Очень Большую Игру» и знаешь смысл, вложенный в это словосочетание{\ldots}

--- Действительно, риторический. Да, оба раза.

--- Следует ли понимать, что ты вступил в Игру?

--- Я вступил в неё ещё до того момента, как появился на свет.

--- Прости, не понимаю.

--- Не попрекая тебя, но отвечая на твой вопрос{\ldots} что ты знаешь об обстоятельствах моего знакомства с Твайлайт? Она тогда написала тебе какое-то письмо про меня.

--- Нет, про это там ничего не было. Она просто поставила меня перед фактом твоего существования и высказала{\ldots} неважно. Вспомни, я же спрашивала тебя, как давно вы знакомы.

--- На момент знакомства я был убеждён, что был нужен родителям в качестве последнего шанса. Что они надеялись, воспитывая меня и заботясь обо мне, продержаться ещё сколько-то лет и не сойти с ума. Теперь я \emph{уверен} в этом. Так что да~--- я давно в Игре.

--- Это не такой ответ, какой я бы хотела услышать.

--- Это такой ответ, какой я хотел дать.

--- Звучит очень знакомо. Хорошо, попробую по-другому. Следует ли понимать, что ты в Игре на стороне грифонов и~чейнджлингов?

--- Я хочу, чтобы выиграли все. В том числе и они.

Правительница шумно вздохнула:

--- Ты можешь пообещать мне ответить на один{\ldots} нет, два моих следующих вопроса однозначно, просто «да» или «нет»?

--- Могу пообещать, что на любое количество твоих вопросов я не скажу ни слова лжи.

--- Ты издеваешься?

--- Нет.

--- Но явно получаешь удовольствие от этой игры словами!

--- Да. Обрати внимание, я только что в точности выполнил твою просьбу и сделал именно то, что ты просила. Два прямых однозначных ответа на два вопроса.

--- Как только Твайлайт тебя выносит{\ldots}

--- Иногда с трудом.

--- {\ldots}и есть ли вообще кто-нибудь, кто может выноси́ть тебя без труда? Кроме моей сестры, разумеется.

--- Трикси, например.

Принцесса хмыкнула. Отпила чай, откусила пирожное.

--- Я всё-таки спрошу{\ldots} ответь уж как хочешь, просто ответь. Скажи, твои симпатии к чейнджлингам и грифонам{\ldots} они в пику мне?

--- Что?!~--- Виндчейзер энергично помотал головой.~--- Совсем нет! Я говорю по-грифоньи{\ldots} не упомню даже с каких лет, но это началось уж точно до того, как я вообще впервые узнал о~твоём существовании. Слушал их стихи и сказки. Прожил среди них почти весь свой восьмой год{\ldots} многому научился. Как, по-твоему, я могу им не симпатизировать?!

--- А чейнджлинги?

--- Они интересный народ. У меня с ними есть кое-что общее, я имею в виду умение работать с эмоциями. Кроме того{\ldots} впрочем, я не уверен, что это не прозвучит как упрёк{\ldots}~--- Селестия отмахнулась и покивала.~--- Меня в них кое-что восхищает. Будучи по сути паразитами, они сумели отлично ужиться с нами в Эквестрии, никому не мешая. Хоть и не без помощи. Как раз полтысячи лет назад случился самый настоящий бум традиций праздновать всё, что только можно{\ldots} это ведь явно не простое совпадение?

--- Да,~--- вздохнула Селестия.~--- Это была наша договорённость: я обеспечила им возможность подпитываться положительными эмоциями, они держали своё слово этой возможностью не злоупотреблять. Какой-никакой опыт с Ночью Кошмаров у меня уже был{\ldots} в итоге выиграли все. Жить-то стало веселее, как бы банально это ни звучало. Но я догадываюсь, к чему ты клонишь: пять веков пони и чейнджлинги прекрасно уживались, а потом я в одночасье всё разрушила этим нашествием на свадьбу{\ldots}

--- Этого я не говорил и никогда не скажу. Ни секунды не сомневаюсь, что та затея была устроена без всяких задних мыслей. Правда, это не отменяет{\ldots} э-э{\ldots} скажем так, излишней хитровывернутости само́й затеи. Но наверняка у тебя были существенные соображения на этот счёт.

--- Да. С грядущим возвращением Кристальной Империи нужно было что-то делать, и я хотела проверить намеченную для этого пару. Заодно чейнджлинги объявились бы так, что не пришлось ничего объяснять. Ещё полгода-год, потом они бы преобразились и предстали уже не врагами, но друзьями. Казалось, что́ могло пойти не так{\ldots}

--- {\ldots}а оказалось, что многое. То-то она тебе потом писала, что вы никудышные актёры, да и как драматурги немногим лучше. Но ты сказала «полгода-год», а между вашими Пьесами прошло почти четыре.

--- Провал первой заставил нас утроить осторожность.

--- Однако ты всё равно отправила свою пару в Кристальную Империю против Сомбры.

--- А что ещё было делать? В отсутствие других кандидатур{\ldots} К тому же Пьеса показала, что при поддержке Твайлайт с подругами у них вполне получается{\ldots} получилось и там. Скажи лучше вот что. Насколько, по-твоему, убедительным вышло то «нашествие»? Понимаю, что тебя-то оно напугало вряд ли.

--- Меня тогда не было в Кантерлоте. Но когда мне рассказывали о нём здешние грифоны{\ldots} если называть вещи своими именами, то им стоило большого труда не лопнуть со смеху. «Нападавшие» действовали настолько глупо, что это сильно бросалось в глаза.

--- Вот как? И почему же никто из них не сказал потом ни слова про эту показушность?

--- Опять-таки, не желая тебя обидеть{\ldots} Для них это выглядело как какое-то дурацкое шоу, сопровождавшее свадьбу. Ну, угодно этим чокнутым поням развлекаться таким вот образом, так и пусть себе развлекаются. Помнится, Твайлайт задавала пятерым ученикам из других народов написать сочинение про эквестрийские праздники и их восприятие{\ldots}

Селестия одновременно поморщилась и хихикнула:

--- Это когда они предпочли остаться на каникулы, но не расставаться? Видела. Да{\ldots} пожалуй, это объясняет. Но да будет тебе известно, жители города были напуганы по-настоящему, и очень серьёзно.

--- Тем больше причин по-настоящему дружить с грифонами. У которых можно научиться не бояться такой ерунды.

--- Я уже говорила, что с твоим языком ты мог бы стать великим дипломатом. Не нужно отвечать, я помню, что́ ты на это сказал{\ldots} У меня остался один вопрос: не боишься?

--- Чего? Заиграться как ты?

--- Нет. Сделать хуже.

--- Не боюсь. Ситуация такая, что все хотят перемен к лучшему, и сейчас для таких перемен самое время. Если прямо назвать и признать прошлые ошибки, чтобы они больше не повторялись, то может стать только лучше.

--- «Все хотят»{\ldots} все ли? Грифоны и чейнджлинги хотят, а пони?

--- А кто вчера разбил витраж, очернявший чейнджлингов? А~такое имя, как Сэндбар, тебе что-нибудь говорит? А если я назову тебе дюжину своих коллег-курьеров, много лет помогающих молодым грифонам устраиваться на работу?

--- Достаточно. Я поняла тебя.

--- Кто знако́м с ситуацией не понаслышке --- те хотят, эксперимент со Школой Дружбы подтверждает. Твайлайт умница.

--- Да, она умница{\ldots} Скажи, а такое понятие, как кооперативная игра,\footnote{Кооперативные игры устроены так, что игроки должны действовать совместно против игровой механики и \textit{сообща} достигнуть какой--то цели.} тебе знакомо?

--- Конечно. Это именно то, на что я хотел тебе намекнуть. Можно играть вместе, даже не испытывая друг к другу особо тёплых чувств.

--- Я стараюсь, --- слабо улыбнулась Селестия. --- Но всё-таки{\ldots} постарайся и ты. Не просто помнить о совершённых ошибках, а~делать из них выводы, ладно? Хотя бы ради Твайлайт, относительно которой я ошибалась несколько раз. Пожалуйста.

--- Понимаю и буду.

--- Тогда ещё одна просьба. Твоя лекция в Школе подтолкнула меня. Вчерашний демарш этих{\ldots} Меткоискателей{\ldots} показал, что всё серьёзно. Сегодняшнее чаепитие подтвердило сей факт. Могу я рассчитывать{\ldots} скажем, на неделю спокойствия? Без лекций, демаршей и несогласованных ходов? Просто чтобы как следует обдумать дальнейшее.

--- С моей стороны~--- можешь. Со стороны своей подруги ко{\ldots} Квик Чаптер~--- тебе виднее, но думаю, тоже. Твайлайт в последнее время очень осторожна, особенно после того понивилльского эксперимента. А что касается троицы на-свои-задницы-искателей{\ldots}

--- Вот именно! Со Старлайт Глиммер они спелись! Тётя Кризалис~--- заинтересованная сторона! А тебя они хоть побаиваются.

--- Пока ты не огласишь своё решение относительно нового витража{\ldots} думаю, они подожмут хвосты. Хотя бы на несколько дней,~--- Виндчейзер усмехнулся.~--- Но это не точно.


\backmatter

\appendix

\cleardoublepage
\thispagestyle{empty}

\mbox{}

\vspace{.3\textheight}
\begin{center}
	\Huge{ПРИЛОЖЕНИЯ}
\end{center}


\chapter*{Хронология}\addcontentsline{toc}{chapter}{ПРИЛОЖЕНИЕ. Хронология}


\newlength{\ado}
\ado=2.15mm

\begin{tabbing}
	≈ 796 БС = 205 до ВНМ\hspace{1em}\=\kill
	\textbf{БС} \> Битва Сестёр и упадок Гармонии\\[\ado]
	≈ 450 БС ≈ 550 до ВНМ \> Появление роя чейнджлингов в Эквестрии\\[\ado]
	796 БС = 205 до ВНМ \> Рождение Виндчейзера\\[\ado]
	22 до ВНМ \> Рождение Твайлайт и Старлайт\\[\ado]
	7 до ВНМ \> Рождение Меткоискателей\\[\ado]
	\textbf{ВНМ} \> Возвращение Найтмэр Мун\\*
	  \> Возрождение Элементов Гармонии\\[\ado]
	2 ВНМ \> Нашествие чейнджлингов на Кантерлот\\[\ado]
	3 ВНМ \> Возвращение Кристальной Империи\\*
	       \> Ученичество Скуталу\\[\ado]
	4 ВНМ \> Вознесение Твайлайт\\[\ado]
	6 ВНМ \> Преображение чейнджлингов\\[\ado]
	7 ВНМ \> Ученичество Свити Белль\\[\ado]
	8 ВНМ \> Открытие Школы Дружбы\\*
	      \> Великий Понивилльский Наноэксперимент\\[\ado]
	9 ВНМ \> Лекция Виндчейзера в Школе\\*
	      \> Ученичество Эппл Блум\\*
	      \> Большое Откровение\\*
	      \> Публикация «Очень Большой Игры»
\end{tabbing} 


\chapter*{«Распределение» (соч. Квик Чаптер)}\addcontentsline{toc}{chapter}{ПРИЛОЖЕНИЕ. «Распределение» (соч. Квик Чаптер)}\label{raspredelenie}

--- Давайте, что там у вас{\ldots} --- Селестия перехватила своим телекинезом стопку бумаг, которые принёс ей проректор, и со вздохом опустила на стол.~--- Сейчас разберу. И пока я буду разбирать{\ldots} Знаете что, подготовьте-ка мне, пожалуйста, материалы по самому перспективному выпускнику Школы.

--- Нет необходимости готовить,~--- проректор пожал плечами и с хлопком телепортировал туда же на стол папку личного дела.~--- Твайлайт Спаркл. Третий год на постдоке,\footnote{Постдок --- молодой учёный, только что получивший степень и начинающий самостоятельные исследования при поддержке учебного заведения. Также название соответствующей программы поддержки молодых учёных.} заявленный объём исследовательской работы по факту выполнен ещё к концу первого.

--- Какие курсы ведёт? Заменить её реально?

--- Ничего не ведёт.

--- Как так?! Вы же говорите, самая перспективная?

--- Она, ваше высочество, из таких, которым проще триста раз самим что-то сделать, чем кого другого научить, и это только во-первых{\ldots}

--- А во-вторых?

--- Во-вторых, она от нагрузки шарахается, и нагрузка от неё тоже{\ldots} в смысле, шарахаются студенты. Таких обычно не держим, но ведь действительно перспективная, ничего не скажешь.

--- Прекрасно, прекрасно{\ldots}~--- рассеянно отозвалась принцесса, просматривая бумаги и черкая резолюции.~--- Не ведёт, так и заменять не понадобится{\ldots} Готовьте приказ. Отчислить досрочно в связи с успешным окончанием и направить на стажировку.

--- Как обычно? Библиотекарем с заданием искать древние книги?

--- Конечно, а зачем изобретать что-то новое.

--- И куда?

--- Да хоть в этот{\ldots} как его{\ldots}~--- Селестия постучала краешком копыта по столу, вспоминая.~--- В Понивилль. Там же как раз большой праздник на носу, всё веселее ей будет переезжать, хоть не со скуки новая жизнь начнётся.

Проректор хмыкнул. Искать древние книги в единственной библиотеке городишка-новостроя, где всё ещё живёт и~здравствует поколение его основателей{\ldots} весёлое начало новой жизни, ничего не скажешь!

--- Раз уж вы упомянули праздник,~--- заметил он, вытягивая из папки личного дела бумажку,~--- то в этой связи должен сказать, что означенная Спаркл пыталась через канцелярию Школы отправить вам меморандум. Это касательно пророчества о тысячелетнем сроке изгнания{\ldots} э{\ldots} ну, вы понимаете. Я~позволил себе придержать сей документ.

--- И правильно сделали{\ldots}~--- принцесса не глядя потянулась пером и черкнула что-то в уголке меморандума.~--- Передайте ей вместе с копией приказа и скажите, что я была восхищена её аналитическим умом.

К некоторому удивлению проректора, резолюция гласила: «Одобряю, выполняйте. С.»

--- Но, ваше высочество,~--- осторожно заметил он,~--- там же нет никаких конкретных предложений. Это именно меморандум, не более.

--- Вот и пусть проявит творческий подход! Зря, что ли, вы мне её тут рекомендовали как перспективную? Ну{\ldots} в конце концов, дайте ей контакты оргкомитета праздника и скажите на словах, что я возлагаю на неё общее копытоводство с целью недопущения{\ldots} и всё такое. Так, вроде с текучкой закончено{\ldots}

Селестия аккуратно отодвинула на край стола просмотренные и подписанные документы, а затем широким взмахом копыта отправила в мусорную корзину папку личного дела. На её месте с хлопком появилась другая~--- точно такая же, только с иным именем на обложке. Принцесса сдула с папки записку «Очень Вас прошу, Ваше Высочество!», и та на лету дематериализовалась.

--- Зачисляйте на освободившееся место,~--- распорядилась она, подталкивая папку к проректору.



\end{document}
